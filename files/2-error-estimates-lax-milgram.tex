% !TeX source = ../main.tex

\chapter[A priori error estimates (Lax-Milgram case)]{A priori error estimates (Lax-Milgram case)}

\section{Scaling argument for Sobolev norms}

%%%%%
% Lecture of 20 of March
%%%%%
\begin{figure}[!ht]
\centering
\includegraphics{affine_mapping}
\caption{A triangle $T_m$ and its reference triangle $\hat{T}$. The affine mapping $F_m$ maps $\hat{T}$ to $T_m$.}
\label{fig:affine_mapping}
\end{figure}

Consider an affine mapping from a reference triangle $\hat T$, to a generic triangle $T_m$ (Figure~\ref{fig:affine_mapping}):
\begin{align}
F_m: &\hat{T} \to T_m \\
&\hat{x} \mapsto x = B \hat{x} + b.
\end{align}
We would like to analyze how Sobolev norms change under the action of $F_m$. Recall the definition of the Sobolev norm in $W^{k,p}$:
\[
\norm{v}^p_{k,p,\Omega} = \sum_{\abs{\alpha} \le k} \norm{D^\alpha v}^p_{L^p(\Omega)},
\]
with the corresponding seminorm:
\[
\abs{v}^p_{k,p,\Omega} = \sum_{\abs{\alpha} = k} \norm{D^\alpha v}^p_{L^p(\Omega)}.
\]

Let
\[
\rho_m := \sup_{\tilde{B} \subset T_m} \diam(\tilde{B}), \quad h_m := \inf_{T_m \subset \tilde{B}} \diam(\tilde{B}),
\]
and similarly for the reference triangle $\hat{T}$:
\[
\hat{\rho} := \sup_{\tilde{B} \subset \hat{T}} \diam(\tilde{B}), \quad \hat{h} := \inf_{\hat{T} \subset \tilde{B}} \diam(\tilde{B}),
\]
where $\tilde{B}$ denotes a ball in the right space (see Figure~\ref{fig:scaling}). We emphasize that both $\hat{\rho}$ and $\hat{h}$ are \emph{constants}.


\begin{figure}[!ht]
\centering
\includegraphics{triangle_scales}
\caption{The scales $\rho_m$ and $h_m$ for a triangle $T_m$, and $\hat \rho$, $\hat h$ for the reference triangle $\hat T$.}
\label{fig:scaling}
\end{figure}


We start with some considerations on the matrix $B$. First, we notice that if $x,y \in T_m$ we have
\[
\abs{x-y} = \abs{B(\hat{x}-\hat{y})} = \abs{B \xi}, \qquad \xi = \hat{x}-\hat{y}.
\]
By the definition of the norm of B, we have that
\begin{equation}
\norm{B} = \sup_{\abs{\xi} = \hat{\rho}} \frac{\abs{B \xi}}{\abs{\xi}} \le
\sup_{\abs{\xi} = \hat{\rho}} \frac{\abs{x-y}}{\hat{\rho}} \le \frac{h_m}{\hat{\rho}}.
\label{eqn:scaling_B}
\end{equation}
With a similar argument:
\begin{equation}
  \norm{B^{-1}} = \sup_{\abs{\xi} = \rho_m} \frac{\abs{B^{-1} \xi}}{\abs{\xi}} \le \frac{\hat{h}}{\rho_m},
  \label{eqn:scaling_B_inv}
\end{equation}
where both $\hat \rho$ and $\hat h$ are constant, and depend only on the choice of the reference triangle $\hat T$.
Also, since $F_m$ is affine, its differential is the matrix $B$ itself. This will come in handy to provide a simple formula for $\hat{D}^\alpha (v \circ F_m)$ when using the chain rule.

Consider a function $v \in W^{k,p}(T_m)$ on the triangle $T_m$. Then:
\begin{align}
\abs{v \circ F_m}^p_{k,p,\hat{T}} 
&= \sum_{\abs{\alpha} = k} \int_{\hat{T}} \abs{ \hat{D}^\alpha (v \circ F_m)}^p \diff\hat{T} \\
&= \sum_{\abs{\alpha} = k} \int_{\hat{T}} \abs{ [(D^\alpha v) \circ F_m)] B^{\abs{\alpha}}}^p \diff\hat{T} \\
&= \sum_{\abs{\alpha} = k} \int_{\hat{T}} \abs{ [(D^\alpha v) \circ F_m)] B^{\abs{\alpha}}}^p J^{-1} J \diff\hat{T} \\
&= \sum_{\abs{\alpha} = k} \int_{T} \abs{ D^\alpha v B^{\abs{\alpha}}}^p J^{-1} \diff T \\
&\le \norm{B}^{kp} J^{-1} \sum_{\abs{\alpha} = k} \int_{T} \abs{ D^\alpha v}^p \diff T,
\end{align}
where $J$ is the absolute value of the Jacobian of the transformation $F_m$ (in the present case, $J=\abs{\det B}$). Hence, using~\eqref{eqn:scaling_B}, we conclude:
\[
\abs{v \circ F_m}_{k,p,\hat{T}} \le \norm{B}^k J^{-\frac{1}{p}} \abs{v}_{k,p,T}
\le c h_m^k J^{-\frac{1}{p}} \abs{v}_{k,p,T},
\]
and similarly, using~\eqref{eqn:scaling_B_inv}:
\[
\abs{v}_{k,p,T} \le \norm{B^{-1}}^k J^{\frac{1}{p}} \abs{v \circ F_m}_{k,p,\hat{T}}
\le c \rho_m^{-k} J^{\frac{1}{p}} \abs{v \circ F_m}_{k,p,\hat{T}},
\]
where $c$ is a constant depending on the reference triangle $\hat{T}$, but not on the triangle $T_m$.

For $h_m \le 1$, we can write similar scaling arguments for the full norms:
\begin{align}
    \norm{v \circ F_m}_{k,p,\hat{T}} & \le c h_m^k J^{-\frac{1}{p}} \norm{v}_{k,p,T},\\
    \norm{v}_{k,p,T} & \le c \rho_m^{-k} J^{\frac{1}{p}} \norm{v \circ F_m}_{k,p,\hat{T}}.
\end{align}

From now on we adopt this notation:
\begin{itemize}
\item $a \lesssim b$ means "$\exists c \text{ s.t. } a \le cb$";
\item $a \gtrsim b$ means "$\exists c \text{ s.t. } ca \ge b$";
\item $a \sim b$ means "$a \lesssim b \wedge b \lesssim a$".
\end{itemize}
For example, if two norms $\norm{\cdot}_X$ and $\norm{\cdot}_Y$ are equivalent, the following statements share the same meaning:
\[
\norm{a}_X \sim \norm{a}_Y \iff \exists c,C \text{ s.t. } c\norm{a}_X \le \norm{a}_Y \le C \norm{a}_X.
\]


\section{Bramble-Hilbert lemma}

\begin{lemma}[Bramble-Hilbert]
Let $V$, $W$, $Q$ be Banach spaces and let $\tau \in \L(V,W)$ be an operator such that $Q \subset \ker(\tau)$. Then:
\begin{enumerate}
\item $\norm{\tau(u)}_W \le \norm{\tau}_* \inf_{q \in Q} \norm{u-q}_V \quad \forall u\in V$.
\item Choosing $V=W^{k+1,p}(\Omega)$, $W=W^{s,p}(\Omega)$, $Q=\P^k(\Omega)$, with $\Omega$ open connected (and bounded Lipschitz) and $0\le s \le k$, we get
\[
\norm{\tau(u)}_{s,p,\Omega} \lesssim \norm{\tau}_* \abs{u}_{k+1,p,\Omega} \quad \forall u\in W^{k+1,p}(\Omega).
\]
\end{enumerate}
\label{lemma:bramble-hilbert}
\end{lemma}

\begin{proof}
The first inequality follows immediately by observing that $\tau$ is linear and that $q\in \ker(\tau)$ implies $\tau(u+q) = \tau(u) + \tau(q) = \tau(u)$, hence:
\[
\norm{\tau(u)}_W = \norm{\tau(u+q)}_W = \le \norm{\tau}_* \norm{u+q}_V \qquad \forall q \in Q.
\]

The second inequality follow from by showing that~\cite{ciarlet78}
\begin{lemma}[Denis-Lions]
  \begin{equation}
    \abs{u}_{k+1,p,\Omega} \sim \inf_{q \in \P^k} \norm{u+q}_{k+1,p,\Omega} \qquad \forall u \in W^{k+1,p}(\Omega).
    \label{eqn:deny-lions}
  \end{equation}
\end{lemma}

From now on, we'll omit $\Omega$ for simplicity where its presence is obvious.
\begin{itemize}
\item The inequality $\abs{u}_{k+1,p} \lesssim \inf_{q \in \P^k} \norm{u+q}_{k+1,p}$ is trivial, because every polynomial of degree at most $k$ has zero derivatives of $(k+1)$-th order. Hence
\[
\abs{u}_{k+1,p} = \abs{u+q}_{k+1,p} \le \norm{u+q}_{k+1,p} \quad \forall q \in \P^k.
\]
\item In order to prove that the converse holds, let $\{v_i\}_{i=0}^N$ be a basis for $\P^k$. Then, with the usual notation, $(\P^k)'=\Span\{v^i\}_{i=0}^N$, where the $v^i$-s are such that $v^i(v_j)=\delta_{ij}$. Since $\P^k \subset W^{k+1,p}$, by the Hahn-Banach theorem we can naturally extend these $v^i$-s to be elements of $(W^{k+1,p})'$. Hence we can consider the projection
\begin{align}
\Pi^k:  W^{k+1,p}(\Omega) & \to \P^k(\Omega)\\
u & \mapsto \sum_{i=0}^N v^i(u) v_i.
\end{align}
The key part is proving that
\begin{equation}\label{eqn:deny-lions-exp}
\norm{u}_{k+1,p}^p \sim \abs{u}_{k+1,p}^p + \sum_{i=0}^N \abs{v^i(u)}^p \quad \forall u \in W^{k+1,p}.
\end{equation}
In fact, the $\gtrsim$ inequality follows from bounding the Sobolev seminorm with the corresponding norm and by the simple inequality
\[
\abs{v^i(u)} \le \norm{v^i}_{(W^{k+1,p})'} \norm{u}_{k+1,p}.
\]
The $\lesssim$ inequality, instead, can be proven by contradiction. If it were not true, then for every constant $c$ there would exist $w_c \in W^{k+1,p}$ such that $\norm{w_c}_{k+1,p} = 1$ and
\[
c\big(\abs{w_c}_{k+1,p}^p + \sum_{i=0}^N \abs{v^i(w_c)}^p\big) < \norm{w_c}_{k+1,p}^p = 1.
\]
If we choose $c=j$ in $[1,2,\dots)$, we obtain a sequence $\{w_j\}$ such that
\begin{equation} \label{eqn:deny-lions-proof}
\abs{w_j}_{k+1,p}^p + \sum_{i=0}^N \abs{v^i(w_j)}^p < \frac{1}{j} \quad \forall j.
\end{equation}
Now, thanks to the compact embedding $W^{k+1,p} \hookrightarrow W^{k,p}$, we have, up to a subsequence, that $w_j \to w$ strongly in $W^{k,p}$ for some $w \in W^{k,p}$.

However, we can easily show that the function $w$ belongs also to $W^{k+1,p}$. In fact, from~\eqref{eqn:deny-lions-proof} we deduce
\[
\abs{w_j}_{k+1,p}^p < \frac{1}{j} \quad \forall j.
\]
This implies that the sequence $\{w_j\}$ is Cauchy also in $W^{k+1,p}$, because
\[
\norm{w_i-w_j}_{k+1,p}^p = \norm{w_i-w_j}_{k,p}^p + \abs{w_i-w_j}_{k+1,p}^p.
\]
Hence, by uniqueness of the limit, $w \in W^{k+1,p}$. Then, by (strong) continuity of the norm, we obtain $\norm{w}_{k+1,p}=1$ and we can pass to the limit the inequality~\eqref{eqn:deny-lions-proof} to get:
\[
\abs{w}_{k+1,p}^p = 0, \quad \sum_{i=0}^N \abs{v^i(w)}^p = 0.
\]
The first equality, since $\Omega$ is an open connected domain, implies that $w \in \P^k(\Omega)$. The second equality implies that $\Pi^k(w)=0$. But for a polynomial in $\P^k$ we have $\Pi^k(w)=w$, hence $w=0$, which is in contradiction with $\norm{w}_{k+1,p}=1$.

We can now conclude, exploiting inequality~\eqref{eqn:deny-lions-exp}, by noticing that
\begin{align}
\inf_{q \in \P^k} \norm{u+q}_{k+1,p} &\le \norm{u- \Pi^k(u)}_{k+1,p} \\
& \lesssim \abs{u-\Pi^k(u)}_{k+1,p}^p + \sum_{i=0}^N \abs{v^i(u-\Pi^k(u))}^p \\
& = \abs{u}_{k+1,p}^p.
\end{align}
where the last equality holds because the summation term is equal to zero by linearity of the $v^i$-s and the seminorm is simplified due to $\Pi^k(u)$ being a polynomial in $\P^k$.
\end{itemize}
\end{proof}

Bramble-Hilbert lemma allows us to provide a-priori error estimates for the interpolation operator of a finite element space, by quantifying the constants in the inequality throught the scaling arguments presented in the previous section:
\begin{theorem}[Interpolation error]
  \label{theo:interpolation_error}
  Let $u\in W^{k+1,p}(\Omega)$, and let
\[
h = \max_{T \in \T_h} h_T, \quad \rho = \min_{T \in \T_h} \rho_T, \quad \Omega = \mathring{\overline{\bigcup_{T \in \T_h} T}}.
\]

Then the interpolation operator $\Pi^k$ satisfies:
\begin{equation}
  \label{eqn:interpolation_error}
\Bigl( \sum_{T \in \T_h} \norm{u- \Pi^k(u)}_{s,p,T}^p \Bigr)^\frac{1}{p} \lesssim
h^{\ell+1} \rho^{-s} \abs{u}_{\ell+1,p,\Omega} \qquad 0 \le s \le \ell \leq k,
\end{equation}
where $k$ indicates the degree of the local polynomial spaces $P^k(T)$ of the finite elements.
\end{theorem}

\begin{proof}
Consider an affine mapping $F_T: \hat{T} \to T$ and its related quantities $h_T$ and $\rho_T$. 

We can use Bramble-Hilbert lemma applied to the operator $(I-\Pi^k)\circ F_T$ to control the $s$-Sobolev norm of the error $u - \Pi^k(u)$. By repeating the scaling arguments on both $\norm{(u- \Pi^k(u)) \circ F_T}_{s,p,\hat{T}}$ and $\abs{u \circ F_T }_{k+1,p,\hat{T}}$, we get, for any triangle $T = F_T(\hat{T})$:
\begin{align}
\norm{u- \Pi^k(u)}_{s,p,T} &\lesssim \rho_T^{-s} J^{\frac{1}{p}} \norm{(u- \Pi^k(u)) \circ F_T}_{s,p,\hat{T}} \\
& = \rho_T^{-s} J^{\frac{1}{p}} \norm{u \circ F_T - \Pi^k(u) \circ F_T}_{s,p,\hat{T}} \\
& \lesssim \rho_T^{-s} J^{\frac{1}{p}} \abs{u \circ F_T }_{k+1,p,\hat{T}} \\
& \lesssim h_T^{k+1} \rho_T^{-s} \abs{u}_{k+1,p,T} \qquad 0\le s \le k.
\end{align}

Overall, we can control the $s$-Sobolev norm of the error up to order $k$ with the Sobolev semi-norm of order $k+1$ of the solution $u$. Since $P^\ell \subseteq P^k$ for $\ell \leq k$, we can repeat the argument for the Sobolev semi-norm of order $\ell+1$. Summing over the cells of the triangulation, we obtain a global \emph{a-priori} inequality:
\[
\sum_{T \in \T_h} \norm{u- \Pi^k(u)}_{s,p,T}^p \lesssim
\sum_{T \in \T_h} \bigl( h_T^{\ell+1} \rho_T^{-s} \abs{u}_{\ell+1,p,T} \bigr)^p, \qquad 0 \le s \le \ell \leq k.
\]
The thesis follows from the definitions of $h$ and $\rho$.
\end{proof}
The LHS is also called a \emph{broken Sobolev norm}, since it is defined element-wise. It is not automatically equal to $\norm{u- \Pi^k(u)}_{s,p,\Omega}$: in order to be, the global space $V_h$, obtained by gluing together the local spaces $P^k(T_m)$, must be \emph{conforming} to $V$. 

\begin{remark}[Conformity] For a finite element space to be conforming, we need to guarantee that the resulting finite dimensional space $V_h$ is indeed included in $V$. While locally every polynomial space $P^k(T)$ is always contained in every Sobolev space $W^{k,p}(T)$ (since $P^k(T) \subset C^{\infty}(T)$ for any $k$), the same cannot be said for the \emph{union} of the local spaces $P^k(T_m)$. Such a space is in general discontinuous (i.e., there is no reason to expect that the polynomials in $P^k(T_m)$ and $P^k(T_n)$ agree on the common face between the two cells $T_m$ and $T_n$), and although continuity is in general not required for Sobolev spaces $W^{k,p}(\Omega)$ (i.e., $W^{k,p}(\Omega)\not\subset C^0(\Omega)$ for $kp\leq d$), one can show that if $k \geq 1/p$, then a discontinuity along a surface of co-dimension one is not allowed.

If we want the finite dimensional space $V_h$ to be conforming with $W^{k,p}(\Omega)$ (for example), then we need to restrict the polynomial spaces on the elements $T_m$ and $T_n$ so that $P^k(T_m)$ and $P^k(T_n)$ agree on the common face between the two cells $T_m$ and $T_n$ for all derivatives up to the degree $k-1$.
\end{remark}

\begin{remark}
The same inequality can be proven if the triangulation $\T_h$ is made of quadrilaterals, and if the mapping is bi-linear instead of affine. In this case, the constants that appear in the error estimates are generally better, and this is why meshes made of quadrilaterals and hexahedra are usually preferred (when available): they are more difficult to construct, but they offer better approximation properties.
\end{remark}


\section{\texorpdfstring{$H^k$}{Hk} error estimates}

Finally, we can put together Ceà's lemma~\ref{lemma:cea} with Bramble-Hilbert lemma~\ref{lemma:bramble-hilbert} to obtain an \emph{a priori} error estimate for finite element methods of order $k$. To use the \emph{a-priori} interpolation estimate given in Theorem~\ref{theo:interpolation_error}, we need to make sure that the solution $u$ of the elliptic problem is indeed in the space $H^{k+1}(\Omega)$. This, in general, will depend on the operator $A$, on the source term $f$, and on the regularity of the domain $\Omega$, and it is what the theory of regularity for PDEs attempts to do. We will not go into details here, but we will just state the result we need.

\begin{definition}
  Let $V \subset H^s(\Omega)$, with $\|\cdot\|_V \sim \|\cdot\|_{s,\Omega}$ and $A \in \L(V,V')$. We say that $A$ is $r$-\emph{regular} if there exists an interval $[s_{min}, s_{max}]$ (with $s_{min}\geq -s$) such that $\forall g \in H^\ell(\Omega)$, with $\ell \in [s_{min}, s_{max}]$, the problem $Au=f$ admits a unique solution $u$ which satisfies
  \[
  \norm{u}_{\ell+r,\Omega} \lesssim \norm{f}_{\ell,\Omega}.
  \]
\end{definition}

With this definition, we can state the following theorem, which puts together all the previous results:
\begin{theorem}[A priori error estimate]
  \label{theo:a_priori_estimate}
Assume that $A$ is bounded and coercive in $H^s(\Omega)$, and that $A$ is $r$-regular for $f \in H^{\ell}(\Omega)$, $\ell \in [s_{min}, s_{max}]$. For a conforming approximation $V_h \subset V \equiv H^s(\Omega)$, constructed with piece-wise polynomial bases of order $k$, we have that the approximate solution $u_h$ satisfies the following \emph{a priori} error estimate:
\begin{equation}
  \label{eq:a_priori_error_estimates}
    \norm{u-u_h}_{s,\Omega} % \lesssim h^{\ell+r-s} \abs{u}_{\ell+r} 
    \lesssim h^{\ell+r-s} \norm{f}_{\ell,\Omega} \qquad  s_{min}+r \le \ell+r \le \min(k + 1,s_{max}+r).
\end{equation}
\end{theorem}
\begin{proof}
  From Ceà's lemma (Lemma~\ref{lemma:cea}) and the interpolation error estimate (Theorem~\ref{theo:interpolation_error}), we have that
  \begin{align} \label{ineq:cea_scaling}
    \norm{u - u_h}_{s,\Omega} &\le \frac{\norm{A}}{\alpha} \inf_{v_h \in V_h} \norm{u - v_h}_{s,\Omega} \\
  & \le \frac{\norm{A}}{\alpha} \norm{u - \Pi^k(u)}_{s,\Omega} \\
  & \lesssim \frac{\norm{A}}{\alpha} \rho^{-s} h^{k+1} \abs{u}_{k+1,\Omega}.
  \end{align}
  
  The thesis follows by the $r$-regularity assumption of the operator $A$.
\end{proof}
The inequality hints at some strategies for error reduction (independently on the solution or on $f$):
\begin{itemize}
\item take $h$ smaller, i.e. \emph{refine the mesh};
\item take $k$ bigger (if possible), which implies increasing the polynomial degree of the finite element space up to the maximum regularity allowed by exact solution $u$;
\end{itemize}

%%%%%
% Lecture of 27 of March
%%%%%


Consider a triangulation $\T_h$ and recall the quantities $h$ and $\rho$ we defined in the previous section:
\[
h := \max_{T \in \T_h} h_T, \quad \rho := \min_{T \in \T_h} \rho_T.
\]
We say that the mesh is \emph{shape-regular} if
\[
\exists \sigma > 0 \text{ s.t. } \rho_T \ge \sigma h_T  \quad \forall T\in \T_h
\]
and we call $\sigma$ the constant of shape-regularity. We say instead that a mesh is \emph{quasi-uniform} if 
\[
\rho \sim h
\]
For such a mesh, inequality~\eqref{ineq:cea_scaling} becomes:
\begin{equation} \label{ineq:cea_scaling2}
\norm{u - u_h}_{s,p,\Omega}
\lesssim \frac{\norm{A}}{\alpha} h^{k+1-s} \abs{u}_{k+1,p,\Omega}.
\end{equation}

\begin{example}
The Poisson problem on a Lipschitz, convex domain is $2$-regular (i.e. $r = 2$), with $s_{min} = -1$ and $s_{max}=0$. If $u \in H^1_0(\Omega)$ (i.e. $s = 1$)  inequality~\eqref{eq:a_priori_error_estimates} becomes
\[
\norm{u-u_h}_1 \lesssim h^{\ell+1} \norm{f}_\ell \quad \forall -1 \le \ell \le 0.
\]

If $\Omega$ is better than Lipschitz, then the range for $s$ becomes better, hence we gain a better convergence of the numerical approximation (provided that $f$ is more regular).
\end{example}

In order to show convergence of the numerical approximation in a weaker norm, we need to establish a relationship between the norm of the error involved and utilize the properties of the finite element spaces.

\subsection{Nitsche's trick}
With some bit of additional work, we can produce error estimates on weaker norms, such as the $L^2$ one, from the estimates on stronger norms.
\begin{lemma}[Nitsche] \label{lemma:nitsche}
Let $A \in \L(V,V')$ an operator as in the hypotheses of Lax-Milgram's lemma. Assume that $V$ is a subspace of some Hilbert space $H$, with the following properties:
\begin{romanlist}
\item $\norm{u}_H \lesssim \norm{u}_V$ $\quad \forall u \in V$;
\item $V = \overline{H}^{\norm{\cdot}_V}$,
i.e. the embedding $V \hookrightarrow H$ is dense and continuous;
\item we identify $H$ with its dual $H'$ via the Riesz representation theorem.
\end{romanlist}
Then the following inequality holds:
\begin{equation}\label{eqn:nitsche_lemma}
\norm{u-u_h}_H \lesssim \sup_{g \in H} \Bigl( \frac{1}{\norm{g}_H} \inf_{\phi_h \in V_h} \norm{\phi_g - \phi_h}_V \Bigr) \norm{u-u_h}_V .
\end{equation}
\end{lemma}

\begin{proof}

With the hypotheses above, if $f \in H \equiv H'$, by the Hahn-Banach theorem $f$ can be extended to a functional $\tilde{f} \in V'$, i.e.
\[
\langle \tilde{f}, v \rangle_V = (f, v)_H \quad \forall v \in V.
\]
Now consider the usual problem in operator form $A u = \tilde{f}$, i.e.
\begin{equation}
\langle Au,v \rangle = \langle \tilde{f},v \rangle = (f, v)_H \quad \forall v\in V.
\end{equation}
By Lax-Milgram lemma, we know that this problem has a unique solution $u \in V$.

Let $g \in H$ and consider the \emph{dual problem} $A^T \phi_g = g$, i.e.
\begin{equation}
\langle Av,\phi_g \rangle = \langle A^T\phi_g, v \rangle = \langle \tilde{g},v \rangle = (g, v)_H \quad \forall v\in V.
\end{equation}
Since also $A^T$ satisfies the hypotheses of  Lax-Milgram lemma, the adjoint problem admits itself a unique solution $\phi_g \in V$.

We start observing that
\begin{equation}\label{eqn:nitsche_norm}
\norm{u-u_h}_H = \sup_{g \in H} \frac{(g, u-u_h)_H}{\norm{g}_H}.
\end{equation}
This is true because $H$ is a Hilbert space, hence $H \cong H'$.

Given $g \in H$, we know from the last remark that the dual problem $A^T \phi = g$ admits a unique solution $\phi_g \in V$. If we set $v = u-u_h$, we get:
\[
\langle A (u-u_h),\phi_g \rangle = (g, u-u_h)_H.
\]
Now we take advantage of the Galerkin orthogonality, i.e. $\langle A (u-u_h), \phi_h \rangle = 0$ $\forall \phi_h \in V_h$:
\[
(g, u-u_h)_H = \langle A (u-u_h),\phi_g - \phi_h \rangle \quad \forall \phi_h \in V_h.
\]
By the continuity of $A$ on~\eqref{eqn:nitsche_norm}, we can produce the desired estimate:
\begin{align}
\norm{u-u_h}_H &= \sup_{g \in H} \frac{(g, u-u_h)_H}{\norm{g}_H} \\
& = \sup_{g \in H} \frac{\langle A (u-u_h),\phi_g - \phi_h \rangle}{\norm{g}_H} \\
& \le \norm{A} \sup_{g \in H} \Bigl( \frac{1}{\norm{g}_H} \inf_{\phi_h \in V_h} \norm{\phi_g - \phi_h}_V \Bigr)
\norm{u-u_h}_V .
\end{align}
\end{proof}

In practice, we want to use this lemma with $V \subset H^s$ and $H \subset H^\ell$, with $0\le \ell < s$, and exploit the $r$-regularity property of the operator to gain better estimates in the $H$ norm, by showing that the term $\norm{\phi_g - \phi_h}_V$ is itself proportional to a power of $h$ and to the $H$ norm of $g$. 

\begin{theorem}[Nitsche's trick]
  \label{theo:nitsche_trick}
  Let $A \in \L(V,V')$ be a $r$-regular, bounded, and coercive operator. Let $V\equiv H^s(\Omega)$, $H \equiv H^\ell(\Omega)$ satisfy the same hypotheses of Lemma~\ref{lemma:nitsche}. Moreover, the following holds:
  \begin{romanlist}
  \item $V_h \subset V$ is a finite element space of polynomials of degree $k$;
  \item $0\leq \ell < s$;
  \item $s+r\leq k+1$;
  \end{romanlist}

  Then, for any $f\in H^{m}(\Omega)$, with $m\in[s_{min}, s_{max}]$, the following estimates hold:
  \begin{equation}
    \label{eqn:nitsche_trick}
    \begin{aligned}
      \norm{u-u_h}_s &\lesssim h^{m+r-s} \norm{f}_m,\\
      \norm{u-u_h}_\ell &\lesssim h^{\ell+r-s}  \norm{u-u_h}_s \lesssim h^{(\ell+m+2(r-s))} \norm{f}_\ell.
    \end{aligned}
  \end{equation}
\end{theorem}

\begin{proof}
  The first part of the estimate comes from
  Theorem~\ref{theo:a_priori_estimate}. The second part of the estimate comes
  from Lemma~\ref{lemma:nitsche} and from the $r$-regularity of the operator
  $A^T$. We refine the estimate of $\norm{\phi_g - \phi_h}_s$ by
  Theorem~\ref{theo:interpolation_error} applied to the function $\phi_g$. 
  Provided that $0\leq s \leq \ell+r$, and that $\ell+r \leq k+1$, we have:
\begin{align}
\inf_{\phi_h \in V_h} \norm{\phi_g - \phi_h}_s &\lesssim \norm{\phi_g - \Pi^k\phi_g}_{s} \lesssim h^{\ell+r-s} \abs{\phi_g}_{\ell+r}  \\
&\lesssim h^{\ell+r-s} \norm{g}_\ell.
\end{align}
Inserting this estimate in Lemma~\ref{lemma:nitsche}, we obtain the thesis.
\end{proof}

\begin{example}
For a Poisson problem on a Lipschitz, convex domain, we have that $r=2$ in the range $[-1,0]$. With this we can estimate also the $L^2$ norm of the error, i.e., we pick $s = 0$ (and we have $k=1$) and we obtain:
\[
\norm{u-u_h}_0 \lesssim h \norm{u-u_h}_1.
\]
By the $2$-regularity, we have
\[
\norm{u-u_h}_1 \lesssim h^{t+1} \norm{f}_t \qquad \forall t \in [-1, 0].
\]
Hence:
\[
\norm{u-u_h}_0 \lesssim h^{t+2} \norm{f}_t.
\]
For instance, if $t=0$, we have convergence of order $1$ in $H^1$, of order $2$ in $L^2$.
\end{example}


\section{Inverse estimates}
Suppose that $V_h|_T \equiv P^k(T)$ and let $v_h \in V_h$. In the previous sections we have proved that, for quasi-uniform meshes, on each triangle $T$ the following inequality holds:
\[
\abs{v_h}_{s,p,T} \lesssim h^{k-s} \abs{v_h}_{k,p,T} \quad \forall 0 \le s \le k.
\]
With the same scaling argument, we can also prove that, if $h<1$, then
\[
\abs{v_h}_{s,p,T} \lesssim h^{\ell-s} \abs{v_h}_{\ell,p,T} \quad \forall 0 \le \ell \le s \le k.
\]
Of course, in either case, $k$, $s$ must be chosen smaller or equal to the degree of $v_h$. The proof relies once more on the fact that all norms in finite dimensional spaces are equivalent, and that the constants in the equivalence depend on the shape of the element and on the size of the element. By moving to a reference element, and exploiting the scaling properties of the norms, we quantify the constants in the equivalence in terms of mesh size $h$ only when moving to (or from) the reference element. There, we can freely exchange the norms, exploiting equivalence of all norms in finite dimensional spaces, on a domain of size $O(1)$, with constants that do not depend on $h$. By going back to the original element with the new norm, we again use the scaling properties to obtain the desired result. 

When moving an $s$ norm to the reference element we pay a factor $h^{-s}$, while when moving an $\ell$ norm from the reference element to the current element we pay a factor $h^{\ell}$. Overall we control $s$ norms with $\ell$ norms with a factor $h^{\ell-s}$, and vice versa.

\section{Trace operators}\label{sec:trace_operators}
In order to deal with the trace of a Sobolev function, it is necessary to give a meaning to the space $W^{s,p}(\Omega)$ for $s\in \R$. Here we will only present the fundamental concepts. For extensive reference, see~\cite{fract_sob}. For a Lipschitz domain $\Omega \subset \R^d$ we can also define the space $W^{s,p}(\Omega)$ in a different way. We let $\lambda \in (0,1)$ such that $s = m+\lambda$ for an integer $m$ and consider the following norm:
\[
\norm{u}_{s,p}^p := \norm{u}_{m,p}^p + \sum_{\abs{\alpha}=m} \abs{D^\alpha u}^p_{\lambda,p}
\]
where $\abs{u}_{\lambda,p}$ is called in literature the \emph{Gagliardo seminorm} of $u$:
\[
\abs{u}^p_{\lambda,p} := \int_{\Omega} \int_{\Omega} \frac{\abs{u(x)-u(y)}^p}{\abs{x-y}^{d+\lambda p}} \diff x \,dy.
\]
Then we define
\[
W^{s,p}(\Omega) := \Set{u \in W^{m,p}(\Omega) : \norm{u}_{s,p}^p < +\infty}.
\]
As in the case $s \in \N$, we also have that $W^{s,p}_0(\Omega) = \overline{C^\infty_0(\Omega)}^{\norm{\cdot}_{s,p}}$. One can prove that the spaces we have constructed are Banach spaces. Moreover, if we restrict to the case $p=2$, they turn out to be Hilbert spaces.

For the case $p=2$, if $\Omega = \R^d$, there is an equivalent definition of the space $H^s(\R^d)$ via the Fourier transform $\F$:
\[
\bigl(\F u\bigr)(\xi) := \Bigl(\frac{1}{2\pi} \Bigr)^{d/2} \int_{\R^d} e^{-i \xi \cdot x} u(x) \diff x.
\]
If we consider the Schwartz space $\mathcal{S}$ of rapidly decaying $C^\infty$ functions in $\R^d$, then $H^s(\R^d)$ can be defined as a subspace of tempered distributions:
\[
H^s(\R^d) := \Set{u \in \mathcal{S}' : \norm{u}_s < +\infty}
\]
where the norm is
\[
\norm{u}_s := \norm{\abs{(1+\abs{\cdot}^2)}^{\frac{s}{2}} \F u(\cdot)}_{L^2(\R^d)}.
\]
In particular, as before we have $H^s_0(\R^d) = \overline{C^\infty_0(\R^d)}^{\norm{\cdot}_s}$.
Without going into detail, the main reason of the equivalence ot the two formulation is Plancherel's formula for the Fourier transform, which is specific for $p=2$. There is no equivalence for different values of $p$.

Now let $\Omega \subset \R^d$ be a Lipschitz domain and $\Gamma$ its boundary. If $u \in H^s(\Omega) \not\subset C^0(\Omega)$, then $\restr{u}{\Gamma}$ makes no sense pointwise. For example, $H^1$ functions are not continuous for $d \ge 2$. In order to give some sense to the expression $\restr{u}{\Gamma}$, we introduce the \emph{trace operator}:
\begin{theorem}[Trace theorem]
Let $\Omega \subset \R^d$ be a Lipschitz domain and $s \in (\frac{1}{2}, 1]$. Then there exists a unique linear bounded mapping
\[
\gamma: H^s(\Omega) \to H^{s-\frac{1}{2}}(\Gamma)
\]
such that:
\begin{enumerate}
\item For every $u \in C^0(\overline{\Omega})$, $\gamma u = \restr{u}{\Gamma}$.
\item For every $v \in H^s(\Omega)$, $\norm{\gamma v}_{s-\frac{1}{2},\Gamma} \lesssim \norm{v}_{s,\Omega}$.
\item $\gamma$ admits a bounded right inverse
\[
E: H^{s-\frac{1}{2}}(\Gamma) \to H^s(\Omega)
\]
i.e. $\forall g\in H^{s-\frac{1}{2}}(\Gamma)$ we have $\norm{E g}_{s,\Omega} \lesssim \norm{g}_{s-\frac{1}{2},\Gamma}$ and $\gamma E g = g$.
\end{enumerate}
In particular, $\ker(\gamma) = H^s_0(\Omega) = \Set{u \in H^s(\Omega) : \gamma u = 0}$.
\end{theorem}

The trace is a fundamental concept when dealing with boundary conditions. Consider as a motivating example the Poisson problem on $\Omega$ with both Dirichlet and Neumann boundary conditions:
\[
\begin{cases} 
-\Delta u = f \qquad &\text{in $\Omega$} \\
u =  g_D \qquad &\text{on $\Gamma_D$} \\
\frac{\partial u}{\partial n} =  g_N \qquad &\text{on $\Gamma_N$}
\end{cases}
\]
where $\Gamma_D$ and $\Gamma_N$ are contained in $\Gamma$. If $u \in H^1$, then its restriction to $\Gamma$ is not well defined. In order to handle Dirichlet boundary conditions, we can resort to the trace operator (with restricted codomain) $\gamma_{_{\Gamma_D}}: H^1(\Omega) \to H^{\frac{1}{2}}(\Gamma_D)$ and require that $\gamma_{_{\Gamma_D}} u = g_D$. However, this trick fails with $\frac{\partial u}{\partial n} \in L^2$, because even its trace makes no sense! We solve this problem incorporating the Neumann boundary condition into the weak form.
To make things clearer: we let
\begin{align}
V_{0,\Gamma_D} &:= \Set{u \in H^1(\Omega) : \gamma_{_{\Gamma_D}}=0}, \\
V_{g_D,\Gamma_D} &:= V_{0,\Gamma_D} + u_D, \quad \text{where } \gamma_{_{\Gamma_D}} u_D = g_D.
\end{align}
The new weak form is: find $u \in V_{g_D,\Gamma_D}$ such that
\[
(\nabla u, \nabla v) = \langle f, v \rangle + \int_{\Gamma_N} g_N v \quad \forall v \in V_{0,\Gamma_D}.
\]
In particular, the minimal regularity we need to require for $f$, $g_D$, $g_N$ is:
\[
f \in \bigl(H^1(\Omega)\bigr)', \quad g_D \in H^\frac{1}{2}(\Gamma_D), \quad g_N \in \bigl(H^\frac{1}{2}(\Gamma_N)\bigr)'.
\]