% !TeX source = ../main.tex

\chapter[Introduction]{Introduction}

\section{Notations on Sobolev spaces}
\label{sec:sobolev_notations}

\begin{figure}[!ht]
  \centering
  \includegraphics{domain}
  \caption{A Lipschitz domain $\Omega$ and its boundary $\partial\Omega$. We generally indicate with $n$ the outer normal to the boundary.}
  \label{fig:domain}
\end{figure}

Let $\Omega \subset \R^d$ be an open, bounded domain with Lipschitz boundary $\partial\Omega$. We begin by introducing some standard functional spaces that will be used throughout this course.

\subsection{Lebesgue spaces}

For $1 \leq p < \infty$, we define the Lebesgue space $L^p(\Omega)$ as the set of measurable functions $u: \Omega \to \R$ such that
\[
\|u\|_{L^p(\Omega)} := \left(\int_\Omega |u(x)|^p \diff x \right)^{1/p} < \infty.
\]

For $p = \infty$, we define $L^\infty(\Omega)$ as the space of essentially bounded measurable functions with the norm
\[
\|u\|_{L^\infty(\Omega)} := \operatorname{ess\,sup}_{x \in \Omega} |u(x)| < \infty.
\]

Finally we define the space of locally integrable functions as $L^1_{loc}(\Omega)$ where every function $u: \Omega \to \R$ is:
\begin{itemize}
\item measurable;
\item $u|_{K} \in L^1(K)$ for each compact subset $K \subseteq \Omega$.
\end{itemize}

These spaces, equipped with their respective norms, are Banach spaces. In particular, $L^2(\Omega)$ is a Hilbert space with the inner product
\[
(u, v)_{L^2(\Omega)} = \int_\Omega u(x) v(x) \diff x.
\]



\subsection{Weak derivatives}

Let $\alpha = (\alpha_1, \alpha_2, \ldots, \alpha_d)$ be a multi-index with $|\alpha| = \alpha_1 + \alpha_2 + \cdots + \alpha_d$. For a function $u \in C^{|\alpha|}(\Omega)$, we define the standard partial derivative
\[
D^\alpha u = \frac{\partial^{|\alpha|} u}{\partial x_1^{\alpha_1} \partial x_2^{\alpha_2} \cdots \partial x_d^{\alpha_d}}.
\]

For functions that are not sufficiently smooth, we introduce the concept of weak derivatives.

\begin{definition}[Weak derivative]
Let $u \in L^1_{\text{loc}}(\Omega)$ and $\alpha$ be a multi-index. A function $v \in L^1_{\text{loc}}(\Omega)$ is called the $\alpha$-th weak derivative of $u$, denoted by $D^\alpha u = v$, if
\[
  \int_\Omega v(x) \phi(x) \diff x =  (-1)^{|\alpha|} \int_\Omega u(x) D^\alpha \phi(x) \diff x  \quad \forall \phi \in C_0^\infty(\Omega).
\]
\end{definition}

\subsection{Sobolev spaces $W^{k,p}(\Omega)$ and $H^k(\Omega)$}

For integers $k \geq 0$ and $1 \leq p \leq \infty$, we define the Sobolev space $W^{k,p}(\Omega)$ as
\[
W^{k,p}(\Omega) = \{u \in L^p(\Omega) : D^\alpha u \in L^p(\Omega) \text{ for all } |\alpha| \leq k\},
\]
where $D^\alpha u$ are weak derivatives. For $p < \infty$, this space is equipped with the norm
\[
\|u\|_{k,p,\Omega} := \left( \sum_{|\alpha| \leq k} \|D^\alpha u\|_{L^p(\Omega)}^p \right)^{1/p}.
\]

For $p = \infty$, we define
\[
\|u\|_{k,\infty,\Omega} = \max_{|\alpha| \leq k} \|D^\alpha u\|_{L^\infty(\Omega)}.
\]

We also define the seminorm
\[
|u|_{k,p,\Omega} = \left( \sum_{|\alpha| = k} \|D^\alpha u\|_{L^p(\Omega)}^p \right)^{1/p},
\]
with a similar modification for $p = \infty$.

When $p = 2$, we denote $W^{k,2}(\Omega)$ by $H^k(\Omega)$, which is a Hilbert space with the inner product
\[
(u, v)_{H^k(\Omega)} = \sum_{|\alpha| \leq k} (D^\alpha u, D^\alpha v)_{L^2(\Omega)}.
\]

In this case, we simplify the notation for the definition of the norm, and we omit the subscript $2$ (and the domain $\Omega$ if no confusion can arise) for the $H^k$-norm:
\[
  \|u\|_{k} \equiv \|u\|_{k,\Omega} := \|u\|_{k,2,\Omega} = \left( \sum_{|\alpha| \leq k} \|D^\alpha u\|_{L^2}^2 \right)^{1/2}.
\]

\subsection{Sobolev spaces with vanishing boundary values}

For functions in Sobolev spaces, we define the subspace of functions vanishing on the boundary $\partial\Omega$. For $1 \leq p < \infty$, we define
\[
W_0^{k,p}(\Omega) = \overline{C_0^\infty(\Omega)}^{\|\cdot\|_{W^{k,p}(\Omega)}},
\]
that is, the closure of $C_0^\infty(\Omega)$ with respect to the $W^{k,p}$-norm.

In particular, we denote $W_0^{k,2}(\Omega)$ by $H_0^k(\Omega)$. A key property of $H_0^1(\Omega)$ is the Poincaré inequality, which states that there exists a constant $C_{\Omega} > 0$ such that
\[
\|u\|_{L^2(\Omega)} \leq C_{\Omega} \|\nabla u\|_{L^2(\Omega)} \quad \forall u \in H_0^1(\Omega).
\]

This implies that the seminorm $|u|_{H^1(\Omega)} = \|\nabla u\|_{L^2(\Omega)}$ is actually a norm on $H_0^1(\Omega)$ equivalent to the standard $H^1$-norm.

\subsection{Dual spaces}

For a Banach space $X$, we denote by $X'$ its dual space, i.e., the space of continuous linear functionals on $X$. For example, $H^{-1}(\Omega) = (H_0^1(\Omega))'$ denotes the dual of $H_0^1(\Omega)$. We indicate with $\langle \cdot, \cdot \rangle$ the duality pairing between $X$ and $X'$, and we write $f(u) \equiv \langle f, u \rangle$ for $f \in X'$ and $u \in X$.

The norm of a functional $f \in X'$ is defined as
\[
\|f\|_{X'} = \sup_{u \in X} \frac{|\langle f, u \rangle|}{\|u\|_X}.
\]

\subsection{Riesz representation theorem}

For Hilbert spaces, we have a powerful characterization of the dual space through the Riesz representation theorem.

\begin{theorem}[Riesz representation theorem]
  \label{theo:riesz_representation}
Let $H$ be a Hilbert space with inner product $(\cdot, \cdot)_H$. For every bounded linear functional $f \in H'$, there exists a unique element $\tau f \in H$ such that
\[
f(v) = \langle f, v \rangle = (\tau f, v)_H \quad \forall v \in H,
\]
and furthermore,
\[
\|f\|_{H'} = \|\tau f\|_H.
\]
The operator $\tau$ is the so called Riesz operator.
\end{theorem}

This theorem establishes an isometric isomorphism between a Hilbert space and its dual. In particular, for $H = L^2(\Omega)$, given $f \in L^2(\Omega)$, we can identify the functional 
\[
v \mapsto \int_\Omega f v \, \mathrm{d}x
\]
with $f$ itself. Similarly, for $H = H^1_0(\Omega)$, every element in $H'$ can be represented uniquely as an inner product with some element in $H$.

The Riesz representation theorem is especially useful in characterizing the action of operators and in proving existence of solutions to variational problems, as it allows us to convert abstract duality pairings into concrete inner products.

\subsection{Bilinear forms and operators}

Let $X$ and $Y$ be Banach spaces. A bilinear form $a: X \times Y \to \mathbb{R}$ is said to be bounded (or continuous) if there exists a constant $C > 0$ such that
\[
|a(u, v)| \leq C \|u\|_X \|v\|_Y \quad \forall u \in X, \forall v \in Y.
\]

Given a bilinear form $a: X \times Y \to \mathbb{R}$, we can define its associated operator $A: X \to Y'$ by
\[
\langle Au, v \rangle = a(u, v) \quad \forall u \in X, \forall v \in Y,
\]
where $\langle \cdot, \cdot \rangle$ denotes the duality pairing between $Y'$ and $Y$.

The operator $A$ is linear and it inherits the boundedness of $a$, i.e.,
\[ 
  \|A\|_* := \|A\|_{\mathcal{L}(X,Y')} = \sup\limits_{u \in X, v \in Y} \frac{|a(u,v)|}{\|u\|_X \|v\|_Y}.
\]

When $X = Y$ is a Hilbert space, a bilinear form $a: X \times X \to \mathbb{R}$ is said to be:
\begin{itemize}
  \item Symmetric if $a(u, v) = a(v, u)$ for all $u, v \in X$
  \item Coercive (or elliptic) if there exists $\alpha > 0$ such that $a(u, u) \geq \alpha \|u\|_X^2$ for all $u \in X$
\end{itemize}

\subsection{Trace spaces and the trace operator}

For functions in $H^1(\Omega),$ we can define their traces on the boundary $\partial\Omega$. The trace operator 
\[
\gamma: H^1(\Omega) \to H^{1/2}(\partial\Omega)
\]
is a bounded linear operator that extends the classical restriction to the boundary. For simplicity, we often write $u|_{\partial\Omega}$ for $\gamma(u)$.

The space $H^{1/2}(\partial\Omega)$ is a fractional Sobolev space on the boundary, and it can be characterized as the set of traces of $H^1(\Omega)$ functions.

A fundamental result is that $u \in H_0^1(\Omega)$ if and only if $u \in H^1(\Omega)$ and $\gamma(u) = 0$.

\subsection{Sobolev embeddings}

Sobolev spaces embed into other spaces in ways that formalize the intuitive notion that functions with more derivatives tend to be more regular. The Sobolev embedding theorems are crucial tools that allow us to relate Sobolev spaces to classical function spaces.

\begin{theorem}[Sobolev embedding]
Let $\Omega \subset \mathbb{R}^d$ be a bounded domain with Lipschitz boundary, $k \geq 0$ an integer, and $1 \leq p < \infty$. Then:

\begin{enumerate}
  \item If $kp > d$, then the following continuous embedding holds:
  \[
  W^{k,p}(\Omega) \hookrightarrow C^{0}(\overline{\Omega}),
  \]
  i.e., every function in $W^{k,p}(\Omega)$ has a continuous representative.
  
  \item If $s \geq 1$, $p,q \in [1, \infty]$, and 
  \[
  s - \frac{d}{p} > k - \frac{d}{q},
  \]
  then the following continous inclusion holds true:
  \[ 
    W^{s,p}(\Omega) \subset W^{t,q}(\Omega).
  \]
\end{enumerate}
\end{theorem}

The first case is of particular interest when we need our functions to be continuous. The condition $kp > d$ gives us a relationship between:

\begin{itemize}
  \item The order of derivatives $k$
  \item The integrability $p$ of those derivatives
  \item The dimension $d$ of the domain
\end{itemize}

For example, in two dimensions ($d=2$):
\begin{itemize}
  \item $H^2(\Omega) = W^{2,2}(\Omega) \hookrightarrow C^0(\overline{\Omega})$ since $2 \cdot 2 > 2$
  \item $W^{1,p}(\Omega) \hookrightarrow C^0(\overline{\Omega})$ for any $p > 2$
  \item $H^{1}(\Omega) \not\hookrightarrow C^0(\overline{\Omega})$.
\end{itemize}

More generally, in $d$ dimensions, we need $k > \frac{d}{p}$ to ensure continuity.

These embedding results are particularly important in numerical analysis, as they help determine when finite element approximations lead to continuous functions. For standard Lagrangian finite elements, the continuity of basis functions is essential, and these embedding theorems provide the theoretical foundation for when such continuity can be expected.

%****************************************************************
% Lezione 27 febbraio
%****************************************************************

\section{Model problem: the Poisson equation}

\subsection{Strong formulation}
Let $\Omega$ be an open, bounded, Lipschitz subset of $\R^d$. Let also $\partial\Omega$ be its boundary.
Consider the following Poisson problem:
\[
\begin{cases} \marginpar{Strong formulation of Poisson problem}
-\Delta u = f \qquad &\text{in $\Omega$} \\
u =  0 \qquad &\text{on $\partial \Omega$}
\end{cases}
\]
where, as usual,
\[
\Delta = \sum_{i=1}^d \frac{\partial^2}{\partial x_i^2}.
\]
If we want to find a numerical solution for this problem, two approaches can be followed:
\begin{itemize}
\item discretize $-\Delta$ (\emph{finite differences});
\item consider the \emph{weak formulation} of the problem (\emph{finite elements}).
\end{itemize}

Finite differences work well if $\Omega$ is a rectangular or parallelepiped domain (e.g. a square, a cube and so on) and if $f$ is regular enough (e.g. continuous). In applications, however, the domain's boundary may not be nice at all, and $f$ may not even be continuous.
The finite element approach tries to overcome these difficulties, by shifting the problem to finding a good finite dimensional space that approximates the functional space where the exact solution lives.

The advantage, when compared to, e.g., finite differences, is that we don't discretize the differential operator -- which maintains its continuous definition --  but instead we restrict our exploration of candidate solutions to simpler spaces (i.e., spaces of piecewise polynomial functions), for which we can easily compute the action of the differential operator in an exact way.

\subsection{Weak formulation}
To make a concrete example, let us consider the Poisson problem, and let's derive its weak formulation.
Let $\phi \in \D$ a test function (usually $\D=C_0^\infty(\Omega)$). We multiply by $\phi$ both sides of the PDE in the strong formulation and then integrate:
\[
\int_\Omega -\Delta u \phi = \int_\Omega f \phi.
\]
If we integrate by parts and use the fact that $\restr{u}{\partial \Omega} = \restr{\phi}{\partial\Omega} = 0$, we get:
\[
\int_\Omega \nabla u \nabla \phi = \int_\Omega f \phi.
\]
In general we replace $\D$ with a Sobolev space $V$ such that every entry of the weak formulation makes sense. In this case, the weak formulation will be: given $f\in V'$, find $u\in V$ such that
\begin{equation} \label{eqn:weak_1} \marginpar{Weak form (1)}
\int_\Omega \nabla u \nabla v = \int_\Omega f v \quad \forall v\in V.
\end{equation}
with the agreement that, being $f \in V'$, the RHS is in fact a duality.

On one side, we need that $\int_\Omega | \nabla u \nabla v | < +\infty$, and on the other side, we need that $u$ and $v$ should be zero on the domain boundary. For the example above, the natural choice is then the Hilbert space $H^1_0(\Omega)$.

\subsection{Lax-Milgram lemma}
How do we guarantee that a solution to such problem exists? We prove this in an abstract Hilbert setting:
\begin{lemma}[Lax-Milgram]\label{lemma:lax-milgram}\marginpar{Lax-Milgram lemma}
Let $V$ be a Hilbert space and let $a: V\times V \to \R$ be a bilinear operator such that:
\begin{itemize}
\item $a$ is \emph{bounded}, i.e. $\exists c>0$ s.t. 
\[
 a(u,v) \le c \norm{u}_V \norm{v}_V \qquad \forall u,v \in V;
 \]
\item $a$ is \emph{coercive}, i.e. $\exists \alpha >0$ s.t.
\[
  a(u,u) \ge \alpha \norm{u}_V^2 \qquad \forall u \in V.
\]
\end{itemize}
Then, given $f\in V'$, the following problem
\begin{equation}\label{eqn:weak_laxmilgram}
a(u,v) = \langle f,v \rangle \quad \forall v\in V
\end{equation}
admits a unique solution $u$. Moreover, $u$ satisfies the following inequality:
\[
\norm{u}_V \le \frac{\norm{f}_{V'}}{\alpha}.
\]
\end{lemma}
\begin{proof}
  Let's consider the linear operator $A: V \to V'$ defined by $A(u) = a(u,\cdot)$.
  Boundedness of $a$ implies that $A$ is a bounded operator, and coercivity implies that 
  \[
    \langle A u, u \rangle = a(u,u) \ge \alpha \norm{u}_V^2 \quad \forall u \in V.
  \]
  
  Let us consider the map $\phi: V \mapsto V$ defined by 
  \[ 
    \phi(v) = v - \rho \tau(A v - f),
  \]
  where $\rho$ is a positive constant to be chosen later, and $\tau$ is the Riesz operator defined in Theorem~\ref{theo:riesz_representation}. By construction, if $u$ is a fixed point of $\phi$, i.e., $u= \phi(u)$, then $u$ necessarily satisfies $A u = f$. Conversely, if $u$ satisfies $Au=f$, then $u$ is also a fixed point of $\phi(u)=u$. 
  
  We want to show that ellipticity is enough to guarantee that there exists a $\rho$ such that $\phi$ is a contraction, i.e., $\exists! ~u$ s.t. $Au=f$.
  
  We have:
  \begin{align*}
    \norm{\phi(u) - \phi(v)}_V^2 & = \norm{u - v - \rho \tau(A(u - v))}_V^2 \\
    & = \norm{u - v}_V^2 - 2 \rho \langle A(u - v), u - v \rangle + \rho^2 \norm{A(u - v)}_{V'}^2 \\
    & \leq \norm{u - v}_V^2 - 2 \rho \alpha \norm{u - v}^2_V + \rho^2 \|A\|^2 \norm{u - v}_V^2 \\
    & \leq  (1 - 2\rho \alpha + \rho^2 \|A\|^2) \norm{u - v}_V^2.
  \end{align*}
  By choosing $\rho$ such that $0<1 - 2\rho \alpha + \rho^2 \|A\|^2 < 1$, (i.e., $\rho<\frac{2\alpha}{\|A\|}$) we have that $\phi$ is a contraction, and by the Banach fixed point theorem, we have that there exists a unique fixed point $u$ of $\phi$, which is the solution of the problem.
  
  Finally, we can estimate the norm of $u$ by the norm of $f$ and the coercivity constant $\alpha$:
  \[
  \alpha \norm{u}_V^2 \leq \langle A u, u \rangle = \langle f, u \rangle \leq \norm{f}_{V'} \norm{u}_V \qquad \Longrightarrow \qquad \norm{u}_V \leq \frac{\norm{f}_{V'}}{\alpha}.
  \]
\end{proof}

In the Poisson problem's case, the bilinear operator is
\[
a(u,v) = \int_\Omega \nabla u \nabla v,
\]
which is clearly bounded and coercive if $V=H_0^1(\Omega)$. In particular, coercivity follows from Poincaré inequality which implies that the norm in $H^1(\Omega)$ is equivalent to the $H^1$ seminorm
\[
\abs{u}_{1,2} = \norm{\nabla u}_{L^2}.
\]
Hence, the weak formulation of the Poisson problem admits a unique solution by Lax-Milgram lemma.

\subsection{Functional minimization}

\begin{theorem}[Minimization problem]
When the bilinear operator $a$ is also symmetric (i.e., $a(u,v) = a(v,u)$ for any pair $u,v \in V$), then existence and uniqueness of a solution to the weak formulation~\eqref{eqn:weak_1} is equivalent to finding the unique minimizer of the quadratic (and strongly convex) minimization problem:
\[
u = \arg\min_{v \in V} \Phi(v) := \frac{1}{2} a(v,v) - \langle f, v \rangle.
\]
\end{theorem}
\begin{proof}
 We start by showing that if $u$ is a minimizer of $\Phi$, then it satisfies the weak formulation. We have in fact that for any $v$ in $V$, and for any $\varepsilon \in \R^+$:
 \begin{equation*}
\begin{split}
  \Phi(u) \leq \Phi(u+\varepsilon v)  & = \frac{1}{2} a(u+\varepsilon v, u+\varepsilon v) - \langle f, u+\varepsilon v \rangle \\
  & = \Phi(u) + \varepsilon a(u,v) + \frac{\varepsilon^2}{2} a(v,v) - \varepsilon \langle f, v \rangle, \\
  - \frac{\varepsilon^2}{2} \alpha \|v\|^2 \leq - \frac{\varepsilon^2}{2} a(v,v) & \leq \varepsilon a(u,v) - \varepsilon \langle f, v \rangle \\
  - \frac{\varepsilon}{2} \alpha \|v\|^2 & \leq a(u,v) - \langle f, v \rangle \leq \frac{\varepsilon}{2} \alpha \|v\|^2,
  \end{split}
 \end{equation*}
 where the second line follows by the symmetry of the bilinear form $a$, while the last chain of inequalities follow from dividing once for $\varepsilon$, and once for $-\varepsilon$ both sides of the inequality. We can now take the limit $\varepsilon \to 0$ and we get that
  \[
  a(u,v) = \langle f, v \rangle \quad \forall v \in V.
  \]
  Conversely, if $u$ satisfies the weak formulation, we can show that it is a minimizer of $\Phi$.
  We have:
  \[
  \begin{split}
  \Phi(v) - \Phi(u) & = \frac{1}{2} a(v,v)  \underbrace{- \langle f, v \rangle}_{= - a(u,v)=-\frac12a(u,v)-\frac12a(v,u)} - \frac{1}{2} a(u,u)  + \underbrace{\langle f, u \rangle}_{=a(u,u)} \\
   & = \frac{1}{2} a(v-u,v-u) \geq \frac{\alpha}{2} \|v-u\|_V^2, 
  \end{split}
  \]
  where we used the coercivity of $a$ and the symmetry of $a$ to rearrange the terms. This shows that $\Phi(u)$ is a global minimum of $\Phi$, and since $\Phi$ is strongly convex, it is also the unique minimizer.
\end{proof}

\subsection{Ceà's lemma}
Lax-Milgram lemma guarantees the existence of a unique solution to the weak
problem, but it does not provide any information about the error we make when we
restrict our search to a finite dimensional space. 

One of the reason why the finite element method is powerful is that the
differential operators stay untouched and, instead, what is to be simplified is
\emph{the set in which the solutions live}. This fact makes it possible to
derive a very simple yet powerful property of the finite element approximations,
namely, the \emph{orthogonality property of the error}. We construct a sequence
of subspaces $V_h \subset V$ such that $V_h = \Span\{v_i\}_{i=1}^n$, with $n$
depending on $h$. Then we restrict the weak problem to $V_h$, which inherits the
norm from $V$: given $f \in V'$, find $u_h \in V_h$ s.t.
\begin{equation} \label{eqn:weak_2} \marginpar{Discrete weak form}
a(u_h,v_h) = \langle f,v_h \rangle \quad \forall v_h\in V_h.
\end{equation}
The element $u_h$ will be our candidate approximate solution. Since $V_h$ is a Hilbert space, and the bilinear operator $a(\cdot, \cdot)$ is coercive in the entire $V$, then also Problem~\eqref{eqn:weak_2} satisfies the hypotheses of Lax-Milgram lemma, and we conclude that there exists a unique solution. For completeness, however, we shall  prove its existence also using Ritz method.

Since $u_h \in V_h$, there exist a unique set of coefficients $\{u^j\}_{j=1}^n$ such that
\[
u_h = \sum_{j} u^j v_j.
\]
The discrete counterpart of the weak problem \ref{eqn:weak_2} then becomes:
\[
a\left(\sum_{j} u^j v_j ,v_h\right) = \langle f,v_h \rangle \quad \forall v_h\in V_h.
\]
In particular, it will suffice for us to check it for a basis of $V_h$:
\[
\sum_{j} a(v_j, v_i)  u^j = \langle v_i,f \rangle \quad \forall i=1,\dots,n.
\]
Here we have rearranged the objects in the brackets and used linearity to make it clearer that this is a matrix identity: if $A$ is the matrix whose entries are $A_{ij}=a(v_j, v_i)$, then we have to solve the linear system
\[
A \mathbf{u} = \mathbf{f}
\]
where $\mathbf{u}=\{u^i\}_{i=1}^n$ and $\mathbf{f}=\{\langle f, v_i\rangle\}_{i=1}^n$.
In particular, $A$ is clearly symmetric and positive definite due to the coercivity of $a$:
\[
\mathbf{u}^T A \mathbf{u} = a(\sum_{i} u^i v_i, \sum_{j} u^j v_j) \ge \alpha \norm{\sum_{i} u^i v_i}^2 \ge 0
\]
and this, by linearity, is zero if and only if every $u^i$ is zero. We conclude that $A$ is non singular, hence $u_h$ exists and is unique.

We now seek a way to control \emph{a priori} the error introduced by the restriction to $V_h$.
\begin{lemma}[Ceà]\marginpar{Ceà's lemma} \label{lemma:cea}
In the setting of Lax-Milgram lemma, let $u\in V$ be the solution of the weak problem~\eqref{eqn:weak_laxmilgram} and $u_h \in V_h$ a solution of~\eqref{eqn:weak_2}. Then:
\[
\norm{u - u_h} \le \frac{\norm{A}}{\alpha} \inf_{v_h \in V_h} \norm{u - v_h}.
\]
\end{lemma}
\begin{proof}
Observe that, since $u$ solves the weak problem in the whole $V$, then also
\[
a(u,v_h) = \langle f,v_h \rangle \quad \forall v_h\in V_h.
\]
By linearity, it follows that
\[
a(u - u_h,v_h) = 0 \quad \forall v_h\in V_h.
\]
In particular, this is also true if we substitute $v_h$ with $v_h - u_h$, which is still in $V_h$:
\[
a(u - u_h, v_h - u_h) = 0 \quad \forall v_h\in V_h.
\]
This is an orthogonality property of the error. Now we exploit the properties of $a$:
\begin{align}
\alpha \norm{u - u_h}^2 & \le a(u - u_h, u - u_h) \\
& = a(u - u_h, u - v_h) + a(u - u_h, v_h - u_h) \\
& = a(u - u_h, u - v_h) \\
& \le \norm{A} ~\norm {u - u_h} ~ \norm {u - v_h} \quad \forall v_h \in V_h.
\end{align}
The thesis follows by simplifying $\norm{u - u_h}$ on both sides of the inequality, and taking the infimum over $v_h \in V_h$.
\end{proof}

The lemma shows that the error in the finite element approximation is proportional to the best possible approximation in $V_h$, i.e., the error is controlled by the distance between the exact solution $u$ and the finite dimensional space $V_h$. This will be exploited in the next sections to derive both \emph{a-priori} and \emph{a-posteriori} error estimates for the finite element method.

\subsection{The one-dimensional case}

Let $\Omega = (a,b)$. We consider a set of $n+2$ points $\Set{x_i}_{i=0}^{n+1}$ such that
\[
a = x_0 < x_1 < \dots < x_{n+1} = b.
\]
To keep things simple, let
\[
x_i = a+ih, \quad h = \frac{b-a}{n+1}.
\]
Let $V=H_0^1((a,b))$. Now consider
\[
V_h = \Set{v \in C^0([a,b]): \restr{v}{[x_i, x_{i+1}]}\in \P^1([x_i, x_{i+1}]) \,\forall i=0,\dots,n, \, v(a)=v(b)=0}
\]
where $\P^1$ denotes the space of polynomials of degree at most 1. This space has dimension $n$ and contains piecewise linear functions on $[a,b]$ which are zero on the boundary.

We proceed to find a basis for it: let $v^i(u) := u(x_i)$ the evaluation of $u$ in the node $x_i$, for $i=1,\dots,n$. If $u$ were in $\D$, then $v^i$ would act as a Dirac delta $\delta(x-x_i)$, since
\[
\langle v^i, u \rangle = \int_\Omega \delta(x-x_i) u(x) \diff x = u(x_i).
\]
\begin{figure}[!htb]
\centering
\includegraphics{p1_basis_1d.pdf}
\caption{The basis functions $v_i$ in the 1D case.}
\label{fig:1d_basis}
\end{figure}
We construct functions $v_j \in V_h$ such that
\[
v^i(v_j) = \delta_{ij} = \begin{cases}
1 \quad \text{if } i=j \\
0 \quad \text{if } i\ne j
\end{cases}
\quad \forall i,j \in \{1,\dots,n\}.
\]
For every node $x_j$ in the interior of $[a,b]$ we are considering a piecewise linear function that is one on that node and zero on any other node. It is clear that the functions $\{v_j\}_{j=1}^{n}$\footnote{These functions are also called "Hat functions" for their peculiar hat form.} form a basis for the space $V_h$. These are the so-called \emph{shape functions}. Instead, the $v^i$-s form a basis for $V_h'$ and are called the \emph{nodal basis functions}. The coefficients $\{u^i\}_{i=1}^n\in \R$ such that $u_h = \sum_{i=1}^{n} u^i v_i$ are the \emph{degrees of freedom} (DoFs) of the finite element function $u_h$. 

In the one-dimensional case, the entries of the matrix $A$ are of the form
\[
A_{ij} = \int_a^b v_j'(x) v_i'(x) \diff x.
\]
Since the basis functions have a compact support, it is immediate to notice that $A_{ij} = 0$ whenever the supports of the involved basis functions do not intersect. This is a distinguishing feature of a finite element method: the matrix $A$ is constructed in a way that makes it \emph{sparse}. In particular, in this case we have
\[
A_{ij} = \begin{cases}
0 \quad &\text{if } \abs{i-j}>1 \\
-\frac{1}{h} \quad &\text{if } \abs{i-j}=1 \\
\frac{2}{h} \quad &\text{if } i=j
\end{cases}.
\]
Notice that we have obtained the same matrix we would get with a finite difference method of the same order (with the exception of a scaling factor of $1/h$).

\section{Finite dimensional spaces}

One of the key aspects of the finite element method is the construction of
finite dimensional spaces that approximate infinite dimensional Sobolev spaces,
with good approximation properties. Independently of how we construct such
finite dimensional spaces, we can exploit the fact all such spaces are
iso-morphic to $\R^n$ for some $n$. This in turns implies that any bilinear form
defined on such spaces can be represented as a matrix, and linear functionals
defined on such spaces can be represented as (co-)vectors, opening up the
possibility to use the machinery of linear algebra to solve the resulting
(non-)linear systems.

Let $V$ be a Banach space on the domain $\Omega$, and let $V_h$ be an $h$-parameter family of finite dimensional discretizations of $V$ with finite dimension $n$. We say that $V_h$ is a conforming discretization of $V$, if, for any $h\in \R$, the space $V_h$ is a proper linear subspace of $V$, i.e., $V_h \subseteq V$. 

\subsection{Basis functions}

Generally speaking, we indicate with the set $\{v_i\}_{i=1}^n$ a choice of (linearly independent) basis functions for $V_h$, i.e., 
\[
V_h = \Span\{v_i\}_{i=1}^n, \qquad \forall u_h \in V_h , \quad \exists! \quad \{u^i\}_{i=1}^n \text{ s.t. } u_h(x) = \sum_{i=1}^n u^i v_i(x),
\]
The functions $v_i$ are often also called \emph{shape functions}, and the coefficients $\{u^i\}_{i=1}^n$ are called \emph{degrees of freedom} (DoFs).

\subsection{Nodal functions}

We denote with $V_h'$ the dual space of $V_h$, i.e., the space of continuous linear functionals on $V_h$. Since $V_h$ is finite dimensional, so is $V_h'$, and its dimension is the same of $V_h$. We identify a canonical basis for the dual space $V_h'$ given by a set of $n$ linear functionals on $V_h$, given by $\{v^i\}_{i=1}^n$ such that 
\[
V_h' = \Span\{v^i\}_{i=1}^n, \qquad \forall f_h \in V_h', \quad \exists! \quad \{f_i\}_{i=1}^n \text{ s.t. } f_h(x) = \sum_{i=1}^n f_i v^i(x).
\]

The elements of the canonical basis also called \emph{nodal functions} and are
denoted by $\{v^i\}_{i=1}^n$. A generic element of $V_h'$ is often called a
\emph{co-vector}. The basis (or \emph{shape functions}) and the dual basis (or
\emph{nodal functions}) are related by the following property:
\[
v^i(v_j) = \delta_{ij} = \begin{cases}
1 \quad &\text{if } i=j \\
0 \quad &\text{if } i\ne j
\end{cases}
\quad \forall i,j \in \{1,\dots,n\}.
\]

\subsection{Projection operators}

The functions $v^i$ are linear functionals on $V_h$. By the Hann-Banach extension theorem, we can extend them to functionals in $V'$. We choose $n$ such extensions and call them $\{\tilde{v}^i\}_{i=1}^n$. They are chosen such that
\[
\tilde{v}^i \in V', \qquad \tilde{v}^i(u_h) = v^i(u_h) \quad \forall u_h \in V_h.
\]

With this choice of extensions, we can define a projection operator $\Pi: V \to V_h$ as
\[
\Pi(u) = \sum_{i=1}^n \tilde{v}^i(u) v_i = \sum_{i=1}^n \duality{\tilde{v}^i, u} v_i, 
\]
where $\tilde{v}^i(u)=\duality{\tilde{v}^i, u}$ is the value of the functional $\tilde{v}^i$ applied on the function $u$, and it represents the $i$-th component (or the $i$-th \emph{degree of freedom}) of the function $\Pi(u)$ in the basis $\{v_i\}_{i=1}^n$. The projection operator $\Pi$ maps any function $u \in V$ to its finite dimensional approximation in $V_h$. 

In particular, $\Pi$ is indeed a projection, since for any $u_h\in V_h$, we have 
\[
\begin{split}
  \Pi(u_h) = & \sum_{i=1}^n \tilde{v}^i(u_h) v_i = \sum_{i=1}^n v^i(u_h) v_i = \\
  &\sum_{i=1}^n v^i\left(\sum_{j=1}^n u^j v_j\right) v_i = \sum_{i=1}^n \sum_{j=1}^n u^j  v^i(v_j) v_i  = \\
  &\sum_{i=1}^n \sum_{j=1}^n u^j  \delta_{ij}  v_i  = \sum_{i=1}^n u^i v_i = u_h.
\end{split}
\]

\subsection{Degrees of freedom}
The set of nodal functions $\Sigma := \{v^i\}_{i=1}^n$ is used to map the finite
dimensional space $V_h$ to the vector of coefficients of the basis functions,
i.e., it can be seen as an operator $\Sigma: V_h \to \R^n$, defined as
\[ 
\R^n \ni \Sigma(u_h) = \left(v^1(u_h), v^2(u_h), \ldots, v^n(u_h)\right) = \left(\duality{v^1, u_h}, \duality{v^2, u_h}, \ldots, \duality{v^n, u_h}\right).
\]

$\Sigma$ is clearly invertible and its inverse $\Sigma^{-1}$ is simply defined as 
\[
\Sigma^{-1}(\mathbf{u}) = \sum_{i=1}^n u^i v_i,
\]
where $\mathbf{u} = \{u^i\}_{i=1}^n$ is the vector of coefficients.

\rev{Talk about the difference between vectors and co-vectors, and why it is important to use them correctly, i.e., there is a difference between $v^i$ and $v_i$, and there is a difference between the ``units of measure'' of the two.}

\subsection{Triangulations}

One of the most common way to define finite dimensional spaces is to use a
\emph{triangulation} of the domain $\Omega$. The idea is to partition the domain
$\Omega$ into simple subdomains (like triangles or quadrilaterals in 2D and
tetrahedra, hexahedra, prisms, or pyramids in 3D) where it is easy to define
local basis functions, and to compute integrals.

Formally speaking we partition $\Omega$
\begin{equation*}
\Omega = \mathring{\overline{\Bigl(\bigcup_{m=1}^M T_m \Bigr)}}
\end{equation*}
into a set of simple (closed, Lipschitz, and convex) subdomains $T_m$ (called \emph{elements} or \emph{cells})
such that 
\begin{equation}
T_i \cap T_j = \begin{cases}\emptyset \\ \text{vertex} \\ \text{common edge or face}.\end{cases}
\end{equation}
We call this partition a \emph{triangulation} $\mathcal{T}_h$, where
``triangulation'' is just an historical term, and we assume that we can have
triangulations composed of different shapes.

In what follows, we assume for simplicity that a triangulation is made by
elements or cells that are all of the same type (e.g. all triangles, all
quadrilaterals, etc.). In this case we say that the triangulation is
\emph{regular}.

Figure~\ref{fig:triangulation} shows an example of a triangulation of a two-dimensional domain. The domain is divided into a set of triangular elements, which are used to approximate the geometry and define finite element spaces.

The two tables below provide the minimal data structure required to describe this triangulation. The first table lists the coordinates of the vertices, while the second table specifies the connectivity of the triangles, i.e., which vertices form each triangle.

\begin{figure}[!htb]
\centering
\includegraphics[width=.7\textwidth]{ball_2.pdf}
\caption{An example of a triangulation in $\R^2$. The domain is divided into triangular elements.}
\label{fig:triangulation}
\end{figure}

\begin{table}[!htb]
  \centering
  \caption{Miniimal data structure of a triangulation: a list of vertex coordinates (left), and a list of triangle connectivity (right).}
  \label{tab:vertex-coordinates}
  \begin{minipage}[t]{0.45\textwidth}
    \centering
  \begin{tabular}{|c|c|c|}
  \hline
  \textbf{Vertex} & $x_1$ & $x_2$ \\ \hline
  1 & 0.0 & 0.0 \\ \hline
  2 & 0.5 & 0.0 \\ \hline
  3 & 1.0 & 0.0 \\ \hline
  4 & 0.5 & 0.5 \\ \hline
  5 & 0.0 & 0.5 \\ \hline
  6 & 1.0 & 0.5 \\ \hline
  7 & 1.5 & 0.5 \\ \hline
  8 & 2.0 & 0.5 \\ \hline
  9 & 0.5 & 1.0 \\ \hline
  10 & 1.5 & 1.0 \\ \hline
  \end{tabular}
\end{minipage}
\begin{minipage}[t]{0.45\textwidth}
  \centering
  \begin{tabular}{|c|c|c|c|}
    \hline
    \textbf{Triangle} & $v_1$ & $v_2$ & $v_3$ \\ \hline
    1 & 1 & 4 &  5 \\ \hline
    2 & 1 & 2 &  4  \\ \hline
    3 & 2 & 7 &  4 \\ \hline
    4 & 2 & 3 &  7 \\ \hline
    5 & 4 & 6 &  5 \\ \hline
    6 & 4 & 7 &  6 \\ \hline
    7 & 3 & 8 &  7 \\ \hline
    8 & 5 & 6 &  9 \\ \hline
    9 & 6 & 10 &  9  \\ \hline
    10 & 6 & 7 &  10 \\ \hline
    11 & 7 & 8 &  10 \\ \hline
    \end{tabular}
\end{minipage}
\end{table}

Implementing such data structure require at the bare minimum two containers, one
for the vertex coordinates (generally stored as floating point numbers) and one
for the triangles (generally stored as a list of vertex indices, pointing to
entries in the first container). In general, it may be necessary also to store
(or to compute on the fly) additional information, such as edge connectivity
(i.e., which triangles share a common edge), vertex connectivty (i.e., which
triangles share a common vertex, or which edges share a common vertex), etc.
This information is useful for many purposes, such as mesh refinement, mesh
generation, and visualization. In practice, the data structure may be more
complex than the one shown in the example, depending on the specific
requirements of the application.

\subsection{Finite element}

\begin{definition}[Ciarlet, 1978] \marginpar{Definition of finite element}
A \emph{finite element} is a triplet $(K,P,\Sigma)$, where:
\begin{romanlist}
\item $K\subset \R^n$ is a closed subset with piecewise smooth boundary;
\item $P$ is a finite dimensional space of \emph{shape functions} $v_i$;
\item $\Sigma$ is a set of basis functions $v^i$ for the space $P'$.
\end{romanlist}
\end{definition}

We emphasize that this definition is of a \emph{local} finite element: the set $K$ should be thought as an element of the triangulation on which shape functions are defined. The set $P$ should be thought as a space of polynomials and the set $\Sigma$ is the set of nodal basis functions extended through Hahn-Banach.


\section{Lagrangian finite elements}