%!TeX encoding = utf8
%!TeX spellcheck = en_US

\documentclass[a4paper, 11pt]{book}

% Suppress undesired warnings
\RequirePackage{silence}
\WarningFilter{classicthesis}{Using package}
\WarningFilter{remreset}{The remreset package}

\usepackage[english]{babel}
\usepackage{amsmath, amssymb, amsthm}
\usepackage{geometry}
\usepackage{emptypage}
\usepackage{graphicx, subfig}
\usepackage{enumitem}
\usepackage{marginnote}
\usepackage[autostyle,italian=guillemets]{csquotes}
\usepackage[colorlinks]{hyperref}
%\usepackage[style=numeric, sorting=ynt, backend=biber, hyperref, backref]{biblatex}
\PassOptionsToPackage{nopatch}{microtype}
\usepackage[eulerchapternumbers, palatino=false, parts=true]{classicthesis}
\usepackage{pgfplots}
\usepackage{nicematrix}
\usepackage{mathtools}
\usepackage{braket}
%\usepackage{widebar}

% Font
\usepackage[proportional, oldstyle]{cochineal}
\usepackage[cochineal, vvarbb]{newtxmath}
\usepackage[enum]{tabfigures}
\microtypesetup{kerning=true}
\DeclareSymbolFont{cochinealit}{\encodingdefault}{\familydefault}{m}{it}
\DeclareMathSymbol{f}{\mathalpha}{cochinealit}{`f}
\DeclareSymbolFontAlphabet{\mathit}{cochinealit}

% Figures
\graphicspath{{Figures/}}
\captionsetup{format=hang}
\pgfplotsset{/pgf/number format/use comma,compat=newest}
\usetikzlibrary{arrows.meta}

% Layout
\geometry{%showframe,
	width=30pc, height=54pc, vmarginratio=1000:1414}
\setlength\parindent{1em}
\setlength\marginparwidth{7em}

\hypersetup{%hidelinks,
	linkcolor=RoyalBlue}

% Bibliography
%\addbibresource{Bibliography.bib}
%\renewcommand*{\nameyeardelim}{\addcomma\space}

% Headings
\newcommand\setupheading[1]{%
	\cleardoublepage\phantomsection
	\markboth{\spacedlowsmallcaps{#1}}{\spacedlowsmallcaps{#1}}%
}
\clearplainofpairofpagestyles
\renewcommand{\sectionmark}[1]{%
	\markright{\textsc{\thesection}\quad\spacedlowsmallcaps{#1}}}


% Lists
\setlist[itemize]{itemsep=0pt}

\newlist{romanlist}{enumerate}{1}
\setlist[romanlist]{label=\normalfont(\roman*),itemsep=0pt}

\newlist{steps}{enumerate}{1}
\setlist[steps]{label=\normalfont\textbf{Step \arabic*.}, ref=\arabic*,
	wide=0pt, leftmargin=0pt, listparindent=\parindent, parsep=0pt, itemsep=\baselineskip}

\newlist{case}{enumerate}{1}
\setlist[case]{label=\normalfont\textbf{Caso \arabic*.}, ref=\arabic*,
	wide=0pt, leftmargin=0pt, listparindent=\parindent, parsep=0pt, itemsep=\baselineskip}

\newlist{exercises}{enumerate}{1}
\setlist[exercises]{ref=\arabic*,
	labelindent=0pt,
	labelwidth=20pt,
	labelsep=10pt,
	itemindent=0pt,
	%leftmargin=\labelindent+\labelwidth+\labelsep,
	listparindent=\parindent,
	parsep=0pt,
	itemsep=\baselineskip,
	label=\normalfont[\arabic*]}

% Theorems
\newtheoremstyle{plain}{}{}{\itshape}{}{}{}{ }{%
	\textbf{\thmname{#1}\thmnumber{ \scshape#2.}}\thmnote{ (#3).}}
\newtheoremstyle{definition}{}{}{}{}{}{}{ }{%
	\textbf{\thmname{#1}\thmnumber{ \scshape#2.}}\thmnote{ (#3).}}

\theoremstyle{plain}
\newtheorem{theorem}{Theorem}[chapter]
\newtheorem{corollary}[theorem]{Corollary}
\newtheorem{proposition}[theorem]{Proposition}
\newtheorem{lemma}[theorem]{Lemma}

\theoremstyle{definition}
\newtheorem{definition}[theorem]{Definition}
\newtheorem{remark}[theorem]{Remark}
\newtheorem{example}[theorem]{Example}

\numberwithin{equation}{chapter}
\numberwithin{figure}{chapter}
\numberwithin{table}{chapter}

% Abbreviations
\newcommand\restr[2]{{\left.\kern-\nulldelimiterspace#1
	\vphantom{\big|}\right|_{#2}}}
\newcommand\citetheorem[1]{\citeauthor{#1} \cite{#1}}
\newcommand\authorimage[3][.9]{%
	\marginpar{\includegraphics[width=#1\linewidth]{#2}\\[1ex]
	#3}}
\let\bm\boldsymbol
\let\.\boldsymbol
\DeclareRobustCommand{\highdash}{\raisebox{.2ex}{--}}
\def\D{\mathcal{D}} % Distributions
\def\P{\mathcal{P}} % Polynomials
\def\L{\mathcal{L}} % Linear continuous operators
\def\F{\mathcal{F}} % Fourier transform
\def\N{\mathbb{N}}
\def\Z{\mathbb{Z}}
\def\R{\mathbb{R}}
\def\C{\mathbb{C}}
\def\T{\mathcal{T}} % Triangulation
\def\E{\mathcal{E}} % Edges
\def\1{\mathbb{1}}
\DeclareMathOperator{\mcd}{mcd}
\DeclareMathOperator{\re}{Re}
\DeclareMathOperator{\im}{Im}
\DeclareMathOperator{\ran}{ran}
\DeclareMathOperator{\rk}{rank}
\DeclareMathOperator{\ind}{ind}
\DeclareMathOperator{\Span}{span}
\DeclareMathOperator{\diag}{diag}
\DeclareMathOperator{\tr}{tr}
\DeclareMathOperator{\adj}{adj}
\DeclareMathOperator{\sgn}{sgn}
\DeclareMathOperator{\conv}{conv}
\DeclareMathOperator{\diam}{diam}
\DeclareMathOperator{\dvg}{div}
\DeclareMathOperator{\RE}{RE} % Lifting from face
\DeclareMathOperator{\SZ}{SZ} % Scott-Zhang operator
\DeclareMathOperator{\card}{card} % Cardinality

% Definition of vvvert symbol for the three-bar norm 
% Math symbol font matha
\DeclareFontFamily{U}{matha}{\hyphenchar\font45}
\DeclareFontShape{U}{matha}{m}{n}{
      <5> <6> <7> <8> <9> <10> gen * matha
      <10.95> matha10 <12> <14.4> <17.28> <20.74> <24.88> matha12
      }{}
\DeclareSymbolFont{matha}{U}{matha}{m}{n}
\DeclareFontSubstitution{U}{matha}{m}{n}
% Math symbol font mathb
\DeclareFontFamily{U}{mathx}{\hyphenchar\font45}
\DeclareFontShape{U}{mathx}{m}{n}{
      <5> <6> <7> <8> <9> <10>
      <10.95> <12> <14.4> <17.28> <20.74> <24.88>
      mathx10
      }{}
\DeclareSymbolFont{mathx}{U}{mathx}{m}{n}
\DeclareFontSubstitution{U}{mathx}{m}{n}
% Symbol definition
\DeclareMathDelimiter{\vvvert}{0}{matha}{"7E}{mathx}{"17}

\DeclarePairedDelimiter{\abs}{\lvert}{\rvert}
\DeclarePairedDelimiter{\norm}{\lVert}{\rVert}
\DeclarePairedDelimiter{\tnorm}{\vvvert}{\vvvert} % Three-bar norm
\DeclarePairedDelimiter{\jump}{\llbracket}{\rrbracket} % Jump
\DeclarePairedDelimiter{\avg}{\{\!\!\{}{\}\!\!\}} % Average

% Revision
\usepackage[many]{tcolorbox}
\long\def\rev#1{\par
	\begin{tcolorbox}[enhanced, before skip=2ex, after skip=2ex,
		arc=0mm, outer arc=0mm, boxrule=0mm, toprule=1pt, bottomrule=1pt,
		colback=yellow!20!white, width=\linewidth, left=0pt]
    \normalfont\normalsize\itshape#1
	\end{tcolorbox}
	\par}
%\long\def\nrev#1{}
\let\nrev\rev

% Tweaks
\renewcommand{\epsilon}{\varepsilon}
\renewcommand{\theta}{\vartheta}
\renewcommand{\rho}{\varrho}
\renewcommand{\phi}{\varphi}
\let\bar\widebar
\mathtoolsset{showonlyrefs}
\makeatletter
\renewcommand\marginpar[1]{
	\marginnote{\graffito@setup#1}
}
\makeatother
%\renewenvironment{equation*}{\begin{equation}}{\end{equation}}
%\def\[{\begin{equation}}
%\def\]{\end{equation}}

\begin{document}

\mainmatter

%\begin{titlepage}
    \centering
    \vspace*{2cm}
    
    \Huge
    \textbf{Numerical Methods for Partial Differential Equations}
    
    \vspace{1.5cm}
    
    \LARGE
    Lecture notes of the course held by prof. L. Heltai\\
    University of Pisa, AY 2023/2024 -- 2024/2025
    
    \vspace{2cm}
    
    \textbf{Stefano Mancini\\
    Andrea Benvenuti\\
    Luca Heltai}
    
    \vfill
    
    \includegraphics[width=\textwidth]{fichera} % Replace with your own image
    
    \vfill
    
    \Large
    Department of Mathematics\\
    University of Pisa\\
    \today
    
\end{titlepage}


%****************************************************************
% Contents
%****************************************************************

%\begingroup
%\manualmark
%\setupheading{\contentsname}
%\tableofcontents
%\endgroup

%\part{Theory}

% !TeX source = ../main.tex

\chapter[Overview]{Overview}

%****************************************************************
% Lezione 27 febbraio
%****************************************************************

\section{Construction of the approximation, Ceà's lemma}

Let $\Omega$ be an open, bounded, Lipschitz subset of $\R^n$. Let also $\partial\Omega$ be its boundary.
Consider the following Poisson problem:
\[
\begin{cases} \marginpar{Strong formulation}
-\Delta u = f \qquad &\text{in $\Omega$} \\
u =  0 \qquad &\text{on $\partial \Omega$}
\end{cases}
\]
where, as usual,
\[
\Delta = \sum_{i=1}^n \frac{\partial^2}{\partial x_i^2}.
\]
If we want to find a numerical solution of this problem, two approaches can be followed:
\begin{itemize}
\item discretize $-\Delta$ (\emph{finite differences});
\item consider the \emph{weak formulation} of the problem (\emph{finite elements}).
\end{itemize}

\rev{Maybe add some reference to the finite difference construction.}

Finite differences work well if $\Omega$ is a nice domain (e.g. a square, a cube and so on) and if $f$ is regular enough (e.g. continuous). In applications, however, the domain's boundary may not be nice at all, and $f$ may not be $C^0$ but, instead, at most $L^2$ or, worse, it may have singularities (e.g., it may be a Dirac delta).
The finite element approach tries to overcome these difficulties. The advantage is that we don't have to discretize the differential operator anymore, but instead we integrate by parts and dump the derivatives onto a test function. Also, the solution $u$ no longer needs to be defined pointwise.

In order to derive a weak formulation for the problem above, let $\phi \in \D$ a test function (usually $\D=C_0^\infty(\Omega)$). We multiply by $\phi$ both sides of the PDE in the strong formulation and then integrate:
\[
\int_\Omega -\Delta u \phi = \int_\Omega f \phi.
\]
If we integrate by parts and suppose that $\restr{u}{\partial \Omega} = \restr{\phi}{\partial\Omega} = 0$, we get:
\[
\int_\Omega \nabla u \nabla \phi = \int_\Omega f \phi.
\]
More in general, if we replace $\D$ with some function space $V$, the weak formulation will be: given $f\in V'$, find $u\in V$ such that
\begin{equation} \label{eqn:weak_1} \marginpar{Weak form (1)}
\int_\Omega \nabla u \nabla v = \int_\Omega f v \quad \forall v\in V.
\end{equation}
with the agreement that, being $f \in V'$, the RHS is in fact a duality.

Some further clarification is needed:
\begin{itemize}
\item Does the LHS make sense? We need that
\[
\int_\Omega | \nabla u \nabla v | < +\infty
\]
hence a possibility is to require $\nabla u, \nabla v \in L^2(\Omega)$, i.e. be weak derivatives.
\item We said before that $u$ and $v$ are assumed to be zero on the domain boundary. What does it mean? Remember that, for instance, Sobolev space functions are not well defined on a zero measure set, such as $\partial \Omega$. Hence, a workaround (the trace) will be needed.
\end{itemize}
Let us choose
\[
V:= H^1_0(\Omega) = \Set{u \in L^2(\Omega) : \nabla u \in L^2(\Omega), \restr{u}{\partial\Omega}=0}.
\]
It is well known that $H^1_0(\Omega) = \overline{C^\infty_0(\Omega)}^{\norm{\cdot}_{1,
2}}$, where $\norm{\cdot}_{1,2}$ is the Sobolev norm induced by the scalar product
\[
(u,v)_{1,2} = \int_\Omega uv + \int_\Omega \nabla u \nabla v.
\]
In particular, the space $V$ with this scalar product is a Hilbert space.

In order to try to approximate a solution of the weak problem, we first need to prove that such a solution exists. To do so, we take advantage of the Lax-Milgram lemma:
\begin{lemma}[Lax-Milgram]\label{lemma:lax-milgram}\marginpar{Lax-Milgram lemma}
Let $V$ a Hilbert space and let $a: V\times V \to \R$ be a bilinear operator such that:
\begin{itemize}
\item $a$ is \emph{bounded}, i.e. $\exists c>0$ s.t. $a(u,v) \le c \norm{u}_V \norm{v}_V$ for every $u,v \in V$ \\ (sometimes, we refer to the constant $c$ as $\norm{A}_*$);
\item $a$ is \emph{coercive} (or \emph{bounding}), i.e. $\exists \alpha >0$ s.t. $a(u,u) \ge \alpha \norm{u}_V^2$.
\end{itemize}
Then, given $f\in V'$, the following problem
\begin{equation}\label{eqn:weak_laxmilgram}
a(u,v) = \langle f,v \rangle \quad \forall v\in V
\end{equation}
admits a unique solution $u_0$. Moreover, $u_0$ satisfies the following inequality:
\[
\norm{u_0}_V \le \frac{\norm{f}_{V'}}{\alpha}.
\]
\end{lemma}

In the Poisson problem's case, the bilinear operator is
\[
a(u,v) = \int_\Omega \nabla u \nabla v,
\]
which is clearly bounded and coercive if $V=H_0^1(\Omega)$. In particular, coercivity follows from using Poincaré inequality to observe that the norm in $V$ is equivalent to the $H^1$ seminorm
\[
\abs{u}_{1,2} = \norm{\nabla u}_{L^2}.
\]
Hence, the weak formulation of the Poisson problem admits a unique solution by the Lax-Milgram lemma.

The reason why the finite element method is powerful is that the differential operators stay untouched and, instead, what is to be simplified is \emph{the set in which the solutions live}. We construct a sequence of subspaces $V_h \subset V$ such that $V_h = \Span\Set{v_i}_{i=1}^n$, with $n$ depending on $h$. Then we restrict the weak problem to $V_h$, which inherits the norm from $V$: given $f \in V'$, we find $u_h \in V_h$ s.t.
\begin{equation} \label{eqn:weak_2} \marginpar{Weak form (2)}
a(u_h,v_h) = \langle f,v_h \rangle \quad \forall v_h\in V_h.
\end{equation}
Our $u_h$ will be our candidate approximate solution. We shall first prove its existence.

Since $u_h \in V_h$, there exist unique coefficients $\Set{u^j}_{j=1}^n$ such that
\[
u_h = \sum_{j=1}^n u^j v_j.
\]
If we use the Einstein notation, i.e. we omit the summation on repeated indices, the weak problem \ref{eqn:weak_2} becomes:
\[
a(u^j v_j ,v_h) = \langle f,v_h \rangle \quad \forall v_h\in V_h.
\]
In particular, in order to achieve this equality, it will suffice for us to check it for a basis of $V_h$:
\[
a(v_i, v_j)  u^j = \langle v_i,f \rangle \quad \forall i=1,\dots,n.
\]
Here we have rearranged the objects in the brackets and used linearity to make it clearer that this is a matricial identity: if $A$ is the matrix whose entries are $A_{ij}=a(v_i, v_j)$, then we have to solve the linear system
\[
A \mathbf{u} = \mathbf{f}
\]
where $\mathbf{u}=(u^i)_{i=1}^n$ and $\mathbf{f}=(\langle v_i,f \rangle)_{i=1}^n$.
In particular, $A$ is clearly symmetric and positive definite due to the coercivity of $a$:
\[
\mathbf{u}^T A \mathbf{u} = a(u^i v_i, u^i v_i) \ge \alpha \norm{u^i v_i}^2 \ge 0
\]
and this, by linearity, is zero if and only if every $u^i$ is zero. We conclude that $A$ is non singular, hence $u_h$ exists and is unique.

We now seek a way to control \emph{a priori} the error introduced by the restriction to $V_h$.
\begin{lemma}[Ceà]\marginpar{Ceà's lemma} \label{lemma:cea}
In the setting of the Lax-Milgram lemma, let $u\in V$ be the solution of the weak problem~\eqref{eqn:weak_laxmilgram} and $u_h \in V_h$ a solution of~\eqref{eqn:weak_2}. Then:
\[
\norm{u - u_h} \le \frac{\norm{A}}{\alpha} \inf_{v_h \in V_h} \norm{u - v_h}.
\]
\end{lemma}
\begin{proof}
Observe that, since $u$ solves the weak problem in the whole $V$, then also
\[
a(u,v_h) = \langle f,v_h \rangle \quad \forall v_h\in V_h.
\]
By linearity, it follows that
\[
a(u - u_h,v_h) = 0 \quad \forall v_h\in V_h.
\]
In particular, this is also true if we substitute $v_h$ with $v_h - u_h$, which is still in $V_h$:
\[
a(u - u_h, v_h - u_h) = 0 \quad \forall v_h\in V_h.
\]
This is an orthogonality property of the error. Now we exploit the properties of $a$:
\begin{align}
\alpha \norm{u - u_h} & \le a(u - u_h, u - u_h) \\
& = a(u - u_h, u - v_h) + a(u - u_h, v_h - u_h) \\
& = a(u - u_h, u - v_h) \\
& \le \norm{A} \norm {u - u_h} \norm {u - v_h} \quad \forall v_h \in V_h.
\end{align}
The conclusion follows by simplifying $\norm{u - u_h}$ on both sides of the inequality.
\end{proof}

The meaning of Ceà's inequality is: it suffices to control the error only in $V_h$ in order to have a good approximation of $u$. This is a step in the right direction, because in practice the functions in $V_h$ will be polynomials, hence their derivatives will be well defined and we won't need to approximate them. However, there are still some problems left:
\begin{itemize}
\item How do we approximate an arbitrary domain $\Omega$?
\item How do we approximate $V$ with $V_h$? Meaning, we seek to construct $V_h$ in such a way that $\text{dist}(V,V_h) = c h^k$ for some $k$, where
\[
\text{dist}(V,V_h) = \sup_{u\in V} \inf_{v_h \in V_h} \norm{u - v_h}.
\]
\item In practice, $u$ is unknown, but the RHS of the error estimate contains $u$. How do we manage this?
\end{itemize}



\section{An example: the 1-dimensional case}

\rev{this section could be written better.}

Let $\Omega = (a,b)$. In order to discretize it, we consider a set of points $\Set{x_i}_{i=1}^n$ such that
\[
a = x_0 < x_1 < \dots < x_{n+1} = b.
\]
To keep things simple, let
\[
x_i = a+ih, \quad h = \frac{b-a}{n+1}.
\]
As prompted in the previous section, let $V=H_0^1((a,b))$. Now consider
\[
V_h = \Set{v \in C^0([a,b]): \restr{v}{[x_i, x_{i+1}]}\in \P^1([x_i, x_{i+1}]) \,\forall i=0,\dots,n, \, v(a)=v(b)=0}
\]
where $\P^1$ denotes the space of polynomials of degree at most 1. This space has dimension $n$ and contains piecewise linear functions on $[a,b]$ which are zero on the boundary.

We proceed to find a basis for it: let $v^i(u) := u(x_i)$ the evaluation of $u$ in the node $x_i$, for $i=1,\dots,n$. If $u$ were in $\D$, then $v^i$ would act as a Dirac delta $\delta(x-x_i)$, since
\[
\langle v^i, u \rangle = \int_\Omega \delta(x-x_i) u(x) \,dx = u(x_i).
\]
We construct functions $v_j \in V_h$ such that
\[
v^i(v_j) = \delta_{ij} = \begin{cases}
1 \quad \text{if } i=j \\
0 \quad \text{if } i\ne j
\end{cases}
\quad \forall i,j \in \Set{1,\dots,n}.
\]
In practice, for every node $x_j$ in the interior of $[a,b]$ we are considering a piecewise linear function that is one on that node and zero on any other node. It is clear that the functions $\Set{v_j}_{j=1}^{n}$ form a basis for the space $V_h$. These are the so-called \emph{shape functions}. Instead, the $v^i$-s form a basis for $V_h'$ and are called the \emph{nodal basis functions}.

Let us now get back to the Poisson problem. In the 1-dimensional case, the entries of the matrix $A$ are of the form
\[
A_{ij} = \int_a^b v_i'(x) v_j'(x) \,dx.
\]
Since the nodal functions have been constructed in a \emph{localized} way, it is immediate to notice that $A_{ij} = 0$ whenever the supports of the involved nodal functions do not intersect. This is a distinguishing feature of a finite element method: the matrix $A$ is constructed in a way that makes it \emph{sparse}. In particular, in this case we have
\[
A_{ij} = \begin{cases}
0 \quad &\text{if } \abs{i-j}>1 \\
-\frac{1}{h} \quad &\text{if } \abs{i-j}=1 \\
\frac{2}{h} \quad &\text{if } i=j
\end{cases}.
\]
Notice that we have obtained the same matrix we would get with a finite difference method of the same order.
\rev{I am a bit uncertain about the last sentence: this is very close to the matrix we would get with second order centered finite differences... but the orders are different. In particular, the most simple centered finite difference approach results in an approximation of order 2.}

\section{The definition of finite element}

\rev{this section has not been revised yet.}

Let $V$ a Banach space \rev{check if $V$ is only Banach} and $A \in \L(V,V')$ a linear, continuous operator from $V$ to its dual space. Given $f \in V'$, we want to solve the problem $Au=f$ in $V'$, i.e.
\begin{equation} \label{eqn:op_pbm} \marginpar{Operatorial form}
\langle Au,v \rangle = \langle f,v \rangle \quad \forall v\in V.
\end{equation}
Let $V \supset V_h = \Span\Set{v_i}_{i=1}^n$, with $\dim V_h = n$. The space $V_h'$ is usually described in terms of the \emph{covectors}:
\[
V_h' = \Span\Set{v^i}_{i=1}^n.
\]
In particular, each $v^i$ is extended via the Hahn-Banach theorem to $V'$, in order to see $V_h'$ as a subspace of $V'$. With this notation, each $v^i$ is the dual vector of $v_i$, i.e.
\[
v^i(v_j) = \delta_{ij}.
\]
\rev{I'm not very sure about this. The $v^i$s are functions in $V'$, but $V_h'$ is not a subspace of $V'$.}
In particular, if $u_h\in V_h$, we have
\[
u_h = \sum_{i=1}^n u^i v_i = \sum_{i=1}^n v^i(u_h) v_i.
\]
This leads to the natural definition of the \emph{interpolation operator} in $V_h$:
\begin{align}
I_{V_h}: &V \to V_h \\
& u \mapsto \sum_{i=1}^n v^i(u) v_i = \sum_{i=1}^n u^i v_i.
\end{align}

\begin{remark}
$I_{V_h}$ is a projection. In fact:
\begin{align}
I_{V_h}(I_{V_h}(u)) &= v^i (u^j  v_j) v_i \\
&= u^j v^i(v_j) v_i \\
&= \delta_{ij} u^j  v_i \\
&= u^i  v_i.
\end{align}
\end{remark}

\begin{definition}[Ciarlet, 1978] \marginpar{Definition of finite element}
A \emph{finite element} is a triplet $(K,P,\Sigma)$, where:
\begin{romanlist}
\item $K\subset \R^n$ is a closed subset with piecewise smooth boundary;
\item $P$ is a finite dimensional space of \emph{shape functions} $v_i$;
\item $\Sigma$ is a set of basis functions $v^i$ for the space $P'$.
\end{romanlist}
\end{definition}



% !TeX source = ../main.tex

\chapter[A priori error estimates (Lax-Milgram case)]{A priori error estimates (Lax-Milgram case)}

\section{Scaling argument for Sobolev norms}

%%%%%
% Lecture of 20 of March
%%%%%
\begin{figure}[!ht]
\centering
\includegraphics{affine_mapping}
\caption{A triangle $T_m$ and its reference triangle $\hat{T}$. The affine mapping $F_m$ maps $\hat{T}$ to $T_m$.}
\label{fig:affine_mapping}
\end{figure}

Consider an affine mapping from a reference triangle $\hat T$, to a generic triangle $T_m$ (Figure~\ref{fig:affine_mapping}):
\begin{align}
F_m: &\hat{T} \to T_m \\
&\hat{x} \mapsto x = B \hat{x} + b.
\end{align}
We would like to analyze how Sobolev norms change under the action of $F_m$. Recall the definition of the Sobolev norm in $W^{k,p}$:
\[
\norm{v}^p_{k,p,\Omega} = \sum_{\abs{\alpha} \le k} \norm{D^\alpha v}^p_{L^p(\Omega)},
\]
with the corresponding seminorm:
\[
\abs{v}^p_{k,p,\Omega} = \sum_{\abs{\alpha} = k} \norm{D^\alpha v}^p_{L^p(\Omega)}.
\]

Let
\[
\rho_m := \sup_{\tilde{B} \subset T_m} \diam(\tilde{B}), \quad h_m := \inf_{T_m \subset \tilde{B}} \diam(\tilde{B}),
\]
and similarly for the reference triangle $\hat{T}$:
\[
\hat{\rho} := \sup_{\tilde{B} \subset \hat{T}} \diam(\tilde{B}), \quad \hat{h} := \inf_{\hat{T} \subset \tilde{B}} \diam(\tilde{B}),
\]
where $\tilde{B}$ denotes a ball in the right space (see Figure~\ref{fig:scaling}). We emphasize that both $\hat{\rho}$ and $\hat{h}$ are \emph{constants}.


\begin{figure}[!ht]
\centering
\includegraphics{triangle_scales}
\caption{The scales $\rho_m$ and $h_m$ for a triangle $T_m$, and $\hat \rho$, $\hat h$ for the reference triangle $\hat T$.}
\label{fig:scaling}
\end{figure}


We start with some considerations on the matrix $B$. First, we notice that if $x,y \in T_m$ we have
\[
\abs{x-y} = \abs{B(\hat{x}-\hat{y})} = \abs{B \xi}, \qquad \xi = \hat{x}-\hat{y}.
\]
By the definition of the norm of B, we have that
\begin{equation}
\norm{B} = \sup_{\abs{\xi} = \hat{\rho}} \frac{\abs{B \xi}}{\abs{\xi}} \le
\sup_{\abs{\xi} = \hat{\rho}} \frac{\abs{x-y}}{\hat{\rho}} \le \frac{h_m}{\hat{\rho}}.
\label{eqn:scaling_B}
\end{equation}
With a similar argument:
\begin{equation}
  \norm{B^{-1}} = \sup_{\abs{\xi} = \rho_m} \frac{\abs{B^{-1} \xi}}{\abs{\xi}} \le \frac{\hat{h}}{\rho_m},
  \label{eqn:scaling_B_inv}
\end{equation}
where both $\hat \rho$ and $\hat h$ are constant, and depend only on the choice of the reference triangle $\hat T$.
Also, since $F_m$ is affine, its differential is the matrix $B$ itself. This will come in handy to provide a simple formula for $\hat{D}^\alpha (v \circ F_m)$ when using the chain rule.

Consider a function $v \in W^{k,p}(T_m)$ on the triangle $T_m$. Then:
\begin{align}
\abs{v \circ F_m}^p_{k,p,\hat{T}} 
&= \sum_{\abs{\alpha} = k} \int_{\hat{T}} \abs{ \hat{D}^\alpha (v \circ F_m)}^p \diff\hat{T} \\
&= \sum_{\abs{\alpha} = k} \int_{\hat{T}} \abs{ [(D^\alpha v) \circ F_m)] B^{\abs{\alpha}}}^p \diff\hat{T} \\
&= \sum_{\abs{\alpha} = k} \int_{\hat{T}} \abs{ [(D^\alpha v) \circ F_m)] B^{\abs{\alpha}}}^p J^{-1} J \diff\hat{T} \\
&= \sum_{\abs{\alpha} = k} \int_{T} \abs{ D^\alpha v B^{\abs{\alpha}}}^p J^{-1} \diff T \\
&\le \norm{B}^{kp} J^{-1} \sum_{\abs{\alpha} = k} \int_{T} \abs{ D^\alpha v}^p \diff T,
\end{align}
where $J$ is the absolute value of the Jacobian of the transformation $F_m$ (in the present case, $J=\abs{\det B}$). Hence, using~\eqref{eqn:scaling_B}, we conclude:
\[
\abs{v \circ F_m}_{k,p,\hat{T}} \le \norm{B}^k J^{-\frac{1}{p}} \abs{v}_{k,p,T}
\le c h_m^k J^{-\frac{1}{p}} \abs{v}_{k,p,T},
\]
and similarly, using~\eqref{eqn:scaling_B_inv}:
\[
\abs{v}_{k,p,T} \le \norm{B^{-1}}^k J^{\frac{1}{p}} \abs{v \circ F_m}_{k,p,\hat{T}}
\le c \rho_m^{-k} J^{\frac{1}{p}} \abs{v \circ F_m}_{k,p,\hat{T}},
\]
where $c$ is a constant depending on the reference triangle $\hat{T}$, but not on the triangle $T_m$.

For $h_m \le 1$, we can write similar scaling arguments for the full norms:
\begin{align}
    \norm{v \circ F_m}_{k,p,\hat{T}} & \le c h_m^k J^{-\frac{1}{p}} \norm{v}_{k,p,T},\\
    \norm{v}_{k,p,T} & \le c \rho_m^{-k} J^{\frac{1}{p}} \norm{v \circ F_m}_{k,p,\hat{T}}.
\end{align}

From now on we adopt this notation:
\begin{itemize}
\item $a \lesssim b$ means "$\exists c \text{ s.t. } a \le cb$";
\item $a \gtrsim b$ means "$\exists c \text{ s.t. } ca \ge b$";
\item $a \sim b$ means "$a \lesssim b \wedge b \lesssim a$".
\end{itemize}
For example, if two norms $\norm{\cdot}_X$ and $\norm{\cdot}_Y$ are equivalent, the following statements share the same meaning:
\[
\norm{a}_X \sim \norm{a}_Y \iff \exists c,C \text{ s.t. } c\norm{a}_X \le \norm{a}_Y \le C \norm{a}_X.
\]


\section{Bramble-Hilbert lemma}

\begin{lemma}[Bramble-Hilbert]
Let $V$, $W$, $Q$ be Banach spaces and let $\tau \in \L(V,W)$ be an operator such that $Q \subset \ker(\tau)$. Then:
\begin{enumerate}
\item $\norm{\tau(u)}_W \le \norm{\tau}_* \inf_{q \in Q} \norm{u-q}_V \quad \forall u\in V$.
\item Choosing $V=W^{k+1,p}(\Omega)$, $W=W^{s,p}(\Omega)$, $Q=\P^k(\Omega)$, with $\Omega$ open connected (and bounded Lipschitz) and $0\le s \le k$, we get
\[
\norm{\tau(u)}_{s,p,\Omega} \lesssim \norm{\tau}_* \abs{u}_{k+1,p,\Omega} \quad \forall u\in W^{k+1,p}(\Omega).
\]
\end{enumerate}
\label{lemma:bramble-hilbert}
\end{lemma}

\begin{proof}
The first inequality follows immediately by observing that $\tau$ is linear and that $q\in \ker(\tau)$ implies $\tau(u+q) = \tau(u) + \tau(q) = \tau(u)$, hence:
\[
\norm{\tau(u)}_W = \norm{\tau(u+q)}_W = \le \norm{\tau}_* \norm{u+q}_V \qquad \forall q \in Q.
\]

The second inequality follow from by showing that~\cite{ciarlet78}
\begin{lemma}[Denis-Lions]
  \begin{equation}
    \abs{u}_{k+1,p,\Omega} \sim \inf_{q \in \P^k} \norm{u+q}_{k+1,p,\Omega} \qquad \forall u \in W^{k+1,p}(\Omega).
    \label{eqn:deny-lions}
  \end{equation}
\end{lemma}

From now on, we'll omit $\Omega$ for simplicity where its presence is obvious.
\begin{itemize}
\item The inequality $\abs{u}_{k+1,p} \lesssim \inf_{q \in \P^k} \norm{u+q}_{k+1,p}$ is trivial, because every polynomial of degree at most $k$ has zero derivatives of $(k+1)$-th order. Hence
\[
\abs{u}_{k+1,p} = \abs{u+q}_{k+1,p} \le \norm{u+q}_{k+1,p} \quad \forall q \in \P^k.
\]
\item In order to prove that the converse holds, let $\{v_i\}_{i=0}^N$ be a basis for $\P^k$. Then, with the usual notation, $(\P^k)'=\Span\{v^i\}_{i=0}^N$, where the $v^i$-s are such that $v^i(v_j)=\delta_{ij}$. Since $\P^k \subset W^{k+1,p}$, by the Hahn-Banach theorem we can naturally extend these $v^i$-s to be elements of $(W^{k+1,p})'$. Hence we can consider the projection
\begin{align}
\Pi^k:  W^{k+1,p}(\Omega) & \to \P^k(\Omega)\\
u & \mapsto \sum_{i=0}^N v^i(u) v_i.
\end{align}
The key part is proving that
\begin{equation}\label{eqn:deny-lions-exp}
\norm{u}_{k+1,p}^p \sim \abs{u}_{k+1,p}^p + \sum_{i=0}^N \abs{v^i(u)}^p \quad \forall u \in W^{k+1,p}.
\end{equation}
In fact, the $\gtrsim$ inequality follows from bounding the Sobolev seminorm with the corresponding norm and by the simple inequality
\[
\abs{v^i(u)} \le \norm{v^i}_{(W^{k+1,p})'} \norm{u}_{k+1,p}.
\]
The $\lesssim$ inequality, instead, can be proven by contradiction. If it were not true, then for every constant $c$ there would exist $w_c \in W^{k+1,p}$ such that $\norm{w_c}_{k+1,p} = 1$ and
\[
c\big(\abs{w_c}_{k+1,p}^p + \sum_{i=0}^N \abs{v^i(w_c)}^p\big) < \norm{w_c}_{k+1,p}^p = 1.
\]
If we choose $c=j$ in $[1,2,\dots)$, we obtain a sequence $\{w_j\}$ such that
\begin{equation} \label{eqn:deny-lions-proof}
\abs{w_j}_{k+1,p}^p + \sum_{i=0}^N \abs{v^i(w_j)}^p < \frac{1}{j} \quad \forall j.
\end{equation}
Now, thanks to the compact embedding $W^{k+1,p} \hookrightarrow W^{k,p}$, we have, up to a subsequence, that $w_j \to w$ strongly in $W^{k,p}$ for some $w \in W^{k,p}$.

However, we can easily show that the function $w$ belongs also to $W^{k+1,p}$. In fact, from~\eqref{eqn:deny-lions-proof} we deduce
\[
\abs{w_j}_{k+1,p}^p < \frac{1}{j} \quad \forall j.
\]
This implies that the sequence $\{w_j\}$ is Cauchy also in $W^{k+1,p}$, because
\[
\norm{w_i-w_j}_{k+1,p}^p = \norm{w_i-w_j}_{k,p}^p + \abs{w_i-w_j}_{k+1,p}^p.
\]
Hence, by uniqueness of the limit, $w \in W^{k+1,p}$. Then, by (strong) continuity of the norm, we obtain $\norm{w}_{k+1,p}=1$ and we can pass to the limit the inequality~\eqref{eqn:deny-lions-proof} to get:
\[
\abs{w}_{k+1,p}^p = 0, \quad \sum_{i=0}^N \abs{v^i(w)}^p = 0.
\]
The first equality, since $\Omega$ is an open connected domain, implies that $w \in \P^k(\Omega)$. The second equality implies that $\Pi^k(w)=0$. But for a polynomial in $\P^k$ we have $\Pi^k(w)=w$, hence $w=0$, which is in contradiction with $\norm{w}_{k+1,p}=1$.

We can now conclude, exploiting inequality~\eqref{eqn:deny-lions-exp}, by noticing that
\begin{align}
\inf_{q \in \P^k} \norm{u+q}_{k+1,p} &\le \norm{u- \Pi^k(u)}_{k+1,p} \\
& \lesssim \abs{u-\Pi^k(u)}_{k+1,p}^p + \sum_{i=0}^N \abs{v^i(u-\Pi^k(u))}^p \\
& = \abs{u}_{k+1,p}^p.
\end{align}
where the last equality holds because the summation term is equal to zero by linearity of the $v^i$-s and the seminorm is simplified due to $\Pi^k(u)$ being a polynomial in $\P^k$.
\end{itemize}
\end{proof}

Bramble-Hilbert lemma allows us to provide a-priori error estimates for the interpolation operator of a finite element space, by quantifying the constants in the inequality throught the scaling arguments presented in the previous section:
\begin{theorem}[Interpolation error]
  \label{theo:interpolation_error}
  Let $u\in W^{k+1,p}(\Omega)$, and let
\[
h = \max_{T \in \T_h} h_T, \quad \rho = \min_{T \in \T_h} \rho_T, \quad \Omega = \mathring{\overline{\bigcup_{T \in \T_h} T}}.
\]

Then the interpolation operator $\Pi^k$ satisfies:
\begin{equation}
  \label{eqn:interpolation_error}
\Bigl( \sum_{T \in \T_h} \norm{u- \Pi^k(u)}_{s,p,T}^p \Bigr)^\frac{1}{p} \lesssim
h^{\ell+1} \rho^{-s} \abs{u}_{\ell+1,p,\Omega} \qquad 0 \le s \le \ell \leq k,
\end{equation}
where $k$ indicates the degree of the local polynomial spaces $P^k(T)$ of the finite elements.
\end{theorem}

\begin{proof}
Consider an affine mapping $F_T: \hat{T} \to T$ and its related quantities $h_T$ and $\rho_T$. 

We can use Bramble-Hilbert lemma applied to the operator $(I-\Pi^k)\circ F_T$ to control the $s$-Sobolev norm of the error $u - \Pi^k(u)$. By repeating the scaling arguments on both $\norm{(u- \Pi^k(u)) \circ F_T}_{s,p,\hat{T}}$ and $\abs{u \circ F_T }_{k+1,p,\hat{T}}$, we get, for any triangle $T = F_T(\hat{T})$:
\begin{align}
\norm{u- \Pi^k(u)}_{s,p,T} &\lesssim \rho_T^{-s} J^{\frac{1}{p}} \norm{(u- \Pi^k(u)) \circ F_T}_{s,p,\hat{T}} \\
& = \rho_T^{-s} J^{\frac{1}{p}} \norm{u \circ F_T - \Pi^k(u) \circ F_T}_{s,p,\hat{T}} \\
& \lesssim \rho_T^{-s} J^{\frac{1}{p}} \abs{u \circ F_T }_{k+1,p,\hat{T}} \\
& \lesssim h_T^{k+1} \rho_T^{-s} \abs{u}_{k+1,p,T} \qquad 0\le s \le k.
\end{align}

Overall, we can control the $s$-Sobolev norm of the error up to order $k$ with the Sobolev semi-norm of order $k+1$ of the solution $u$. Since $P^\ell \subseteq P^k$ for $\ell \leq k$, we can repeat the argument for the Sobolev semi-norm of order $\ell+1$. Summing over the cells of the triangulation, we obtain a global \emph{a-priori} inequality:
\[
\sum_{T \in \T_h} \norm{u- \Pi^k(u)}_{s,p,T}^p \lesssim
\sum_{T \in \T_h} \bigl( h_T^{\ell+1} \rho_T^{-s} \abs{u}_{\ell+1,p,T} \bigr)^p, \qquad 0 \le s \le \ell \leq k.
\]
The thesis follows from the definitions of $h$ and $\rho$.
\end{proof}
The LHS is also called a \emph{broken Sobolev norm}, since it is defined element-wise. It is not automatically equal to $\norm{u- \Pi^k(u)}_{s,p,\Omega}$: in order to be, the global space $V_h$, obtained by gluing together the local spaces $P^k(T_m)$, must be \emph{conforming} to $V$. 

\begin{remark}[Conformity] For a finite element space to be conforming, we need to guarantee that the resulting finite dimensional space $V_h$ is indeed included in $V$. While locally every polynomial space $P^k(T)$ is always contained in every Sobolev space $W^{k,p}(T)$ (since $P^k(T) \subset C^{\infty}(T)$ for any $k$), the same cannot be said for the \emph{union} of the local spaces $P^k(T_m)$. Such a space is in general discontinuous (i.e., there is no reason to expect that the polynomials in $P^k(T_m)$ and $P^k(T_n)$ agree on the common face between the two cells $T_m$ and $T_n$), and although continuity is in general not required for Sobolev spaces $W^{k,p}(\Omega)$ (i.e., $W^{k,p}(\Omega)\not\subset C^0(\Omega)$ for $kp\leq d$), one can show that if $k \geq 1/p$, then a discontinuity along a surface of co-dimension one is not allowed.

If we want the finite dimensional space $V_h$ to be conforming with $W^{k,p}(\Omega)$ (for example), then we need to restrict the polynomial spaces on the elements $T_m$ and $T_n$ so that $P^k(T_m)$ and $P^k(T_n)$ agree on the common face between the two cells $T_m$ and $T_n$ for all derivatives up to the degree $k-1$.
\end{remark}

\begin{remark}
The same inequality can be proven if the triangulation $\T_h$ is made of quadrilaterals, and if the mapping is bi-linear instead of affine. In this case, the constants that appear in the error estimates are generally better, and this is why meshes made of quadrilaterals and hexahedra are usually preferred (when available): they are more difficult to construct, but they offer better approximation properties.
\end{remark}


\section{\texorpdfstring{$H^k$}{Hk} error estimates}

Finally, we can put together Ceà's lemma~\ref{lemma:cea} with Bramble-Hilbert lemma~\ref{lemma:bramble-hilbert} to obtain an \emph{a priori} error estimate for finite element methods of order $k$. To use the \emph{a-priori} interpolation estimate given in Theorem~\ref{theo:interpolation_error}, we need to make sure that the solution $u$ of the elliptic problem is indeed in the space $H^{k+1}(\Omega)$. This, in general, will depend on the operator $A$, on the source term $f$, and on the regularity of the domain $\Omega$, and it is what the theory of regularity for PDEs attempts to do. We will not go into details here, but we will just state the result we need.

\begin{definition}
  Let $V \subset H^s(\Omega)$, with $\|\cdot\|_V \sim \|\cdot\|_{s,\Omega}$ and $A \in \L(V,V')$. We say that $A$ is $r$-\emph{regular} if there exists an interval $[s_{min}, s_{max}]$ (with $s_{min}\geq -s$) such that $\forall g \in H^\ell(\Omega)$, with $\ell \in [s_{min}, s_{max}]$, the problem $Au=f$ admits a unique solution $u$ which satisfies
  \[
  \norm{u}_{\ell+r,\Omega} \lesssim \norm{f}_{\ell,\Omega}.
  \]
\end{definition}

With this definition, we can state the following theorem, which puts together all the previous results:
\begin{theorem}[A priori error estimate]
  \label{theo:a_priori_estimate}
Assume that $A$ is bounded and coercive in $H^s(\Omega)$, and that $A$ is $r$-regular for $f \in H^{\ell}(\Omega)$, $\ell \in [s_{min}, s_{max}]$. For a conforming approximation $V_h \subset V \equiv H^s(\Omega)$, constructed with piece-wise polynomial bases of order $k$, we have that the approximate solution $u_h$ satisfies the following \emph{a priori} error estimate:
\begin{equation}
  \label{eq:a_priori_error_estimates}
    \norm{u-u_h}_{s,\Omega} % \lesssim h^{\ell+r-s} \abs{u}_{\ell+r} 
    \lesssim h^{\ell+r-s} \norm{f}_{\ell,\Omega} \qquad  s_{min}+r \le \ell+r \le \min(k + 1,s_{max}+r).
\end{equation}
\end{theorem}
\begin{proof}
  From Ceà's lemma (Lemma~\ref{lemma:cea}) and the interpolation error estimate (Theorem~\ref{theo:interpolation_error}), we have that
  \begin{align} \label{ineq:cea_scaling}
    \norm{u - u_h}_{s,\Omega} &\le \frac{\norm{A}}{\alpha} \inf_{v_h \in V_h} \norm{u - v_h}_{s,\Omega} \\
  & \le \frac{\norm{A}}{\alpha} \norm{u - \Pi^k(u)}_{s,\Omega} \\
  & \lesssim \frac{\norm{A}}{\alpha} \rho^{-s} h^{k+1} \abs{u}_{k+1,\Omega}.
  \end{align}
  
  The thesis follows by the $r$-regularity assumption of the operator $A$.
\end{proof}
The inequality hints at some strategies for error reduction (independently on the solution or on $f$):
\begin{itemize}
\item take $h$ smaller, i.e. \emph{refine the mesh};
\item take $k$ bigger (if possible), which implies increasing the polynomial degree of the finite element space up to the maximum regularity allowed by exact solution $u$;
\end{itemize}

%%%%%
% Lecture of 27 of March
%%%%%


Consider a triangulation $\T_h$ and recall the quantities $h$ and $\rho$ we defined in the previous section:
\[
h := \max_{T \in \T_h} h_T, \quad \rho := \min_{T \in \T_h} \rho_T.
\]
We say that the mesh is \emph{shape-regular} if
\[
\exists \sigma > 0 \text{ s.t. } \rho_T \ge \sigma h_T  \quad \forall T\in \T_h
\]
and we call $\sigma$ the constant of shape-regularity. We say instead that a mesh is \emph{quasi-uniform} if 
\[
\rho \sim h
\]
For such a mesh, inequality~\eqref{ineq:cea_scaling} becomes:
\begin{equation} \label{ineq:cea_scaling2}
\norm{u - u_h}_{s,p,\Omega}
\lesssim \frac{\norm{A}}{\alpha} h^{k+1-s} \abs{u}_{k+1,p,\Omega}.
\end{equation}

\begin{example}
The Poisson problem on a Lipschitz, convex domain is $2$-regular (i.e. $r = 2$), with $s_{min} = -1$ and $s_{max}=0$. If $u \in H^1_0(\Omega)$ (i.e. $s = 1$)  inequality~\eqref{eq:a_priori_error_estimates} becomes
\[
\norm{u-u_h}_1 \lesssim h^{\ell+1} \norm{f}_\ell \quad \forall -1 \le \ell \le 0.
\]

If $\Omega$ is better than Lipschitz, then the range for $s$ becomes better, hence we gain a better convergence of the numerical approximation (provided that $f$ is more regular).
\end{example}

In order to show convergence of the numerical approximation in a weaker norm, we need to establish a relationship between the norm of the error involved and utilize the properties of the finite element spaces.

\subsection{Nitsche's trick}
With some bit of additional work, we can produce error estimates on weaker norms, such as the $L^2$ one, from the estimates on stronger norms.
\begin{lemma}[Nitsche] \label{lemma:nitsche}
Let $A \in \L(V,V')$ an operator as in the hypotheses of Lax-Milgram's lemma. Assume that $V$ is a subspace of some Hilbert space $H$, with the following properties:
\begin{romanlist}
\item $\norm{u}_H \lesssim \norm{u}_V$ $\quad \forall u \in V$;
\item $V = \overline{H}^{\norm{\cdot}_V}$,
i.e. the embedding $V \hookrightarrow H$ is dense and continuous;
\item we identify $H$ with its dual $H'$ via the Riesz representation theorem.
\end{romanlist}
Then the following inequality holds:
\begin{equation}\label{eqn:nitsche_lemma}
\norm{u-u_h}_H \lesssim \sup_{g \in H} \Bigl( \frac{1}{\norm{g}_H} \inf_{\phi_h \in V_h} \norm{\phi_g - \phi_h}_V \Bigr) \norm{u-u_h}_V .
\end{equation}
\end{lemma}

\begin{proof}

With the hypotheses above, if $f \in H \equiv H'$, by the Hahn-Banach theorem $f$ can be extended to a functional $\tilde{f} \in V'$, i.e.
\[
\langle \tilde{f}, v \rangle_V = (f, v)_H \quad \forall v \in V.
\]
Now consider the usual problem in operator form $A u = \tilde{f}$, i.e.
\begin{equation}
\langle Au,v \rangle = \langle \tilde{f},v \rangle = (f, v)_H \quad \forall v\in V.
\end{equation}
By Lax-Milgram lemma, we know that this problem has a unique solution $u \in V$.

Let $g \in H$ and consider the \emph{dual problem} $A^T \phi_g = g$, i.e.
\begin{equation}
\langle Av,\phi_g \rangle = \langle A^T\phi_g, v \rangle = \langle \tilde{g},v \rangle = (g, v)_H \quad \forall v\in V.
\end{equation}
Since also $A^T$ satisfies the hypotheses of  Lax-Milgram lemma, the adjoint problem admits itself a unique solution $\phi_g \in V$.

We start observing that
\begin{equation}\label{eqn:nitsche_norm}
\norm{u-u_h}_H = \sup_{g \in H} \frac{(g, u-u_h)_H}{\norm{g}_H}.
\end{equation}
This is true because $H$ is a Hilbert space, hence $H \cong H'$.

Given $g \in H$, we know from the last remark that the dual problem $A^T \phi = g$ admits a unique solution $\phi_g \in V$. If we set $v = u-u_h$, we get:
\[
\langle A (u-u_h),\phi_g \rangle = (g, u-u_h)_H.
\]
Now we take advantage of the Galerkin orthogonality, i.e. $\langle A (u-u_h), \phi_h \rangle = 0$ $\forall \phi_h \in V_h$:
\[
(g, u-u_h)_H = \langle A (u-u_h),\phi_g - \phi_h \rangle \quad \forall \phi_h \in V_h.
\]
By the continuity of $A$ on~\eqref{eqn:nitsche_norm}, we can produce the desired estimate:
\begin{align}
\norm{u-u_h}_H &= \sup_{g \in H} \frac{(g, u-u_h)_H}{\norm{g}_H} \\
& = \sup_{g \in H} \frac{\langle A (u-u_h),\phi_g - \phi_h \rangle}{\norm{g}_H} \\
& \le \norm{A} \sup_{g \in H} \Bigl( \frac{1}{\norm{g}_H} \inf_{\phi_h \in V_h} \norm{\phi_g - \phi_h}_V \Bigr)
\norm{u-u_h}_V .
\end{align}
\end{proof}

In practice, we want to use this lemma with $V \subset H^s$ and $H \subset H^\ell$, with $0\le \ell < s$, and exploit the $r$-regularity property of the operator to gain better estimates in the $H$ norm, by showing that the term $\norm{\phi_g - \phi_h}_V$ is itself proportional to a power of $h$ and to the $H$ norm of $g$. 

\begin{theorem}[Nitsche's trick]
  \label{theo:nitsche_trick}
  Let $A \in \L(V,V')$ be a $r$-regular, bounded, and coercive operator. Let $V\equiv H^s(\Omega)$, $H \equiv H^\ell(\Omega)$ satisfy the same hypotheses of Lemma~\ref{lemma:nitsche}. Moreover, the following holds:
  \begin{romanlist}
  \item $V_h \subset V$ is a finite element space of polynomials of degree $k$;
  \item $0\leq \ell < s$;
  \item $s+r\leq k+1$;
  \end{romanlist}

  Then, for any $f\in H^{m}(\Omega)$, with $m\in[s_{min}, s_{max}]$, the following estimates hold:
  \begin{equation}
    \label{eqn:nitsche_trick}
    \begin{aligned}
      \norm{u-u_h}_s &\lesssim h^{m+r-s} \norm{f}_m,\\
      \norm{u-u_h}_\ell &\lesssim h^{\ell+r-s}  \norm{u-u_h}_s \lesssim h^{(\ell+m+2(r-s))} \norm{f}_\ell.
    \end{aligned}
  \end{equation}
\end{theorem}

\begin{proof}
  The first part of the estimate comes from
  Theorem~\ref{theo:a_priori_estimate}. The second part of the estimate comes
  from Lemma~\ref{lemma:nitsche} and from the $r$-regularity of the operator
  $A^T$. We refine the estimate of $\norm{\phi_g - \phi_h}_s$ by
  Theorem~\ref{theo:interpolation_error} applied to the function $\phi_g$. 
  Provided that $0\leq s \leq \ell+r$, and that $\ell+r \leq k+1$, we have:
\begin{align}
\inf_{\phi_h \in V_h} \norm{\phi_g - \phi_h}_s &\lesssim \norm{\phi_g - \Pi^k\phi_g}_{s} \lesssim h^{\ell+r-s} \abs{\phi_g}_{\ell+r}  \\
&\lesssim h^{\ell+r-s} \norm{g}_\ell.
\end{align}
Inserting this estimate in Lemma~\ref{lemma:nitsche}, we obtain the thesis.
\end{proof}

\begin{example}
For a Poisson problem on a Lipschitz, convex domain, we have that $r=2$ in the range $[-1,0]$. With this we can estimate also the $L^2$ norm of the error, i.e., we pick $s = 0$ (and we have $k=1$) and we obtain:
\[
\norm{u-u_h}_0 \lesssim h \norm{u-u_h}_1.
\]
By the $2$-regularity, we have
\[
\norm{u-u_h}_1 \lesssim h^{t+1} \norm{f}_t \qquad \forall t \in [-1, 0].
\]
Hence:
\[
\norm{u-u_h}_0 \lesssim h^{t+2} \norm{f}_t.
\]
For instance, if $t=0$, we have convergence of order $1$ in $H^1$, of order $2$ in $L^2$.
\end{example}


\section{Inverse estimates}
Suppose that $V_h|_T \equiv P^k(T)$ and let $v_h \in V_h$. In the previous sections we have proved that, for quasi-uniform meshes, on each triangle $T$ the following inequality holds:
\[
\abs{v_h}_{s,p,T} \lesssim h^{k-s} \abs{v_h}_{k,p,T} \quad \forall 0 \le s \le k.
\]
With the same scaling argument, we can also prove that, if $h<1$, then
\[
\abs{v_h}_{s,p,T} \lesssim h^{\ell-s} \abs{v_h}_{\ell,p,T} \quad \forall 0 \le \ell \le s \le k.
\]
Of course, in either case, $k$, $s$ must be chosen smaller or equal to the degree of $v_h$. The proof relies once more on the fact that all norms in finite dimensional spaces are equivalent, and that the constants in the equivalence depend on the shape of the element and on the size of the element. By moving to a reference element, and exploiting the scaling properties of the norms, we quantify the constants in the equivalence in terms of mesh size $h$ only when moving to (or from) the reference element. There, we can freely exchange the norms, exploiting equivalence of all norms in finite dimensional spaces, on a domain of size $O(1)$, with constants that do not depend on $h$. By going back to the original element with the new norm, we again use the scaling properties to obtain the desired result. 

When moving an $s$ norm to the reference element we pay a factor $h^{-s}$, while when moving an $\ell$ norm from the reference element to the current element we pay a factor $h^{\ell}$. Overall we control $s$ norms with $\ell$ norms with a factor $h^{\ell-s}$, and vice versa.

\section{Trace operators}\label{sec:trace_operators}
In order to deal with the trace of a Sobolev function, it is necessary to give a meaning to the space $W^{s,p}(\Omega)$ for $s\in \R$. Here we will only present the fundamental concepts. For extensive reference, see~\cite{fract_sob}. For a Lipschitz domain $\Omega \subset \R^d$ we can also define the space $W^{s,p}(\Omega)$ in a different way. We let $\lambda \in (0,1)$ such that $s = m+\lambda$ for an integer $m$ and consider the following norm:
\[
\norm{u}_{s,p}^p := \norm{u}_{m,p}^p + \sum_{\abs{\alpha}=m} \abs{D^\alpha u}^p_{\lambda,p}
\]
where $\abs{u}_{\lambda,p}$ is called in literature the \emph{Gagliardo seminorm} of $u$:
\[
\abs{u}^p_{\lambda,p} := \int_{\Omega} \int_{\Omega} \frac{\abs{u(x)-u(y)}^p}{\abs{x-y}^{d+\lambda p}} \diff x \,dy.
\]
Then we define
\[
W^{s,p}(\Omega) := \Set{u \in W^{m,p}(\Omega) : \norm{u}_{s,p}^p < +\infty}.
\]
As in the case $s \in \N$, we also have that $W^{s,p}_0(\Omega) = \overline{C^\infty_0(\Omega)}^{\norm{\cdot}_{s,p}}$. One can prove that the spaces we have constructed are Banach spaces. Moreover, if we restrict to the case $p=2$, they turn out to be Hilbert spaces.

For the case $p=2$, if $\Omega = \R^d$, there is an equivalent definition of the space $H^s(\R^d)$ via the Fourier transform $\F$:
\[
\bigl(\F u\bigr)(\xi) := \Bigl(\frac{1}{2\pi} \Bigr)^{d/2} \int_{\R^d} e^{-i \xi \cdot x} u(x) \diff x.
\]
If we consider the Schwartz space $\mathcal{S}$ of rapidly decaying $C^\infty$ functions in $\R^d$, then $H^s(\R^d)$ can be defined as a subspace of tempered distributions:
\[
H^s(\R^d) := \Set{u \in \mathcal{S}' : \norm{u}_s < +\infty}
\]
where the norm is
\[
\norm{u}_s := \norm{\abs{(1+\abs{\cdot}^2)}^{\frac{s}{2}} \F u(\cdot)}_{L^2(\R^d)}.
\]
In particular, as before we have $H^s_0(\R^d) = \overline{C^\infty_0(\R^d)}^{\norm{\cdot}_s}$.
Without going into detail, the main reason of the equivalence ot the two formulation is Plancherel's formula for the Fourier transform, which is specific for $p=2$. There is no equivalence for different values of $p$.

Now let $\Omega \subset \R^d$ be a Lipschitz domain and $\Gamma$ its boundary. If $u \in H^s(\Omega) \not\subset C^0(\Omega)$, then $\restr{u}{\Gamma}$ makes no sense pointwise. For example, $H^1$ functions are not continuous for $d \ge 2$. In order to give some sense to the expression $\restr{u}{\Gamma}$, we introduce the \emph{trace operator}:
\begin{theorem}[Trace theorem]
Let $\Omega \subset \R^d$ be a Lipschitz domain and $s \in (\frac{1}{2}, 1]$. Then there exists a unique linear bounded mapping
\[
\gamma: H^s(\Omega) \to H^{s-\frac{1}{2}}(\Gamma)
\]
such that:
\begin{enumerate}
\item For every $u \in C^0(\overline{\Omega})$, $\gamma u = \restr{u}{\Gamma}$.
\item For every $v \in H^s(\Omega)$, $\norm{\gamma v}_{s-\frac{1}{2},\Gamma} \lesssim \norm{v}_{s,\Omega}$.
\item $\gamma$ admits a bounded right inverse
\[
E: H^{s-\frac{1}{2}}(\Gamma) \to H^s(\Omega)
\]
i.e. $\forall g\in H^{s-\frac{1}{2}}(\Gamma)$ we have $\norm{E g}_{s,\Omega} \lesssim \norm{g}_{s-\frac{1}{2},\Gamma}$ and $\gamma E g = g$.
\end{enumerate}
In particular, $\ker(\gamma) = H^s_0(\Omega) = \Set{u \in H^s(\Omega) : \gamma u = 0}$.
\end{theorem}

The trace is a fundamental concept when dealing with boundary conditions. Consider as a motivating example the Poisson problem on $\Omega$ with both Dirichlet and Neumann boundary conditions:
\[
\begin{cases} 
-\Delta u = f \qquad &\text{in $\Omega$} \\
u =  g_D \qquad &\text{on $\Gamma_D$} \\
\frac{\partial u}{\partial n} =  g_N \qquad &\text{on $\Gamma_N$}
\end{cases}
\]
where $\Gamma_D$ and $\Gamma_N$ are contained in $\Gamma$. If $u \in H^1$, then its restriction to $\Gamma$ is not well defined. In order to handle Dirichlet boundary conditions, we can resort to the trace operator (with restricted codomain) $\gamma_{_{\Gamma_D}}: H^1(\Omega) \to H^{\frac{1}{2}}(\Gamma_D)$ and require that $\gamma_{_{\Gamma_D}} u = g_D$. However, this trick fails with $\frac{\partial u}{\partial n} \in L^2$, because even its trace makes no sense! We solve this problem incorporating the Neumann boundary condition into the weak form.
To make things clearer: we let
\begin{align}
V_{0,\Gamma_D} &:= \Set{u \in H^1(\Omega) : \gamma_{_{\Gamma_D}}=0}, \\
V_{g_D,\Gamma_D} &:= V_{0,\Gamma_D} + u_D, \quad \text{where } \gamma_{_{\Gamma_D}} u_D = g_D.
\end{align}
The new weak form is: find $u \in V_{g_D,\Gamma_D}$ such that
\[
(\nabla u, \nabla v) = \langle f, v \rangle + \int_{\Gamma_N} g_N v \quad \forall v \in V_{0,\Gamma_D}.
\]
In particular, the minimal regularity we need to require for $f$, $g_D$, $g_N$ is:
\[
f \in \bigl(H^1(\Omega)\bigr)', \quad g_D \in H^\frac{1}{2}(\Gamma_D), \quad g_N \in \bigl(H^\frac{1}{2}(\Gamma_N)\bigr)'.
\]

% !TeX source = ../main.tex

\chapter[A posteriori error estimates]{A posteriori error estimates}

\section{Interpolation of non-smooth functions}

This section is concerned with the problem of interpolating non-smooth functions, e.g. functions that are too rough to be in the domain of the Lagrange interpolation operator. This situation occurs, for instance, when interpolating discontinuous functions, e.g. in $L^2(\Omega)$ or $H^1(\Omega)$ in dimension $d \ge 2$.
Throughout this section, $\Omega$ is a polyhedron and $\{\T_h\}_{h>0}$ is a shape-regular family of affine, simplicial, geometrically conformal meshes.

\subsection{Scott-Zhang interpolation}

Let $V_h$ be the space of piecewise polynomials of order $k$ on every $T \in \T_h$. This is the usual $H^1$-conformal approximation space based on the $\P^k$ Lagrange finite element, so $V_h \subset H^1$.
Our goal is to define an interpolation operator $\SZ: H^1(\Omega) \to V_h$ such that:
\begin{enumerate}
    \item $\SZ v \in H^1_0(\Omega)$ for every $v \in H^1_0(\Omega)$;
    \item $\SZ u_h = u_h$ for every $u_h \in V_h$.
\end{enumerate}
This will act as a regularization operator based on \emph{macroelements} consisting of element patches. In particular, the first condition means that $\SZ$ will preserve homogeneous boundary conditions.

\begin{figure}[!ht]
    \centering
    \includegraphics{figures/scott_zhang.pdf}
    \caption{Example choice of the sets $\Xi$ for different types of support points in the Scott-Zhang interpolator. Case $a_i$: support point on the interior of a triangle -- $\Xi_i$ is the triangle itself. Case $a_j$: support point on an internal vertex of the triangulation -- $\Xi_i$ is one of the edges that insist on that vertex. Case $a_k$: support point on the interior of an edge within $\Omega$ -- $\Xi_k$ is the edge itself. Case $a_\ell$: support point on the boundary of the triangulation -- $\Xi_\ell$ is one of the edges that contain $a_\ell$. When multiple choices are possible, we can pick any one (shown as partial red lines in the figure).}
    \label{fig:scott_zhang}
\end{figure}

How do we construct it? Recall that $V_h = \Span\{v_i\}_{i=1}^N$ for some $N$. Let $\{a_1, \dots , a_N\}$ be the Lagrange nodes (support points).
Given a node $a_i$, we associate either a $d$-simplex or a $(d-1)$-simplex $\Xi_i$ to it, as follows (see Figure~\ref{fig:scott_zhang}):
\begin{itemize}
    \item If $a_i \in \mathring{\overline{T}}$ for some $T \in \T_h$, i.e. $a_i$ is in the interior of an element, then $\Xi_i := T$ itself.
    \item If $a_i \in \partial T \setminus \partial \Omega$, i.e. $a_i$ is on some face in the interior of $\Omega$, we choose a face $F$ containing $a_i$ and set $\Xi_i := F$.
    \item If $a_i \in \partial T \cap \partial \Omega$, i.e. $a_i$ is on some face at the boundary, we choose a face $F$ \emph{at the boundary} containing $a_i$ and set $\Xi_i := F$.
\end{itemize}
In the first case, $\Xi_i$ is a $d$-simplex, otherwise it is a $(d-1)$-simplex. Notice that the choice of the face may not be unique, but this will not be relevant. It is only important to choose faces at the boundary for nodes at the boundary, in order for $\SZ$ to preserve homogeneous boundary conditions.

Next, let $n_i$ be the number of nodes belonging to $\Xi_i$ and let $\{v_{i,q}\}_{q=1}^{n_i}$ the restrictions to $\Xi_i$ of the basis functions associated to those support points. Conventionally, we order those functions to have $v_i = v_{i,1}$. Then, for every $q=1,\dots,n_i$, consider the function $\gamma_{i,q} \in \Span\{v_{i,q}\}$ such that
\begin{equation}\label{eqn:sz_def}
    \int_{\Xi_i} \gamma_{i,q} v_{i,r} = \delta_{qr} \quad \forall r=1,\dots,n_i.
\end{equation}
Such a function is unique because of the properties of the $v_{i,r}$ (we have to solve a non-singular linear system). Finally, we define:
\[
\SZ u := \sum_{i=1}^N v_i \int_{\Xi_i} \gamma_{i,1} u = 
\sum_{i=1}^N S^i(u) v_i,
\]
where each linear functional $S^i$ is defined as
\[
S^i(u) := \int_{\Xi_i} \gamma_{i,1} u.
\]
Let us verify that $\SZ$ has indeed the required properties:
\begin{itemize}
    \item If $\gamma$ is the trace operator and $u \in H^1_0(\Omega)$, then $\gamma \SZ u = 0$ by construction.
    \item By the linearity of the $S^i$-s, $\SZ$ is a linear operator. Then, to check if $\SZ u_h = u_h$ for every $u_h \in V_h$, it is sufficient to prove it on a basis of $V_h$. Indeed,
    \[
        \SZ v_j = \sum_{i=1}^N v_i \int_{\Xi_i} \gamma_{i,1} v_j = \sum_{i=1}^N v_i \delta_{ij} = v_j.
    \]
    In particular, the equality
    \[
    \int_{\Xi_i} \gamma_{i,1} v_j = \delta_{ij}
    \]
    follows from \eqref{eqn:sz_def} because, for $j \ne i$, $v_j$ will be either outside $\Span\{v_{i,q}\}$ or equal to $v_{i,r}$ for $r \ne 1$.
\end{itemize}

\begin{remark}
    As we know, the global shape functions $v_i$-s satisfy
    \[
    v^i(v_j) = \delta_{ij}
    \]
    where the $v^i$-s are the global degrees of freedom. We have just seen that the same property holds for the $S^i$-s. Therefore we can see $\SZ$ as a modified Lagrange-type interpolation operator. Remember that the usual Lagrange interpolation is defined by
    \[
    \Pi u = \sum_{i=1}^N v^i(u) v_i.
    \]
\end{remark}

\begin{figure}
    \centering
    \includegraphics{figures/scott_zhang_patches.pdf}
    \caption{The approximation properties of the Scott-Zhang interpolation operator $\SZ$ are defined in terms of the local patches $\Delta T$ and $\Delta F$. The patch $\Delta T$ is the union of all elements that share at least one vertex with $T$, while the patch $\Delta F$ is the union of all elements that share at least one vertex with $F$.}
    \label{fig:scott_zhang_patches}
\end{figure}

The interpolation properties of $\SZ$ are stated in the following lemma.
\begin{lemma}[Scott-Zhang]
    Let $1\le p < +\infty$ and choose $\ell\ge1$ if $p=1$ or $\ell>\frac{1}{p}$ otherwise. Then the operator $\SZ$ is a projection from $W^{\ell,p}(\Omega)$ to $\P^{k,0}(\Omega)$, with the following properties:
    \begin{enumerate}
        \item Stability: for all $0 \le m \le \min(1,\ell)$,
        \[
        \forall h, \, \forall v \in W^{\ell,p}(\Omega) \quad 
        \norm{\SZ v}_{m,p,\Omega} \lesssim \norm{v}_{\ell,p,\Omega}.
        \]
        In particular, this is true for $m=\ell$.
        \item Approximation on elements: given $0\le m\le \ell\le k+1$,
        \[
        \forall h, \, \forall T \in \T_h, \, \forall v \in W^{\ell,p}(\Delta T) \quad 
        \norm{v- \SZ v}_{m,p,T} \lesssim h_T^{\ell-m} \abs{v}_{\ell,p,\Delta T}.
        \]
        \item Approximation on interfaces: given $m+\frac{1}{p}\le \ell\le k+1$,
        \[
        \forall h, \, \forall F \in \E^i, \, \forall v \in W^{\ell,p}(\Delta F) \quad 
        \norm{v- \SZ v}_{m,p,F} \lesssim h_F^{\ell-m-\frac{1}{p}} \abs{v}_{\ell,p,\Delta F}.
        \]
    \end{enumerate}
    Here $\Delta T$ is the set of elements in $\T_h$ that share at least one vertex with $T$. Analogously, if $F$ is an interface between two elements of $\T_h$, $\Delta F$ is the set of elements in $\T_h$ which share at least one vertex with $F$ (see Figure~\ref{fig:scott_zhang_patches}).
\end{lemma}

\subsection{Orthogonal projections}

Take $V_h = \P^{k,0}(\Omega)$ as in the previous subsection. Since $V_h \subset H^1(\Omega)$, we can consider the following orthogonal projection operators:
\[
    \Pi^0: L^2(\Omega) \to V_h, \qquad
    \Pi^1: H^1(\Omega) \to V_h,
\]
which are, respectively, the projections induced by the scalar products:
\begin{align}
    (u,v)_{0,\Omega} &= \int_\Omega uv, \\
    (u,v)_{1,\Omega} &= \int_\Omega uv + \int_\Omega \nabla u \nabla v.
\end{align}
In practice, $\Pi^0 u$ (respectively $\Pi^1 u$) gives us the closest function to $u$ with respect to the $L^2$ (resp. $H^1$) norm. Moreover,
\begin{align}
    (\Pi^0 u, v)_0 &= (u, v)_0 \quad \forall v \in V_h, \\
    (\Pi^1 u, v)_1 &= (u, v)_1 \quad \forall v \in V_h.
\end{align}
We refer to $\Pi^0$ as the \emph{$L^2$ projection}, to $\Pi^1$ as the \emph{Riesz projection} or \emph{elliptic projection}.

\textbf{------------start of probably wrong part\\}
We look for a way to express these with respect to the basis functions $\{v_i\}$. For simplicity, from now on we will use Einstein's notation. If $u \in L^2(\Omega)$, then $\Pi^0 u$ can be seen as a linear combination of the $v_i$-s:
\[
\Pi^0 u = u^i v_i
\]
where $u^j = v^j(u)$.
\rev{I think this is not true. This would mean that the Lagrange interpolation is an orthogonal projection, which is false in general.}
Set $U^0_i = (u,v_i)_0$. If $M$ is the matrix such that $M_{ij} = (v_j, v_i)_0$, then by linearity
\[
U^0_i = M_{ij} u^j
\]
Now take the inverse matrix of $M$: with this notation, we will call it $M^{ij}$ to stress the fact that
\[
M^{ij} M_{jk} = \delta^i_k.
\]
Then,
\[
\Pi^0 u = M^{ik} (u,v_k)_0 v_i.
\]
\textbf{-------------end of probably wrong part}

\begin{lemma}[stability of orthogonal projections]
    Let $k\ge1$. The following estimates hold:
    \begin{align}
        \forall v \in L^2(\Omega) \quad \norm{\Pi^0 v}_{0,\Omega} &\le \norm{v}_{0,\Omega}, \\
        \forall v \in H^1(\Omega) \quad \norm{\Pi^1 v}_{1,\Omega} &\le \norm{v}_{1,\Omega}.
    \end{align}
    Moreover, if the family of meshes $\{\T_h\}_{h>0}$ is quasi-uniform, then
    \[
        \forall v \in H^1(\Omega) \quad \norm{\Pi^0 v}_{1,\Omega} \lesssim \norm{v}_{1,\Omega}
    \]
    and the constant in the inequality is independent of $h$.
\end{lemma}

\begin{lemma}[approximation of smooth functions]
    Let $k\ge1$ and $1\le l \le k$. Then, for all $v \in H^{l+1}(\Omega)$ we have:
    \begin{align}
        \norm{v - \Pi^0 v}_{0,\Omega} &\lesssim h^{l+1} \abs{v}_{l+1,\Omega}, \\
        \norm{v - \Pi^1 v}_{1,\Omega} &\lesssim h^l \abs{v}_{l+1,\Omega},
    \end{align}
    and the constant is independent of $h$. Moreover, if the family of meshes $\{\T_h\}_{h>0}$ is quasi-uniform, then for all $v \in H^{l+1}(\Omega)$ we have:
    \[
        \norm{v - \Pi^0 v}_{1,\Omega} \lesssim h^l \abs{v}_{l+1,\Omega},
    \]
    again with the constant independent of $h$.
\end{lemma}


\section{A posteriori error analysis}

Turn back to our model Poisson problem:
\[
\begin{cases}
    -\Delta u = f \quad &\text{in } \Omega \\
    u = 0 \quad &\text{on } \partial\Omega
\end{cases}
\]
where the equality $-\Delta u = f$ is taken in $H^{-1}=(H^1_0)'$. So far, we controlled the approximation error \emph{a priori}, i.e. by exploiting theoretical properties of the PDE, like coercivity, to get error estimates. Let us try now a different approach and look for estimates that involve only known quantities, such as the size of the mesh cells, the problem data and the approximate solution $u_h$.

For every element $T$ in the triangulation, we would like to build an estimator $\eta_T = \eta_T(u_h, f)$ such that
\begin{equation}\label{eqn:apost_est1}
    \eta_T^2 \lesssim \abs{u - u_h}_{1,T}^2 \lesssim \eta_T^2,
\end{equation}
locally, and
\begin{equation}\label{eqn:apost_est2}
\sum_{T \in \T_h} \eta_T^2 \lesssim \abs{u - u_h}_{1,\Omega}^2 \lesssim \sum_{T \in \T_h} \eta_T^2,
\end{equation}
globally. In particular, the lower bound of the inequality is referred as \emph{optimality}, the upper one is the \emph{reliability}.
Should such an estimator exist, we could \emph{adapt} the mesh $\T_h$ in order to satisfy two requirements:
\begin{itemize}
    \item Given a tolerance $\texttt{tol}$,
    \[
    \Bigl( \sum_{T \in \T_h} \eta_T^2 \Bigr)^\frac{1}{2} \le \texttt{tol}.
    \]
    \item The error estimator $\eta_T$ is balanced over all the mesh, i.e.
    \[
    \frac{\texttt{tol}}{M} \lesssim \eta_T \lesssim \frac{\texttt{tol}}{M}
    \]
    where $M$ is the number of elements in $\T_h$.
\end{itemize}
Unfortunately, the estimates \eqref{eqn:apost_est1} and \eqref{eqn:apost_est2} are false in general. However, we can prove some weaker -- but still very useful -- results.

In general, we can prove that
\[
\norm{u - u_h}_{1,\Omega}^2 \lesssim \sum_{T \in \T_h} \eta_T^2,
\]
hence global reliability holds. Instead, the lower bound takes an "almost local" form:
\[
\eta_T \lesssim \norm{u-u_h}_{1,\Delta T} + \Pi(h_T,\Delta T,f)
\]
where $\Delta T$ is a patch of elements around $T$ and $\Pi(h_T,\Delta T,f)$ is a perturbation that is either negligible or is asymptotically of the same order as the
error $\norm{u-u_h}_{1,\Delta T}$.

For the Poisson problem, this is the estimator we will be interested in:
\begin{equation}\label{eqn:estimator}
    \eta_T := h_T \norm{f + \Delta u_h}_{0,T}
    + \sum_{F \subset \partial T \setminus \partial\Omega} \frac{1}{2} h_F^{\frac{1}{2}} \norm{\jump{\nabla u_h}}_{0,F}.
\end{equation}
\rev{it seems that the jump function used here is not exactly the same as the jump fn used in discontinuous Galerkin methods, at least if we want to sum also over faces at the boundary. Check this better!}


\subsection{Global reliability}

Assume that the mesh $\T_h$ is shape regular. We have shown in the previous chapters that, if $u$ is regular enough and $A$ is coercive, then
\[
\norm{u-u_h}_{1,\Omega} \lesssim \frac{\norm{A}}{\alpha} \inf_{v_h \in V_h} \norm{u-v_h}
\lesssim \Bigl( \sum_{T \in \T_h} h_T^{2k} \abs{u}_{k+1,T}^2 \Bigr)^{\frac{1}{2}}.
\]
This is an a priori error estimate. In the case $Au = -\Delta u$, the simplest a posteriori estimate we can get is the following:
\begin{equation}\label{eqn:apost_rough}
    \alpha \norm{u-u_h}_{1,\Omega} \lesssim \norm{A(u-u_h)}_{-1,\Omega} = \norm{f+\Delta u_h}_{-1,\Omega}.
\end{equation}
The inequality relies again on coercivity but, leaving room to generalizations, can be shown also if $A$ only satisfies the so called \emph{inf-sup condition}, i.e. the BNB theorem. We will talk about this theorem in chapter~\ref{chap:saddle}.

Inequality~\eqref{eqn:apost_rough} has an issue: it cannot be localized, because the norm in $H^{-1}(\Omega)$ is not local. However, we can work around it following these steps:
\begin{enumerate}
    \item exploit the definition of the $H^{-1}$ norm and the orthogonality of the error;
    \item integrate by parts;
    \item use the properties of Scott-Zhang interpolation.
\end{enumerate}
\boxed{Step 1} For every $v_h \in V_h$ we have
\begin{align}
    \norm{u-u_h}_{1,\Omega} &\lesssim \frac{1}{\alpha} \norm{A(u-u_h)}_{-1,\Omega} \\
    &= \frac{1}{\alpha} \sup_{v \in H^1_0(\Omega)} \frac{\abs{\langle A(u-u_h),v \rangle}}{\norm{v}_{1,\Omega}} \\
    &= \frac{1}{\alpha} \sup_{v \in H^1_0(\Omega)} \frac{\abs{\langle A(u-u_h),v-v_h \rangle}}{\norm{v}_{1,\Omega}}.
\end{align}
\boxed{Step 2} Split the contributions from $u$ and $u_h$ to the integral and remember that $u\in H^1_0(\Omega)$ and $-\Delta u = f$:
\begin{align}
    \langle A(u-u_h),v-v_h \rangle &= \int_\Omega \nabla u \nabla(v-v_h) - \int_\Omega \nabla u_h \nabla(v-v_h) \\
    & = \int_\Omega -\Delta u (v-v_h) - \int_\Omega \nabla u_h \nabla(v-v_h) \\
    & = \sum_{T \in \T_h} \Bigl(\int_T f (v-v_h) - \int_T \nabla u_h \nabla(v-v_h) \Bigr) \\
    & = \sum_{T \in \T_h} \Bigl(\int_T (f + \Delta u_h) (v-v_h) - \int_{\partial T} n \cdot \nabla u_h (v-v_h) \Bigr).
\end{align}
In particular, let us focus on the boundary terms: $v-v_h$ vanishes on $\partial\Omega$, hence the integrals on the boundary involve only interior faces. Globally, each interior face is counted twice, once per side, hence for each cell a jump term shows up with a factor $\frac{1}{2}$. Finally, we take advantage of Cauchy-Schwarz inequality:
\begin{align}
    \langle A(u-u_h),v-v_h \rangle &=
    \sum_{T \in \T_h} \Bigl(\int_T (f + \Delta u_h) (v-v_h) -
    \frac{1}{2} \sum_{F \subset \partial T\setminus\partial\Omega} \int_{F} \jump{\nabla u_h} (v-v_h) \Bigr) \\
    &\lesssim \sum_{T \in \T_h} \Bigl( \norm{f + \Delta u_h}_{0,T} \norm{v-v_h}_{0,T} +
    \frac{1}{2} \sum_{F \subset \partial T\setminus\partial\Omega} \norm{\jump{\nabla u_h}}_{0,F} \norm{v-v_h}_{0,F} \Bigr).
\end{align}
Notice that we have used the continuity of $v-v_h$ across $F$.\\
\boxed{Step 3} The choice of $v_h$ is arbitrary, therefore we can set $v_h = \SZ v$. Then,
\[
\norm{v-\SZ v}_{0,T} \lesssim h_T \abs{v}_{1,\Delta T}
\]
which is the approximation property with $m=0$, $l=1$. Analogously,
\[
\norm{v-\SZ v}_{0,F} \lesssim h_F^{\frac{1}{2}} \abs{v}_{1,\Delta F}.
\]
Wrapping things up, we have shown that
\begin{align}
\abs{\langle A(u-u_h),v-\SZ v \rangle} &\lesssim
\sum_{T \in \T_h} \Bigl( h_T \norm{f + \Delta u_h}_{0,T} +
    \frac{1}{2} \sum_{F \subset \partial T\setminus\partial\Omega} h_F^\frac{1}{2} \norm{\jump{\nabla u_h}}_{0,F} \Bigr) c_T(v) \\
    &= \sum_{T \in \T_h} \eta_T \, c_T(v) \\
    &\le \Bigl( \sum_{T \in \T_h} \eta_T^2 \Bigr)^\frac{1}{2}
    \Bigl( \sum_{T \in \T_h} c_T(v)^2 \Bigr)^\frac{1}{2},
\end{align}
where
\[
c_T(v) = \max \bigl( \abs{v}_{1,\Delta T}, \max_{F \subset \partial T} \abs{v}_{1,\Delta F} \bigr).
\]
\boxed{Step 4} We have to show that
\[
\sum_{T \in \T_h} c_T(v)^2 \le c \norm{v}_{1,\Omega}^2.
\]
for some constant $c$ independent of $h$. Let us exploit the domain additivity of the integral: how many times will $\abs{v}_{1,T}$ show up, at most, in the sum? Let $M$ be the maximum number of times that an element $T$ is part of the neighborhood of another one:
\[
M = \max_{T \in \T_h} \card \Set{T'\in\T_h \text{ s.t. } T \in \Delta T'}.
\]
Similarly, let $N$ be the maximum number of faces an element can be a neighbour of:
\[
N = \max_{T \in \T_h} \card \Set{F\in\E_h \text{ s.t. } T \in \Delta F}.
\]
Clearly, the integers $M$ and $N$ only depend on the shape-regularity of the mesh, and are bounded by a constant independent of $h$, hence we get:
\[
\sum_{T \in \T_h} c_T(v)^2 \le \max(M,N) \norm{v}_{1,\Omega}^2
\le c \norm{v}_{1,\Omega}^2.
\]
This concludes our proof.


\subsection{Quasi-local optimality}

With some effort, we can also show a \emph{quasi-local optimality bound}. Some things need to be defined:
\begin{itemize}
    \item Given a face $F$, we let $D_F$ be the union of elements that share it:
    \[
    D_F := \Set{\mathring{\overline{\bigcup_m \overline{T_m}}} \,\text{ s.t. } \,\overline{T_{m_1}} \cap \overline{T_{m_2}} = F}.
    \]
    In particular, with the assumption that a face can be shared only by two elements, $D_F$ is reduced to the interior part of $\overline{T_{m_1}} \cup \overline{T_{m_1}}$ for some indices $m_1$ and $m_2$.
    \item Given an element $T$, let $D_T$ the union of elements that share a face with $T$:
    \[
    D_T := \Set{\mathring{\overline{\bigcup_m \overline{T_m}}} \,\text{ s.t. }\, \overline{T_m} \cap \overline{T} = F \ne \varnothing, \text{ with }\dim(F)=d-1}.
    \]
\end{itemize}
We shall also introduce \emph{bubble functions}. A bubble function on an element $T$ is a function $b_T$ such that:
\begin{itemize}
    \item $0 \le b_T \le 1$ in $\overline{T}$;
    \item $b_T \in H^1_0(T)$;
    \item $\exists D\subset T$ s.t. $\abs{D}>0$ and $\restr{b_T}{D} \ge \frac{1}{2}$.
\end{itemize}
When $T \subset \R^d$ is a simplex or a hypercube, bubble functions satisfy the following properties, given $\phi \in \P^k(T)$ with $k\ge1$:
\begin{itemize}
    \item $\norm{b_T \phi}_{0,T} \lesssim \norm{\phi}_{0,T} \lesssim \norm{b_T^{\frac{1}{2}} \phi}_{0,T}$;
    \item $\abs{b_T \phi}_{1,T} \lesssim h_T^{-1} \norm{\phi}_{0,T}$ (inverse inequality).
\end{itemize}
\rev{maybe say a word on how this is proved (passing to the finite element, bla bla)}
We want to also introduce bubble functions on faces. These are more complicated, because we would like to \emph{lift} a face, i.e. build an object on $T$ from its face $F$. In order to do this, we consider a \emph{lifting operator}
\[
\RE: \P^k(F) \to \P^k(D_F)
\]
where $\P^k(D_F)$ is considered element-wise.
\rev{tell how this works}

A bubble function on a face $F$ \emph{and} on $D_F$, meaning that it is a bubble function for both $F$ as a $(d-1)$-manifold and for $D_F$ as a $d$-manifold, is a function $b_F$ such that:
\begin{itemize}
    \item $0 \le b_F \le 1$ in $\overline{F}$;
    \item $\restr{b_F}{F} \in H^1_0(F)$ and $b_F \in H^1_0(D_F)$;
    \item $\exists \Tilde{D}_F\subset F$ such that $\abs{\Tilde{D}_F}>0$ and $\restr{b_F}{\Tilde{D}_F} \ge \frac{1}{2}$;
    \item $\exists \Tilde{D}_{D_F}\subset D_F$ such that $\abs{\Tilde{D}_{D_F}}>0$ and $\restr{b_F}{\Tilde{D}_{D_F}} \ge \frac{1}{2}$.
\end{itemize}
Note that $\abs{\Tilde{D}_F}$ is a measure of dimension $d-1$, while $\abs{\Tilde{D}_{D_F}}$ is a measure dimension $d$. Given $\phi \in \P^k(F)$, bubble functions on faces satisfy the following properties:
\begin{itemize}
    \item $\norm{b_F \phi}_{0,F} \lesssim \norm{\phi}_{0,F} \lesssim \norm{b_F^{\frac{1}{2}} \phi}_{0,F}$;
    \item $h_F^\frac{1}{2} \norm{\phi}_{0,F} \lesssim \norm{b_F \RE(\phi)}_{0,D_F} \lesssim h_F^\frac{1}{2} \norm{\phi}_{0,F}$;
    \item $\abs{b_F \RE(\phi)}_{1,D_F} \lesssim h_F^{-\frac{1}{2}} \norm{\phi}_{0,F}$.
\end{itemize}

Bubble functions will come in handy to prove the following result, because using them as weights when integrating by parts will cancel out boundary terms.
\begin{theorem}[local optimality, Verf\"{u}rt 1992-94]
    In the setting above, if $T$ is \emph{shape regular}, i.e. $h_F \sim h_T$, the following inequality holds:
    \[
    \eta_T \lesssim \abs{u - u_h}_{1,D_T} + h_T \inf_{v_h \in V_h} \norm{f - v_h}_{0,D_T}.
    \]
\end{theorem}
\begin{proof}
    We start by estimating the term $\norm{f + \Delta u_h}_{0,T}$ via the triangle inequality:
    \[
    \norm{f + \Delta u_h}_{0,T} \le \norm{f - v_h}_{0,T} + \norm{v_h + \Delta u_h}_{0,T} \qquad \forall v_h \in V_h.
    \]
    Next, consider a bubble function $b_T$ on $T$ and exploit its properties to give an estimate to the second term at the right-hand side:
    \begin{align}
        \norm{v_h + \Delta u_h}_{0,T}^2
        & \lesssim \norm{b_T^{\frac{1}{2}} (v_h + \Delta u_h)}_{0,T}^2 \\
        & = \int_T (v_h + \Delta u_h) b_T (v_h + \Delta u_h) \\
        & = \int_T (v_h -f + \Delta (u_h-u)) b_T (v_h + \Delta u_h) \\
        & = \int_T (v_h -f) b_T (v_h + \Delta u_h) + \int_T \Delta(u_h -u) b_T (v_h + \Delta u_h) \\
        & \lesssim \norm{v_h -f}_{0,T} \norm{b_T (v_h + \Delta u_h)}_{0,T} + \int_T \Delta(u_h -u) b_T (v_h + \Delta u_h) \\
        & \lesssim \norm{v_h -f}_{0,T} \norm{v_h + \Delta u_h}_{0,T} + \int_T \Delta(u_h -u) b_T (v_h + \Delta u_h).
    \end{align}
    In particular, $v_h + \Delta u_h$ is indeed a polynomial on $T$ and we have added $0 = f + \Delta u$ at a certain point. Now integrate by parts: being $b_T \in H^1_0(T)$, we get no boundary terms, hence the chain of inequalities continues:
    \begin{align}
        \norm{v_h + \Delta u_h}_{0,T}^2
        & \lesssim \norm{v_h -f}_{0,T} \norm{v_h + \Delta u_h}_{0,T} - \int_T \nabla(u_h -u) \nabla(b_T (v_h + \Delta u_h)) \\
        & \lesssim \norm{v_h -f}_{0,T} \norm{v_h + \Delta u_h}_{0,T} + \abs{u_h-u}_{1,T} \abs{b_T (v_h + \Delta u_h)}_{1,T} \\
        & \lesssim \norm{v_h -f}_{0,T} \norm{v_h + \Delta u_h}_{0,T} + \abs{u_h-u}_{1,T} \cdot h_T^{-1} \norm{v_h + \Delta u_h}_{0,T} .
    \end{align}
    After canceling $\norm{v_h + \Delta u_h}_{0,T}$ from both sides of the inequality, we get
    \[
        \norm{v_h + \Delta u_h}_{0,T}
        \lesssim \norm{v_h -f}_{0,T} + h_T^{-1} \abs{u_h-u}_{1,T}
    \]
    which implies
    \begin{equation}\label{eqn:verfurt1}
    h_T \norm{f + \Delta u_h}_{0,T}
    \lesssim h_T \norm{f - v_h}_{0,T} + \abs{u_h-u}_{1,T} \qquad \forall v_h \in V_h.        
    \end{equation}

    On to the second term in the error estimator: the idea is to recover $\abs{u_h - u}_{1,T}$ via a bubble function $b_F$ on $F$. We have:
    \begin{align}
        \norm{\jump{\nabla u_h}}_{0,F}^2
        & \lesssim \norm{b_F^{\frac{1}{2}} \jump{\nabla u_h}}_{0,F}^2 \\
        & = \int_F \jump{\nabla u_h} b_F \jump{\nabla u_h} \, dF \\
        & = \int_F \jump{\nabla u_h - \nabla u} b_F \jump{\nabla u_h} \, dF \\
        & = \int_F \jump{\nabla u_h - \nabla u} b_F \RE(\jump{\nabla u_h}) \, dF .
    \end{align}
    In particular, the last two equalities follow from $u$ being the exact solution, hence satisfying $\jump{\nabla u} = 0$ on $F$, and from $\jump{\nabla u_h} = \RE(\jump{\nabla u_h})$ on $F$. To proceed, we use the following remark:
    \begin{remark}
        If $v, z \in H^1(D_F)$, integration by parts implies:
        \begin{align}
            \sum_{T \in D_F} \int_T -\Delta z v
            & = \sum_{T \in D_F} \Bigl(\int_T \nabla z \nabla v - \int_{\partial T} \nabla z \cdot n \, v \Bigr) \\
            & = \int_{D_F} \nabla z \nabla v - \int_F \jump{\nabla z} v - \int_{\partial D_F} \nabla z \cdot n \, v
        \end{align}
        In particular, if also $v \in H^1_0(D_F)$ we obtain:
        \[
        \int_F \jump{\nabla z} v = \int_{D_F} \Delta z v + \int_{D_F} \nabla z \nabla v.
        \]
    \end{remark}
    In our case, the function $v = b_F \RE(\jump{\nabla u_h})$ is indeed in $H^1_0(D_F)$, hence we can continue the chain of inequalities above:
    \begin{align}
        \norm{\jump{\nabla u_h}}_{0,F}^2
        & \lesssim \int_{D_F} (\Delta u_h - \Delta u) b_F \RE(\jump{\nabla u_h}) + \int_{D_F} \nabla (u_h - u) \nabla (b_F \RE(\jump{\nabla u_h}))\\
        & = \int_{D_F} (\Delta u_h + f) b_F \RE(\jump{\nabla u_h}) + \int_{D_F} \nabla (u_h - u) \nabla (b_F \RE(\jump{\nabla u_h})) \\
        & \lesssim \norm{\Delta u_h + f}_{0,D_F} \norm{b_F \RE(\jump{\nabla u_h})}_{0,D_F} + \abs{u_h - u}_{1,D_F} \abs{b_F \RE(\jump{\nabla u_h})}_{1,D_F} \\
        & \lesssim \norm{\Delta u_h + f}_{0,D_F} \cdot h_F^\frac{1}{2} \norm{\jump{\nabla u_h}}_{0,F} + \abs{u_h - u}_{1,D_F} \cdot h_F^{-\frac{1}{2}} \norm{\jump{\nabla u_h}}_{0,F}.
    \end{align}
    As a consequence, simplifying $\norm{\jump{\nabla u_h}}_{0,F}$ from both the sides of the inequality leads to
    \[
    \norm{\jump{\nabla u_h}}_{0,F} \lesssim h_F^\frac{1}{2} \norm{\Delta u_h + f}_{0,D_F} 
    + h_F^{-\frac{1}{2}} \abs{u_h - u}_{1,D_F}
    \]
    and finally
    \begin{equation}\label{eqn:verfurt2}
    h_F^\frac{1}{2} \norm{\jump{\nabla u_h}}_{0,F}
    \lesssim h_F \norm{\Delta u_h + f}_{0,D_F} + \abs{u_h - u}_{1,D_F}.
    \end{equation}
    The conclusion follows immediately grouping inequalities~\eqref{eqn:verfurt1} and~\eqref{eqn:verfurt2} and taking the infimum over $v_h \in V_h$. In particular, summing over $F\subset \partial T$ is the reason why $D_T$ appears in the end.
\end{proof}


\section{Using the error estimator to refine a mesh}

Having a reliable error estimator means that we can impose a certain tolerance to be satisfied:
\[
\abs{u - u_h}_{1,\Omega} \lesssim \Bigl( \sum_T \eta_T^2 \Bigr)^\frac{1}{2} \lesssim \text{tol}.
\]
This way, we can control the error on the solution after having computed the approximation and, eventually, refine the mesh to improve accuracy.

If $\T^i$ is the $i$-th stage of refinement of a triangulation $\T$ and $M^i := \#\T^i$ is its number of elements, one possibility is to require for the following stage $\T^{i+1}$ to satisfy
\[
\eta_T^2 \lesssim \frac{\text{tol}^2}{M^{i+1}} \qquad \forall T \in \T^{i+1}.
\]
\rev{is this someway related to the fixed number algorithm?}

Another way is to use the \emph{bulk chasing algorithm}, also called \emph{D\"{o}rfler} or \emph{fixed fraction} algorithm, or \emph{bulk criterion}. Given $0<\theta<1$, we define the total error at stage $i$
\[
E_{tot} = \Bigl( \sum_{T \in \T^i} \eta_T^2 \Bigr)^\frac{1}{2}.
\]
The aim is to mark for refinement the smallest subset $M \subset \T^i$ such that
\[
\theta E_{tot} \le \Bigl( \sum_{T \in M} \eta_T^2 \Bigr)^\frac{1}{2}
\]
with the following steps:
\begin{enumerate}
    \item Order the elements $T^i_k \in \T^i$ based on the reliability estimator, from the greatest value to the smallest one:
    \[
    \eta_{T^i_k} \le \eta_{T^i_j} \quad \text{when } j \le k.
    \]
    \item Compute $E_{tot}$.
    \item Find $\overline{k}$ such that
    \[
    \Bigl( \sum_{j=1}^{\overline{k}} \eta_{T^i_j}^2 \Bigr)^\frac{1}{2} \ge \theta E_{tot}, \qquad \Bigl( \sum_{j=1}^{\overline{k}-1} \eta_{T^i_j}^2 \Bigr)^\frac{1}{2} < \theta E_{tot}.
    \]
    Then mark for refinement up to element $\overline{k}$.
\end{enumerate}
With an analogous procedure, we can mark another fraction of elements for coarsening, if desired. The result of this process is that the error tends to be distributed "uniformly" over the mesh. In the context of quasi uniform meshes (i.e. $h_{min} \sim h_{max}$) we have
\[
\min_{T \in \T} h_T \lesssim \max_{T \in \T} h_T \lesssim \max_{T \in \T} h_T^{-1}
\lesssim \min_{T \in \T} h_T^{-1},
\]
therefore
\[
\frac{\abs{\Omega}}{M} \sim h_T^d, \quad \text{i.e. } h_T \sim M^{-\frac{1}{d}}.
\]
To give an idea of the impact of a mesh refinement on the global error, let us recall the a priori error estimate~\eqref{ineq:cea_scaling2} we produced in the previous chapter:
\[
\norm{u-u_h}_m \lesssim h^{l-m} \abs{u}_{l} \quad \forall 0\le m \le l-1 \text{ suitable},
\]
with $l \le k+1$, where $k$ is the finite element degree.
\rev{did I say that last thing in the previous chapter? Check}
Then, this implies:
\[
\norm{u-u_h}_m \lesssim M^{-\frac{l-m}{d}} \abs{u}_{l} \quad \forall 0\le m \le l-1 \text{ suitable}.
\]
For example, in the case of a linear FEM approximation ($k=1$) for a Poisson problem on a Lipschitz, convex domain, we have said that this equality holds for $l=2$ and $m=0,1$, because we have also Nitsche's lemma in hand:
\[
\norm{u-u_h}_m \lesssim M^{-\frac{2-m}{d}} \abs{u}_{2}, \quad m=0,1.
\]
This means that the assumption $u \in H^2 \cap H^1_0$ is essential. What happens if $u \not\in H^2$? It can be proven that, if the a posteriori error estimator $\eta$ is equidistributed, i.e. $\eta_k \sim \delta \, \forall k$ (for some value $\delta$), then
\[
\norm{u-u_h}_m \lesssim M^{-\frac{2-m}{d}} \abs{u}_{1}, \quad m=0,1.
\]
Moreover, in that latter context, the hypothesis of quasi-uniformity of the mesh is not needed.

\rev{there's also another inequality in my notes}

Refining the mesh a posteriori implies that, in order to reach a desired level of accuracy, we may need to compute the numerical solution $u_h$ multiple times on progressively refined meshes. Moreover, for time dependent PDEs, further problems arise when dealing with meshes and basis functions that may be different from one time step to another. For the moment, let us stick to stationary problems and conclude this section with some comments and tips on how to use adaptive mesh refinement intelligently:
\begin{itemize}
    \item Calculating the solution multiple times may be worrying in terms of computational time, but needs not to be always. We can put ourselves in the situation where the leading numerical cost is comparable to that of the iteration with the most refined mesh. For example, suppose that each mesh refinement roughly doubles the number of cells and suppose we are using linear elements. If we have $M$ cells at some time, at previous iterations the cells were $\frac{M}{2}$, $\frac{M}{4}$, etc. If we had a perfect solver for the linear system which takes $O(M)$ flops, the total computational effort would be $O(M)+O\bigl(\frac{M}{2}\bigr)+O\bigl(\frac{M}{4}\bigr)+\dots=O(2M)$. To make a comparison with the best possible scenario, a global refinement in 2D quadruplicates the number of cells, hence the total cost would be $O(\frac{4}{3}M)$, which is not too much different.
    Furthermore, when the finite element degree is greater than $1$, this phenomenon is even more evident.
    \item At starting point, i.e. when the numerical solution has not been computed yet, we can assume $u_h=0$, hence
    \[
    \eta_T = h_T \norm{f}_{0,T}.
    \]
    This can be used as a first estimate in order to make the error already equidistributed.
    \item At each refinement step, instead of providing a random guess to start the iterative solver, we can use the solution $u_h$ computed at the previous step, which hopefully will be nearer the exact solution and speed up the convergence. This requires to interpolate that solution to the new mesh, in order to use a vector of the right dimension.
    \item Some solvers such as the conjugate gradient method have a speed-up property: high frequency oscillations do converge very fast (= in a few steps) to the exact ones. Also, those high frequency oscillations carry the largest part of the error. Therefore, at intermediate refinement steps, one could apply just a few iterations of the solver and obtain a crude approximation of the solution, which still will give good indications for what cells need to be refined. Then, after some refinements, the solver can be called with the right tolerance and a full computation of the solution can be done.
    \item In some problems, a lighter version of the estimator $\eta_T$, containing only the jump terms, is used:
    \[
        \tilde{\eta}_T = \sum_{F \subset \partial T} \frac{1}{2} h_F^{\frac{1}{2}} \norm{\jump{\nabla u_h}}_{0,F}.
    \]
    This is known as the \emph{Kelly error estimator}. Kelly, Babu\v{s}ka and others proved in~\cite{kelly83} that it is equivalent to the full estimator for the Laplace equation with constant coefficients.
    \rev{check the reference}
\end{itemize}

\rev{the problem of hanging nodes should be at least mentioned}


% !TeX source = ../main.tex

\chapter{Discontinuous Galerkin methods}

\section{Strang's lemmas}
Consider a bounded and coercive operator $A \in \L(V,V')$. As we know, to solve numerically the problem $A u = f$ in $V'$, we build a sequence of subspaces $V_h \subset V$ and restrict the problem to $V_h$:
\[
\text{find $u_h \in V_h$ s.t. } A u_h = f \quad \text{in $V'$}.
\]
In reality, we don't have at disposal $A$ and $f$ themselves, since their action on the shape functions $v_h \in V_h$ involves integrals, which are approximated by quadrature formulas. Hence, we are implicitly substituting $A$ and $f$ with a sequence of numerical approximations $A_h$ and $f_h$, and the real (approximated) problem we solve on a computer is the following:
\[
\text{find $u_h \in V_h$ s.t. } A_h u_h = f_h \quad \text{in ${V_h}'$}
\]
where $A_h \in \L(V_h,{V_h}')$ and $f_h \in {V_h}'$. This action is called in literature a \emph{variational crime}, because we are changing the variational problem. We now want to quantify the \emph{consistency error} introduced by this substitution.

\begin{lemma}[First Strang's lemma]
In the setting above, suppose that the operators $A_h$ are $h$-uniformly bounded and coercive, i.e. there exist $c_0, \alpha_0$ such that $\forall h$
\begin{align}
\langle A_h u_h, v_h \rangle \le c_0 \norm{u_h}_V \norm{v_h}_V \qquad &\forall u_h,v_h \in V_h, \\
\langle A_h u_h, u_h \rangle \ge \alpha_0 \norm{u_h}_V^2 \qquad &\forall u_h \in V_h.
\end{align}
Then:
\[
\norm{u-u_h}_V \le \Bigl( 1+\frac{c_0}{\alpha_0} \Bigr) \inf_{v_h \in V_h} \bigl( \norm{u-v_h}_V + 
\norm{A_h v_h - A v_h}_{{V_h}'} \bigr) + \norm{f-f_h}_{{V_h}'}
\]
where the norm of a functional $f \in V_h'$ is defined as usual:
\[
\norm{f}_{{V_h}'} := \sup_{v_h \in V_h} \frac{\langle f, v_h \rangle}{\norm{v_h}}.
\]
\end{lemma}
\rev{check the hypotheses (see below). Write bounded and coercive or continuous and elliptic? Also, if $A_h \in \L(V,V')$ what is the right inequality?}

\begin{remark}
The term $\norm{A_h v_h - A v_h}_{{V_h}'}$ is the \emph{$A$-consistency error}. The term $\norm{f-f_h}_{{V_h}'}$ is the \emph{$f$-consistency error} and is essential, because we really want to check for consistency only in the FEM subspace.
Remember that the norms in $V$ and $V_h$ are the same. Instead, the norms in $V'$ and ${V_h}'$ are different.

\rev{Add some other comments from the notes.}
\end{remark}

\begin{proof}
We begin applying the triangle inequality:
\[
\norm{u-u_h}_V \le \norm{u-v_h}_V + \norm{v_h - u_h}_V.
\]
Then we look for a bound for $\norm{v_h - u_h}_V$. By coercivity of $A_h$ we have
\[
\alpha_0 \norm{v_h - u_h}_V^2 \le \langle A_h(v_h - u_h), v_h- u_h \rangle.
\]
Now add and subtract $\langle A u - A v_h, v_h- u_h \rangle$. After rearranging the terms and observing that $Au = f$ and $A_h u_h = f_h$, we get:
\begin{align}
\alpha_0 \norm{v_h - u_h}_V^2 &\le  \langle A v_h - A u, v_h- u_h \rangle +
\langle A_h v_h - A v_h, v_h- u_h \rangle + \langle f - f_h, v_h- u_h \rangle \\
& \le \bigl( \norm{A} \norm{u-v_h}_V + \norm{A_h v_h - A v_h}_{{V_h}'} + \norm{f - f_h}_{{V_h}'} \bigr)
\norm{v_h - u_h}_V.
\end{align}
If we divide by $\alpha_0 \norm{v_h - u_h}_V$, we conclude that
\[
\norm{v_h - u_h}_V\le \frac{1}{\alpha_0} \Bigl( \norm{A} \norm{u-v_h}_V + \norm{A_h v_h - A v_h}_{{V_h}'} + \norm{f - f_h}_{{V_h}'} \Bigr).
\]
The conclusion now is immediate. \rev{...but something is wrong with the constants. Also the hypothesis of $h$-uniform boundness for the $A_h$ seems not necessary.}
\end{proof}

We can also consider a second \emph{variational crime}: the case of non-conforming finite element function spaces, i.e. $V_h \not\subset V$. For instance, think of the situation of approximating continuous functions with piecewise continuous ones. 
\rev{Is really this the meaning of non-conforming space? Or $V_h \not\subset V$ is a consequence of it?}
In this case, we can find some bound for the error considering the direct sum $\tilde{V} := V + V_h$, provided with some norm $\tnorm{\cdot}$, and working in the space $\tilde{V}$. In particular, the operators $A_h$ must be extended to be in $\L(\tilde{V},\tilde{V}')$ and, similarly, every $f_h$ must be extended to $\tilde{V}$.
\rev{This is a slightly revisited version of the theorem in the notes. I get the concrete idea of having the norm equivalence $\norm{\cdot} \sim \tnorm{\cdot}$ in $V$, but it is not necessary in the lemma. So how do we want to use it? Is the equivalence needed also in $V_h$?}
%suppose the norm $\tnorm{\cdot}$ is equivalent to the norm $\norm{\cdot}$ on $V$:
%\[
%\exists c, C>0 \text{ s.t. } c\norm{u} \le \tnorm{u} \le C\norm{u} \quad \forall u \in V;
%\]

\begin{lemma}[Second Strang's lemma]
Let $\tilde{V} = V + V_h$ and let $\tnorm{\cdot}$ be a norm in $\tilde{V}$ (possibly $h$-dependent). Let also $A_h$ be operators in $\L(\tilde{V},\tilde{V}')$. Suppose that the operators $A_h$ are $h$-uniformly bounded \emph{in $\tilde{V}$} and coercive \emph{in $V_h$} with respect to the norm $\tnorm{\cdot}$, i.e. there exist $c_0, \alpha_0$ such that $\forall h$
\begin{align}
\langle A_h u, v \rangle \le c_0 \tnorm{u} \tnorm{v} \qquad &\forall u,v \in \tilde{V}, \\
\langle A_h u, u \rangle \ge \alpha_0 \tnorm{u}^2 \qquad &\forall u \in V_h.
\end{align}
Then, if $u_h \in V_h$ is the solution of $A_h u_h = f_h$ in ${V_h}'$, we have:
\[
\tnorm{u-u_h} \le \Bigl(1 + \frac{c_0}{\alpha_0} \Bigr) \inf_{v_h \in V_h} \tnorm{u-v_h} + \frac{1}{\alpha_0} \tnorm{A_h u - f_h}_{*,h}
\]
where $\tnorm{\cdot}_{*,h}$ is defined as follows:
\[
\tnorm{f}_{*,h} := \sup_{v_h \in V_h} \frac{\langle f, v_h \rangle}{\tnorm{v_h}}.
\]
\end{lemma}
\begin{proof}
We proceed in the same fashion of the first Strang's lemma. We start applying the triangle inequality:
\[
\tnorm{u-u_h} \le \tnorm{u-v_h} + \tnorm{v_h - u_h}.
\]
Then we look for a bound for $\tnorm{v_h - u_h}$. By coercivity of $A_h$ we have
\begin{align}
\alpha_0 \tnorm{v_h - u_h}^2 &\le \langle A_h(v_h - u_h), v_h- u_h \rangle \\
&= \langle A_h v_h - f_h , v_h- u_h \rangle.
\end{align}
Now add and subtract $\langle A_h u, v_h- u_h \rangle$ and rearrange the terms to use the boundedness of $A_h$:
\begin{align}
\alpha_0 \tnorm{v_h - u_h}^2 &\le  \langle A_h (v_h - u), v_h- u_h \rangle +
\langle A_h u - f_h, v_h- u_h \rangle \\
& \le \bigl( c_0 \tnorm{u-v_h} + \tnorm{A_h u - f_h}_{*,h} \bigr)
\tnorm{v_h - u_h}.
\end{align}
To get to the conclusion, divide by $\alpha_0 \tnorm{v_h - u_h}$ and couple the inequality with the triangle one that we used at the start of the proof.
\end{proof}



\section{Discontinuous Galerkin methods for the Poisson problem}

Consider the following model problem:
\begin{equation}\label{eqn:poisson_1}
\begin{cases}
-\Delta u = f \quad &\text{in $\Omega$} \\
u = 0 \quad &\text{on $\partial\Omega$}.
\end{cases}
\end{equation}
with $\Omega$ convex and polygonal, $f\in L^2(\Omega)$. As always, the space $V$ chosen for its weak formulation is $H^1_0(\Omega)$. If $V_h \not\subset V$, we see every triangle of the mesh $\T_h$ as an individual (not as part of a whole). We set:
\begin{align}
&\T := \bigcup_{T \in \T_h} \mathring{T} = \T_h \setminus \bigcup_{T} \partial T && \text{(cells)} \\
&\E^0 := \bigcup_{T_i,T_j \in \T_h} \overline{T_i} \cap \overline{T_j} && \text{(interior edges)} \\
&\E^{\partial\Omega} := \bigcup_{T_i \in \T_h} \overline{T_i} \cap \partial\Omega && \text{(boundary edges)} \\
&\E := \E^0 \cup \E^{\partial\Omega} && \text{(all edges)}.
\end{align}
In particular, we denote each edge in the interior as $e_{ij} := \overline{T_i} \cap \overline{T_j}$ (of course, in 3D what we call "edge" will be a "face"). Observe that, with this notation, we also have
\begin{align}
&\E = \bigcup_{T \in \T_h} \partial T, \\
&\E^0 = \E \setminus \partial \Omega, \\
&\E^{\partial\Omega} = \E \cap \partial\Omega.
\end{align}
On the new mesh $\T$ we consider the "broken" scalar product on cells
\[
(u,v)_\T := \sum_{T \in \T_h} (u,v)_T
\]
and the "broken" scalar product on edges
\[
\langle u,v \rangle_\E := \sum_{e \in \E} (u,v)_e.
\]
Consider now the following choice for $V_h$:
\[
V_h = \Set{v \in L^2(\T) : \restr{v}{T} \in \P^l(T) \,\,\forall T \in \T_h}
\]
and set $\tilde{V} = V + V_h$. The FEM space we introduced is the space of \emph{discontinuous piecewise polynomials} on $\T$.

We have said before that all cells in $\T$ are considered as individuals. Therefore, when two cells, say $T^+$ and $T^-$, share an edge $e$ (we call this edge an \emph{interface}), a function $u\in\tilde{V}$ may have different values on it, depending on which cell we are looking at. We would like to have some control on this behaviour, thus we call $u^+$ the value of $u$ on $e \cap T^+$, $u^-$ the value of $u$ on $e \cap T^-$. Also we define $n^+$ as the outer normal to $T^+$ and $n^-$ as the outer normal to $T^-$. With these in mind, we introduce the \emph{jump} of $u$:
\[
\jump{u} := \begin{cases}
u^+ \cdot n^+ + u^- \cdot n^- \quad &\text{in $\E^0$} \\
u \cdot n \quad &\text{in $\E^{\partial\Omega}$}
\end{cases}
\]
and the \emph{average} of $u$:
\[
\avg{u} := \begin{cases}
\frac{1}{2} u^+ + \frac{1}{2} u^-  \quad &\text{in $\E^0$} \\
u \quad &\text{in $\E^{\partial\Omega}$}.
\end{cases}
\]
\begin{remark}
The product involved in the definition of jump has to be intended as a "trivial" product if $u$ is scalar, as a scalar product if $u$ is a vector. Remember:
\begin{itemize}
    \item the jump of a scalar is a vector, e.g. $\jump{u} = u^+ \mathbf{n^+} + u^- \mathbf{n^-}$;
    \item the jump of a vector is a scalar, e.g. $\jump{\nabla u} = \nabla u^+ \cdot \mathbf{n^+} + \nabla u^- \cdot \mathbf{n^-}$.
\end{itemize}
Later on we will use the following identity, which holds for $a$ scalar and $\mathbf{b}$ vector (in reality, it is also true for \emph{any} $a$ and $b$, being careful with the products):
\begin{equation}\label{eqn:jump_avg}
a^+ \mathbf{n^+} \cdot \mathbf{b^+} + a^- \mathbf{n^-} \cdot \mathbf{b^-} = \jump{a} \cdot \avg{\mathbf{b}} + \avg{a} \jump{\mathbf{b}}.
\end{equation}
The proof is simple: just add and subtract the right quantities, then observe that $\mathbf{n^+}=-\mathbf{n^-}$.
\end{remark}

In order to use Strang's lemma to produce an error estimate on the space $\tilde{V}$, we consider the following norm on it:
\[
\tnorm{u}^2 := \sum_{T \in \T} \abs{u}^2_{1,T} + \sum_{e \in \E} \frac{1}{\abs{e}} \norm{\jump{u}}^2_{0,e}
\]
where $\abs{e}$ denotes the diameter of the edge $e$.
\rev{check that it is indeed the diameter of the edge}
This is indeed a norm on the space $\tilde{V}$; in particular, observe that $\tnorm{u} = 0$ implies $u=0$. In fact:
\begin{itemize}
\item $u$ would have $H^1$ seminorm equal to zero on each triangle, hence it would be constant on each $T$;
\item $u$ also would have its jump equal to zero on every edge, hence such constants would be zero for every $T$.
\end{itemize}
The norm we have defined is clearly mesh dependent, but coincides with the $H^1$ seminorm $\abs{\cdot}_{1,\Omega}$ for functions in $V = H^1_0(\Omega)$ (remember that $\Omega$ is polygonal).

We then rewrite the problem~\eqref{eqn:poisson_1} in a form that involves more deeply the triangulation $\T$:
\begin{equation}\label{eqn:poisson_2}
\begin{cases}
-\Delta u = f \quad &\text{in $\T$} \\
\jump{u} = 0 \quad &\text{in $\E$} \\
\jump{\nabla u} = 0 \quad &\text{in $\E^0$}.
\end{cases}
\end{equation}
We have to prove that the two forms are equivalent. Since $f \in L^2(\Omega)$, by the regularity theory for the Laplace problem, we know that the solution $u$ of~\eqref{eqn:poisson_1} is in $H^2(\Omega) \cap H^1_0(\Omega)$. Therefore, it makes sense to consider the space below:
\[
H^2(\T_h) := \Set{u \in L^2(\Omega) : \restr{u}{T} \in H^2(T) \,\, \forall T\in\T_h}
\]
and look, a priori, for solutions $u\in H^2(\T_h)$. Indeed, if $u$ solves~\eqref{eqn:poisson_1}, it trivially solves also~\eqref{eqn:poisson_2}. The converse can be shown via a weak formulation for~\eqref{eqn:poisson_2}.

Let us look for a weak form for~\eqref{eqn:poisson_2}. It has been shown in~\cite{bcms04} that all discontinuous Galerkin methods used for the Poisson problem (also for more general ones) can be written in the following form:
\[
(-\Delta u - f, B_0 v)_\T + \langle \jump{u}, B_1 v \rangle_\E + \langle \jump{\nabla u}, B_2 v \rangle_{\E^0} = 0
\]
for some operators $B_0$, $B_1$, $B_2$ from $H^2(\T_h)$ to $L^2(\Omega)$, $\bigl(L^2(\E)\bigr)^d$ and $L^2(\E^0)$, respectively. This can be seen as a way of enforcing a linear relation among the three residuals of the three equations in~\eqref{eqn:poisson_2}. In order to find an expression for these operators, we observe that integration by parts leads to this identity on each triangle $T$:
\[
(- \Delta u - f , v )_T = (\nabla u, \nabla v)_T - (f,v)_T - \int_{\partial T} \nabla u \cdot n \,v.
\]
The last integral can be split into a sum of contributions from the edges that belong to $\partial T$, i.e.
\[
\int_{\partial T} \nabla u \cdot n \,v = \sum_{e \in \E \cap \partial T} \int_{e} \nabla u \cdot n \,v
\]
If we sum over $T$, each edge in the interior contributes to the sum two times, one for each cell that shares it. Hence, using the identity~\eqref{eqn:jump_avg} we get:
\[
(- \Delta u - f , v )_\T = (\nabla u, \nabla v)_\T - (f,v)_\T -
\langle \jump{\nabla u}, \avg{v} \rangle_{\E^0} -
\langle \avg{\nabla u}, \jump{v} \rangle_{\E}.
\]
Notice that we have used the identity
\[
\langle \jump{\nabla u}, \avg{v} \rangle_{\E^{\partial\Omega}}
= \langle \avg{\nabla u}, \jump{v} \rangle_{\E^{\partial\Omega}}
\]
in order to have to consider $\jump{\nabla u}$ only in $\E^0$. This suggests that, if we make for $B_0$ the most simple choice, i.e. $B_0 v = v$, one could set $B_2 v = \avg{v}$, cancelling out the term $\langle \jump{\nabla u}, \avg{v} \rangle_{\E^0}$ in the weak formulation. Of course, this is not the only choice available for $B_2$, but it is the preferred one in most discontinuous Galerkin methods.

The weak formulation, so far, has become:
\[
(\nabla u, \nabla v)_\T -
\langle \avg{\nabla u}, \jump{v} \rangle_{\E} + 
\langle \jump{u}, B_1 v \rangle_\E = (f,v)_\T \qquad \forall \, v \in H^2(\T_h).
\]
It is worth noting that this formulation contains as a particular case the usual conforming finite element approximation, since the terms $\jump{u}$ and $\jump{v}$ vanish when $u$, $v$ are continuous. In particular, the operator $B_1$ is "not active" for \emph{any} choice of $B_0$ and $B_2$, meaning that conforming finite element methods can be seen as ones that enforce a linear relation between the residual inside the element and the jump in the normal component of the gradient across interelement boundaries.

Let us now fix $B_1$. Our goal is to use quantities that will come in handy when producing estimates in the norm $\tnorm{\cdot}$: a possible choice could be to set $B_1 v = - \avg{\nabla v}$, which would render the discrete formulation symmetric: this is the \emph{symmetric interior penalty method} (SIPG), in its classical nonstabilized version. As will become clear later, we need to stabilize the formulation introducing a \emph{penalisation term}, i.e. we change $B_1$ in the following way:
\[
B_1 v = \gamma \jump{v} - \avg{\nabla v}.
\]
Here is the final formulation:
\[
(\nabla u, \nabla v)_\T -
\langle \avg{\nabla u}, \jump{v} \rangle_{\E} -
\langle \jump{u}, \avg{\nabla v} \rangle_\E +
\gamma \langle \jump{u}, \jump{v} \rangle_\E =
(f,v)_\T \quad \forall \, v \in H^2(\T_h).
\]
In particular, this defines the following operator $A_h$:
\[
\langle A_h u, v \rangle = 
(\nabla u, \nabla v)_\T -
\langle \avg{\nabla u}, \jump{v} \rangle_{\E} -
\langle \jump{u}, \avg{\nabla v} \rangle_\E +
\gamma \langle \jump{u}, \jump{v} \rangle_\E.
\]
We want to prove that $A_h$ satisfies the hypotheses of the second Strang's lemma if $\gamma \ge \frac{c}{h}$ for a suitable value of $c$. The key tools to do this are trace inequalities and the following one:
\[
\pm ab \le \frac{1}{2\epsilon} a^2 + \frac{\epsilon}{2} b^2
\]
which is based on $\Bigl(\frac{a}{\sqrt{\epsilon}}\pm b\sqrt{\epsilon}\Bigr)^2 \ge 0$ and holds for every $\epsilon>0$.
\begin{remark}
    If $u,v \in H^1_0(\Omega)$, then automatically $\langle A_h u, v \rangle = \langle A u, v \rangle$, where $A$ is the usual operator
    \[
    \langle A_h u, v \rangle = (\nabla u, \nabla v)_\T.
    \]    
    This means that solutions that satisfy~\eqref{eqn:poisson_2}, hence its weak formulation, also satisfy the weak formulation of~\eqref{eqn:poisson_1}.
\end{remark}

\textbf{Step 1:} prove that
\[
\langle A_h u, v \rangle \le c_0 \tnorm{u} \tnorm{v} \qquad \forall u,v \in \tilde{V}.
\]
As usual, we have the easy inequality
\[
(\nabla u, \nabla v)_\T \le \abs{u}_{1,\T} \abs{v}_{1,\T}.
\]
Moreover, by regularity theory we know that $u \in \tilde{V}$ implies $u \in H^2$, therefore $\nabla u$ makes sense on edges and the following estimate on each edge $e$ makes sense:
\begin{align}
    \int_e \nabla u \cdot n v &\le \norm{\nabla u}_{0,e} \norm{v}_{0,e} \\
    & = \abs{u}_{1,e} \norm{v}_{0,e} \\
    & \lesssim \abs{e}^{-\frac{1}{2}} \abs{u}_{\frac{1}{2},e} \norm{v}_{0,e} \\
    & \lesssim \abs{e}^{-\frac{1}{2}} \abs{u}_{1,T} \norm{v}_{0,e}.
\end{align}
\rev{I think this doesn't work. Trace theorem would imply the presence of a norm instead of a seminorm, but this would break the proof (we're in $\tilde{V}$, not in $V$). Also, I'm not completely sure about the assumption that we derived from the regularity theory, since being in $\tilde{V}$ is not the same as being in $V$.}
In particular, the last two steps of the inequality are due to an inverse estimate and the trace theorem. The same argument holds of course if we reverse $u$ and $v$. As a consequence, we get:
\begin{align}\label{ineq:dg_stability}
    \abs{\langle \jump{u}, \avg{\nabla v} \rangle_e}
    &\lesssim \abs{e}^{-\frac{1}{2}} \norm{v}_{1,T} \norm{\jump{u}}_{0,e} \\
    &\lesssim \frac{\abs{e}^{-1}}{2\epsilon} \norm{\jump{u}}_{0,e}^2 +
    \frac{\epsilon}{2} \abs{v}_{1,T}^2
\end{align}
and, in a similar fashion:
\[
    \abs{\langle \avg{\nabla u}, \jump{v} \rangle_{e}}
    \lesssim \frac{\abs{e}^{-1}}{2\epsilon} \norm{\jump{v}}_{0,e}^2 +
    \frac{\epsilon}{2} \abs{u}_{1,T}^2.
\]
Finally, we have
\[
\langle \jump{u}, \jump{v} \rangle_e
\le \norm{\jump{u}}_{0,e} \norm{\jump{v}}_{0,e}.
\]
Putting things together and eventually using the $\epsilon$-inequality concludes the proof.

\textbf{Step 2:} prove coercivity, i.e.
\[
\langle A_h u, u \rangle \ge \alpha_0 \tnorm{u}^2 \qquad \forall u \in V_h.
\]
Since the domain is polygonal, we have:
\[
    \langle A_h u, u \rangle =
    \abs{u}_{1,\Omega}^2 -
    2 \sum_e \langle \avg{\nabla u}, \jump{u} \rangle_{e} +
    \gamma \sum_e \norm{\jump{u}}_{0,e}^2.
\]
In particular, we have used the equality $\langle \avg{\nabla u}, \jump{u} \rangle_{e} = \langle \avg{u}, \jump{\nabla u} \rangle_{e}$ to merge two terms. Now set $\gamma \ge \frac{c}{h}$, with $h \le \min_e \abs{e}$, and apply inequality~\eqref{ineq:dg_stability}:
\begin{align}
    \langle A_h u, u \rangle & \ge
    \abs{u}_{1,\Omega}^2 -
    2 \sum_e \langle \avg{\nabla u}, \jump{u} \rangle_{e} +
    c \sum_e \frac{1}{\abs{e}} \norm{\jump{u}}_{0,e}^2 \\
    & \ge \bigl(1-\tilde{c}\epsilon\bigr) \abs{u}_{1,\Omega}^2 +
    \Bigl(c-\frac{\tilde{c}}{\epsilon}\bigr) \sum_e \frac{1}{\abs{e}} \norm{\jump{u}}_{0,e}^2
\end{align}
where the constant $\tilde{c}$ is the one which comes out in inequality~\eqref{ineq:dg_stability}. In order for this expression to be greater or equal than $\alpha_0 \tnorm{u}^2$ for some $\alpha_0$, we have to require the quantities in the parentheses to be greater than zero. It is easy to see that this happens if $c>\tilde{c}$.

\rev{still something doesn't work. In my reference, step 2 is required to be proved only in $V_h$, but here being $\tilde{V}$ is mandatory, otherwise we cannot use the result from the previous step. Still, the original problem about that result is not solved.}

As a consequence, by Strang's lemma we have:
\begin{align}
    \tnorm{u-u_h} &\le \Bigl(1 + \frac{c_0}{\alpha_0} \Bigr) \inf_{v_h \in V_h} \tnorm{u-v_h} + \frac{1}{\alpha_0} \tnorm{A_h u - f_h}_{*,h} \\
    & = \Bigl(1 + \frac{c_0}{\alpha_0} \Bigr) \inf_{v_h \in V_h} \tnorm{u-v_h}.
\end{align}
In fact, the exact solution $u \in V$ satisfies $A u = A_h u = f_h$, hence the second term vanishes.

\rev{final comments to be added, there's another estimate at the end.}

\section{Nitsche's method}

Let us modify the Poisson problem from the previous section and add a nonzero Dirichlet boundary condition:
\begin{equation}\label{eqn:poisson_dbc1}
\begin{cases}
-\Delta u = f \quad &\text{in $\Omega$} \\
u = g \quad &\text{on $\Gamma = \partial\Omega$}.
\end{cases}
\end{equation}
We have already seen in section~\ref{sec:trace_operators} a way to handle boundary conditions: we can split the solution $u$ into a sum $u_0 + u_g$, such that $u_0 = 0$ on $\Gamma$ and $u_g = g$ on $\Gamma$ (in terms of trace), then solve in $H^1_0(\Omega)$. However, this approach has some problems with time-dependent problems, because $u_g$ changes at each time step. Moreover, $V_h$ is a subspace of $H^1(\Omega)$, but at the same time $V_h \not\subset H^1_0(\Omega)$. As an alternative, Nitsche proposed to enforce the boundary condition directly in the weak formulation. In particular, given the usual weak formulation in $H^1_0(\Omega)$:
\[
(\nabla u, \nabla v)_\Omega - \langle \nabla u \cdot n, v \rangle_\Gamma = (f,v)_\Omega
\]
we can add a penalisation term $\gamma \langle u-g, v \rangle_\Gamma$ in order to force $V_h$ to be as close as possible to being a subset of $H^1_0(\Omega)$.
\rev{... meaning that here $V_h = H^1_0$?}
In fact, this term would be equal to zero if $u = g$ on $\Gamma$. Moreover, we decide to keep the formulation symmetric, hence we also add $\langle u, \nabla v \cdot n \rangle_\Gamma$ to the left-hand side. This needs to be compensated somehow in order to keep the problem consistent: the right way to do it is to notice that also $\langle u-g, \nabla v \cdot n \rangle_\Gamma$ would be zero if $u = g$ on $\Gamma$. With all these considerations in mind, we get to the following weak formulation:
\[
(\nabla u, \nabla v)_\Omega - \langle \nabla u \cdot n, v \rangle_\Gamma - \langle u, \nabla v \cdot n \rangle_\Gamma + \gamma \langle u, v \rangle_\Gamma = (f,v)_\Omega + \gamma \langle g, v \rangle_\Gamma - \langle g, \nabla v \cdot n \rangle_\Gamma.
\]
One advantage of this method is that now the terms on the boundary show up in the matrix and in the right-hand side, in particular the matrix can be computed once and for all, or at most has to be updated only when the mesh is changed.

If $\gamma \ge \frac{c}{h}$, the method is consistent.
\rev{why?}

Element-wise, this is the symmetric interior penalty method with Dirichlet boundary conditions enforced via Nitsche's method. 
\rev{complete this}

% !TeX source = ../main.tex

\chapter{Saddle point problems}\label{chap:saddle}

\section{BNB conditions}
In this section we will examine a more general version of the classic problem \eqref{eqn:weak_laxmilgram}, in which the target space of the operator $A$ is allowed to be different from the dual space of the domain.
\begin{remark}
    Recall that a Banach space $W$ is said to be \emph{reflexive} if the natural map from $W$ to $W''$ is an isomorphism (it is, in fact, a canonical isomorphism, since it does not depend on the choice of a basis). This map takes a vector $w$ and associates it to the linear map $J_w$ such that $J_w(f)=f(w)$ for every $f\in W'$. In practice we require this hypothesis so that we can identify $W=W''$.
\end{remark}
Consider a Banach space $V$ and a \emph{reflexive} Banach space $W$. As usual, we have the option to either consider a linear continuous operator $A\in\L(V;W')$ or a bilinear operator $a\in\L(V\times W;\R)$. Given $f\in W'$, the problem we want to solve is: find $u\in V$ such that
\begin{equation}\label{eqn:petrov-galerkin}
    Au=f \ \text{in} \ W', \ \text{or equivalently} \ a(u,w) = \langle f,w \rangle \ \forall w\in W,
\end{equation}
where as usual $a(\cdot,\cdot)$ is defined by $a(u,w):=\langle Au,w \rangle$ for all $u\in V$ and $w\in W$.\par
The first question we must ask ourselves is whether the problem is well-posed. Note that we can not rely on the Lax-Milgram Lemma since here the bilinear operator is $a:V\times W\to \R$, with $V\neq W$ in general.
\begin{definition}\label{eqn:BNB}\marginpar{BNB conditions}
    A problem of the form \eqref{eqn:petrov-galerkin} is said to be \emph{well-posed} (in the sense of Hadamard) if the following two conditions are satisfied:
    \begin{itemize}
        \item for every $f\in W'$, there exists a unique $u\in V$ such that $Au=f$,
        \item $A$ is bounding, i.e. there exists $\alpha>0$ such that $\norm{u}_V \leq \tfrac{1}{\alpha}\norm{Au}_{W'}$ for every $u\in V$.
    \end{itemize}
\end{definition}
The previous two conditions are also known as \emph{BNB (Banach-Ne\v{c}as-Babuska) conditions} in the literature.
\begin{remark}\label{rmk:bnb_conditions}
	The BNB conditions can be summarized by saying that $A$ is required to be:
	\begin{enumerate}
		\item injective,
		\item surjective,
		\item bounding.
	\end{enumerate}
	Note that injectivity is implied by being bounding, since no element in $\ker A$, other than zero, can satisfy the bounding condition.
	
	1., 2. and 3. are all different aspects of the \emph{closed range theorem} and the \emph{open mapping theorem}, the latter of which is recalled below.
\end{remark}

\begin{theorem}[Open mapping theorem]
	Let $T: E \to F$ be a surjective linear continuous map between Banach spaces. Then $T$ is an open mapping, i.e., if $U \subset E$ is an open set, then also $T(U)$ is open.
\end{theorem}

\rev{not sure about the marginpar argument of the following theorem}
\begin{theorem}\label{thm:closed_range}\marginpar{Closed range theorem}
    Let $A\in\L(V;W')$ for $V,W$ Banach spaces. Then the following are equivalent:
    \begin{enumerate}
        \item $\im A = \overline{\im A}$,
%        \item A is bounding on $V \setminus \ker(A)$, i.e. there exists $\alpha >0$ such that $\norm{Au}_{W'}\geq\alpha\norm{u}_V$ for every $u\in V\setminus\ker(A)$.
        \item there exists $\alpha >0$ such that
        \begin{equation}
        	\forall w' \in \im A \, \exists u \in V \, \text{s.t. }\, Au = w' \, \text{and }\, \norm{Au}_{W'} = \norm{w'}_{W'} \ge \alpha\norm{u}_V.
        \end{equation}
    \end{enumerate}
\end{theorem}
%\begin{remark}
%	The second condition (also called \emph{bounding condition}) is equivalent to the following \emph{limsup condition}:
%	\begin{equation}\label{eqn:limsup condition}
%		\inf_{u\in V\setminus\ker(A)}\sup_{w\in W}\frac{\langle Au,w \rangle}{\norm{u}_V\norm{w}_W}\geq\alpha.
%	\end{equation}
%	This can be easily seen by remembering that $\sup_{w\in W}\frac{\langle Au,w \rangle}{\norm{w}_W}=\norm{Au}_{W'}$ by definition of operator norm, hence the limsup condition can be rewritten as
%	\begin{equation*}
%		\inf_{u\in V\setminus\ker(A)}\frac{\norm{Au}_{W'}}{\norm{u}_V}\geq\alpha,
%	\end{equation*}
%	which is clearly equivalent to the bounding condition.
%\end{remark}
\begin{proof}
    \hfill
    \begin{enumerate}[itemindent=25pt]
        \item[1. $\implies$ 2.] By hypothesis, $\im A$ is a closed linear subspace of $W'$, hence it is a Banach space. As a consequence, $A: V \to \im A$ is still a linear continuous operator, moreover it is surjective. Consider now the open unit ball $B_1(0_V)$: by the open mapping theorem, $A(B_1(0_V))\subseteq\im A$ is open in $\im A$. Therefore, being $A(0_V)=0_{W'}$ by linearity, there exists $\gamma>0$ such that $B_\gamma(0_{W'})\subseteq A(B_1(0_V))$.\par
        Pick any $\alpha < \gamma$ and let $w'\in \im A$. If $w' = 0_{W'}$, then it is sufficient to choose $u = 0_V$. Otherwise, $\overline{w'}:=\alpha\frac{w'}{\norm{w'}_{W'}}\in B_\gamma(0_{W'})$, which is contained in the image of $B_1(0_V)$. This in turn implies the existence of a vector $z\in B_1(0_V)$ such that $Az=\overline{w'}$.\par
        From this, we can easily conclude: take $u:=\frac{\norm{w'}_{W'}}{\alpha}z$, so that $Au=w'$. Then we have
        \begin{equation*}
            \norm{u}_V=\frac{\norm{w'}_{W'}}{\alpha}\norm{z}_V\leq\frac{\norm{w'}_{W'}}{\alpha}=\frac{\norm{Au}_{W'}}{\alpha}.
        \end{equation*}
        \item[2. $\implies$ 1.] We need to prove that $\im A$ is closed in $W'$. Take a Cauchy sequence $\{w_n'\}\subset \im A$ and let $w'\in W'$ be its limit. If we prove that $w'\in\im A$ we are done. For every $w_n'$, take $v_n\in V$ such that $Av_n=w_n'$. Then
        \begin{equation*}
            \norm{w_n'-w_m'}_{W'}=\norm{A(v_n-v_m)}_{W'}\geq\alpha\norm{v_n-v_m}_V,
        \end{equation*}
        hence also $\{v_n\}$ is Cauchy in $V$. If we call $v\in V$ the limit, by continuity of $A$ we must have $Av=w'$.
    \end{enumerate}
\end{proof}

It is useful to relate the bounding condition to other equivalent properties.
\begin{proposition}\marginpar{inf-sup condition}\label{prop:inf-sup_condition}
    Let $V, W$ be Banach spaces, with $W$ reflexive. Let $A\in\L(V;W')$. Then the following are equivalent:
    \begin{enumerate}
    	\item $A^T \in \L(W,V')$ is surjective.
    	\item $A$ is injective and $\im A = \overline{\im A}$.
    	\item $A$ is bounding.
    	\item The \emph{inf-sup condition} is satisfied:
		\begin{equation}\label{eqn:infsup_condition}
			\exists \, \alpha>0 \, \text{s.t. } \inf_{u\in V}\sup_{w\in W}\frac{\langle Au,w \rangle}{\norm{u}_V\norm{w}_W} \ge \alpha.
		\end{equation}    	
    \end{enumerate}
\end{proposition}
\begin{proof}
	The equivalence between 3. and 4. follows directly from the definition of operator norm:
	\[
	\norm{Au}_{W'} = \sup_{w\in W}\frac{\langle Au,w \rangle}{\norm{w}_W}.
	\]
	In fact, the inf-sup condition can be rewritten as
	\begin{equation*}
		\inf_{u\in V}\frac{\norm{Au}_{W'}}{\norm{u}_V}\geq\alpha,
	\end{equation*}
	which is clearly equivalent to the bounding condition.
		
	Let us prove that 1. $\iff$ 2. $\iff$ 3.
	\begin{enumerate}[itemindent=25pt]
		\item[1. $\implies$ 3.] Assume that this is not the case; then there exists a sequence $\{v_n\}\subset V$ with $\norm{v_n}_{V}=1$ and $\norm{Av_n}_{W'}<1/n$ for all $n\in\N$. Note that, since $\norm{v_n}_{V}=1$, for every $n$ there exists a functional $v_n'\in V'$ such that $|v_n'(v_n)|\geq 1/2$. Since $A^T$ is surjective, by the open mapping theorem there exists $\gamma > 0$ such that $A^T(B_1(0_W))$ contains $B_\gamma(0_{V'})$. Thus we get a sequence $\{w_n\}\subset B_1(0_W)$ such that $A^Tw_n=\gamma v_n'$ for all $n$. Finally, observe that this implies that
		\begin{equation*}
			|Av_n(w_n)|=|A^Tw_n(v_n)|=\gamma|v_n'(v_n)|\geq\gamma/2,
		\end{equation*}
		hence $\norm{Av_n}_{W'}\geq\gamma/2$, a contradiction.
		\item[3. $\implies$ 2.] This is a consequence of Theorem \ref{thm:closed_range}, together with Remark \ref{rmk:bnb_conditions}.
		\item[2. $\implies$ 1.] Being $\im A$ closed, a result from the book \emph{Functional Analysis} by H. Brezis assures that $\im (A^T) = (\ker A)^\perp$. Since $A$ is injective, $(\ker A)^\perp = V$.
		\rev{Leaving this last implication a bit sketchy until an "official" proof is chosen. I added that reference to Brezis because I couldn't come up with a self-contained proof for $2.\implies1.$ or $3.\implies1.$. Probably, mentioning this result may be useful also for the following example. Otherwise, is there some better idea?}
	\end{enumerate}
\end{proof}

In a similar fashion, if $V$ is reflexive as well, one can prove an analogous result for $A^T\in\L(W;V')$. In particular, we deduce that the following implications hold in general:
\begin{align}
	\text{$A$ surjective} &\implies \text{$A^T$ injective} \\
	\text{$A^T$ surjective} &\implies \text{$A$ injective}
\end{align}
Unfortunately, the converses fail, unless we deal with finite dimensional spaces. In fact, in finite dimensions, any subspace is closed by default, hence either $\im A$ and $\im (A^T)$ are closed.

\rev{In the teacher's notes there is also a relation between boundingness of $A$ and $A^T$. Expand on this. The main idea should be: if either $V$ and $W$ are reflexive, showing that both $A$ and $A^T$ are bounding is enough to prove the invertibility of $A$.}

\begin{example}
	To understand what might go wrong in infinite dimensions, consider the inclusion
	\[
	A: H_0^1(\Omega) \hookrightarrow L^2(\Omega) = \bigl(L^2(\Omega)\bigr)',
	\]
	i.e., $Au = u$ for all $u\in H_0^1(\Omega)$. Clearly, $\ker(A)=\{0\}$, but $A$ is not surjective: in fact, functions with a jump discontinuity can be in $L^2$ but not in $H^1_0$. Moreover, $\im A = H_0^1(\Omega)$ is dense in $L^2(\Omega)$ because $H_0^1$ is the closure of $C_0^\infty$, which is dense in $L^2$. Hence, $\im A\neq\overline{\im A}$.
	
	If we were in finite dimensions, $\ker A = \{0\}$ would be sufficient for $A$ to be invertible. However, let us look at this example from another point of view. The adjoint operator $A^T: L^2(\Omega) \to \bigl(H_0^1(\Omega)\bigr)'$ satisfies
	\[
	(A^T q)(u) = \int_{\Omega} uq \qquad \forall q \in L^2(\Omega).
	\]
	If $(A^T q)(u) = 0$ for every $u \in H_0^1(\Omega)$, then $q = 0$, thanks to standard results (fundamental lemma of calculus of variations). Hence, $\ker (A^T)$ is trivial. In finite dimensions, having $\ker (A^T) = \{0\}$ would imply that $A$ is surjective, which is not the case.
\end{example}

Now we want to rewrite the BNB conditions in the special case of Hilbert spaces.
\begin{lemma}\label{eqn:BNB for Hilbert}
    Let $V,W$ be Hilbert spaces, and let $A\in\L(V;W')$. Then the BNB conditions \eqref{eqn:BNB} are equivalent to the following: there exists an $\alpha>0$ such that
    \begin{enumerate}
        \item $\norm{Au}_{W'}\geq\alpha\norm{u}_V$,
        \item $\norm{A^Tw}_{V'}\geq\alpha\norm{w}_W$,
    \end{enumerate}
\end{lemma}
\begin{proof}
    Clearly 2. in \eqref{eqn:BNB} and 1. in \eqref{eqn:BNB for Hilbert} are equivalent. On the other hand, if for every $w\in W$ we define $w'\in W'$ such that $w'(z):=\tfrac{1}{\norm{w}_W}\langle w,z \rangle$, and we call $Au^*=w'$, then we have
    \begin{equation*}
        \norm{A^Tw}_{V'}=\sup_{u\in V}\frac{\langle A^Tw,u \rangle}{\norm{u}_V}\geq\frac{\langle A^Tw,u^* \rangle}{\norm{u^*}_V}=\frac{\langle Au^*,w \rangle}{\norm{u^*}_V}=\frac{1}{\norm{u^*}_V}\norm{w}_W.
    \end{equation*}
\end{proof}
\rev{The proof does not work, it needs to be fixed.}
\begin{remark}
    Using the closed range theorem and Remark \ref{eqn:infsup_condition}, 1. and 2. of the previous Lemma can be more suggestively written as
    \begin{align}
        \inf_{u\in V}\sup_{w\in W}\frac{\langle Au,w \rangle}{\norm{u}_V\norm{w}_W}&\geq\alpha \\
        \inf_{w\in W}\sup_{u\in V}\frac{\langle Au,w \rangle}{\norm{u}_V\norm{w}_W}&\geq\alpha.
    \end{align}
\end{remark}
Finally, we conclude this section with the following result, which we do not prove.
\begin{theorem}
    Using the notations above, if we assume that $V=W$ and that $A$ is $\alpha$-elliptic, then ``Lax-Milgram implies BNB'', i.e. if the hypothesis of Lemma \ref{lemma:lax-milgram} are satisfied, then \eqref{eqn:BNB} holds.
\end{theorem}


\section{Ceà's Lemma for Petrov-Galerkin method}
We consider the following slightly less general setting for Petrov-Galerkin: let $V,Q$ be Hilbert spaces, let $V_h\subset V$ and $Q_h\subset Q$ be some subspaces of finite dimension, and let $A\in\L(V;Q')$ be a linear continuous operator and let and $A_h\in\L(V_h;Q_h')$ be its discretization. It will prove useful in the following to introduce two \emph{projection} operators from the whole space to the discrete space $\Pi_h:V\to V_h$ and $P_h:Q\to Q_h$ defined as
\begin{equation*}
    \langle\Pi_h v,v_h\rangle=\langle v,v_h\rangle \ \forall\, v\in V,
\end{equation*}
and analogously for $P_h$.\par
Consider the discrete version of problem \eqref{eqn:petrov-galerkin}: given $f\in V'$, find $u_h\in V_h$ such that
\begin{equation}\label{eqn:discrete petrov-galerkin}\marginpar{Discrete Petrov-Galerkin}
    \langle Au_h, q_h \rangle= \langle f,q_h \rangle \ \forall q_h\in Q_h.
\end{equation}
\begin{remark}
    The following observation will be useful later on. Note that if $u$ is a solution to the continuous problem, then $\langle Au,q_h\rangle=\langle f,q_h\rangle$ for all $q_h\in Q_h\subset Q$. At the same time, given a solution $u_h$ to the discrete problem, we have $\langle Au_h,q_h\rangle=\langle f,q_h\rangle$ for all $q_h\in Q_h$. Subtracting the two gives us
    \begin{equation}\label{eqn:petrov-galerkin orthogonality}
        \langle A(u-u_h),q_h \rangle=0 
    \end{equation}
    for all $q_h\in Q_h$.
\end{remark}
As it has been discussed in the previous section, the solution exists and is unique if and only if the BNB conditions hold: in this setting we call them \emph{discrete infsup conditions} (or \emph{discrete BNB conditions}) to distinguish them from the infsup conditions for the continuous problem. Their formulation is however exactly the same:
\begin{align}\label{eqn:discrete infsup}\marginpar{Discrete infsup}
    \inf_{u_h\in V_h}\sup_{q_h\in Q_h}\frac{\langle Au_h,q_h \rangle}{\norm{u_h}_V\norm{q_h}_{Q_h}}&\geq\alpha_h>0 \\
    \inf_{q_h\in Q_h}\sup_{u_h\in V_h}\frac{\langle Au_h,q_h \rangle}{\norm{u_h}_V\norm{q_h}_{Q_h}}&\geq\alpha_h>0.
\end{align}
\begin{remark}
    We wrote $\alpha_h$ to stress the fact that this constant is generally different to the one in the continuous infsup, but $\alpha_h$ \emph{does not} depend on $h$.
\end{remark}
\begin{lemma}[Ceà's lemma for Petrov-Galerkin]
    Assume that the infsup conditions hold for both the continuous and the discrete problem. Let $u\in V$ be the solution to \eqref{eqn:petrov-galerkin} and let $u_h\in V_h$ be the solution to \eqref{eqn:discrete petrov-galerkin}. Then
    \begin{equation*}
        \norm{u-u_h}_V\leq\left( 1+\frac{\norm{A}}{\alpha_h} \right)\inf_{v_h\in V_h}\norm{u-v_h}_V.
    \end{equation*}
\end{lemma}
\begin{proof}
    From the discrete infsup, we have
    \begin{equation*}
        \alpha_h\leq \inf_{v_h\in V_h}\sup_{q_h\in Q_h}\frac{\langle Av_h,q_h \rangle}{\norm{v_h}_V\norm{q_h}_{Q_h}}= \inf_{v_h\in V_h}\frac{\norm{Av_h}_{*,Q_h}}{\norm{v_h}_V},
    \end{equation*}
    where $\norm{\cdot}_{*,Q_h}$ is the operator norm on the space $Q_h$ (i.e. the last equality follows by the definition of $\norm{\cdot}_{*,Q_h}$). From this follows that
    \begin{equation*}
        \frac{1}{\alpha_h}\norm{Av_h}_{*,Q_h}\geq\norm{v_h}_V \quad \forall\, v_h\in V_h.
    \end{equation*}
    In particular, it is also true that
    \begin{equation*}
        \frac{1}{\alpha_h}\norm{A(u_h-v_h)}_{*,Q_h}\geq\norm{u_h-v_h}_V \quad \forall\, v_h\in V_h,
    \end{equation*}
    since clearly $u_h-v_h\in V_h$. Now, we start doing some inequalities:
    \begin{align*}
        \norm{u-u_h}_V&\leq\norm{u-v_h}_V+\norm{v_h-u_h}_V\\
        &\leq\norm{u-v_h}_V+\frac{1}{\alpha_h}\norm{A(v_h-u_h)}_{*,Q_h}
    \end{align*}
    for all $v_h\in V_h$. At this point, to conclude it is enough to observe that
    \begin{align*}
        \norm{A(v_h-u_h)}_{*,Q_h}&=\sup_{q_h\in Q_h}\frac{\langle A(v_h-u_h),q_h \rangle}{\norm{q_h}_{Q_h}}\\
        &=\sup_{q_h\in Q_h}\frac{\langle A(v_h-u),q_h \rangle + \langle A(u-u_h),q_h\rangle}{\norm{q_h}_{Q_h}}\\
        &=\sup_{q_h\in Q_h}\frac{\langle A(v_h-u),q_h \rangle}{\norm{q_h}_{Q_h}}\\
        &=\norm{A(v_h-u)}_{*,Q_h},
    \end{align*}
    where we used \eqref{eqn:petrov-galerkin orthogonality} for the third equality. Plugging this into the inequality we obtained previously yields
    \begin{align*}
        \norm{u-u_h}_V&\leq\norm{u-v_h}_V+\frac{1}{\alpha_h}\norm{A(v_h-u)}_{*,Q_h}\\
        &\leq\left( 1+\frac{\norm{A}}{\alpha_h} \right)\norm{v_h-u}_V,
    \end{align*}
    for all $v_h\in V_h$. Taking the inf on $V_h$ gives the wanted inequality.
\end{proof}


\section{Mixed Problems}
Saddle point problems are a class of optimization problems characterized by the presence of both primal and dual variables, leading to solutions that are critical points of a Lagrangian function. These problems are common in various applications such as fluid dynamics, structural mechanics, and optimization.\par
The setting is the following: take  two Hilbert spaces $V,Q$ and two linear continuous operators
\begin{equation*}
    A:V\to V', \quad B:V\to Q'.
\end{equation*}
Take also $f\in V'$ and $g\in Q'$. The problem we aim to solve is to find $(u, p)\in V\times Q$ such that:
\begin{equation}\label{eqn:general mixed prb} \marginpar{General mixed problem}
    \begin{aligned}
        Au + B^Tp &= f \quad \text{in} \ V', \\
        Bu &= g  \quad \text{in} \ Q'.
    \end{aligned}
\end{equation}

\begin{remark}
    Assume that $g$ is in the image of $B$. Then there exists $u_g\in V$ such that $Bu_g=g$. Call $Z:=\ker{B}$; then the solution $u\in V$ can be written as $u=u_0 + u_g$ for some $u_0\in Z$. Substituting in \eqref{eqn:general mixed prb} we obtain:
    \begin{equation}\label{eqn:u0 prb}
        \begin{aligned}
            Au_0 + B^Tp &= f - Au_g=:\Tilde{f} , \\
            Bu_0 &= 0,
        \end{aligned}
    \end{equation}
    and the second equation is satisfied automatically by construction.
\end{remark}

By the previous Remark, we can therefore restrict ourselves to the study of the so-called ``$u_0$ problem'', i.e. finding $u_0\in Z$ such that
\begin{equation*}
    \langle Au_0,v_0 \rangle + \langle B^Tp,v_0 \rangle = \langle \Tilde{f},v_0 \rangle \quad \forall \ v_0\in Z
\end{equation*}
(this is just the first \eqref{eqn:u0 prb} written in terms of scalar products). Since $v_0\in Z$, the second term of the left-hand side vanishes, and thus we are left with the equation
\begin{equation*}
    \langle Au_0,v_0 \rangle = \langle \Tilde{f},v_0 \rangle \quad \forall \ v_0\in Z.
\end{equation*}
As we have said multiple times, the solution $u_0\in Z$ to this problem exists and is unique if and only if the BNB conditions hold in $Z$. Explicitly, we ask that there exists an $\alpha>0$ such that
\begin{align}\label{eqn:ell-ker conditions}\marginpar{Ell-ker conditions}
    \inf_{v_0\in Z}\sup_{u_0\in Z}\frac{\langle Au_0,v_0 \rangle}{\norm{u_0}_V\norm{v_0}_V}&\geq\alpha \\
    \inf_{u_0\in Z}\sup_{v_0\in Z}\frac{\langle Au_0,v_0 \rangle}{\norm{u_0}_V\norm{v_0}_V}&\geq\alpha.
\end{align}
In this setting, the BNB conditions are also called \emph{ell-ker conditions} (for ellipticity in the kernel).\par
If we manage to solve the $u_0$-problem, the solution to the original problem can be found by solving the \emph{$p$-problem}: find $p\in Q$ such that
\begin{align*}
    \langle B^Tp,v\rangle &= -\langle Au,v\rangle + \langle f,v\rangle\\
    &= -\langle Au_0,v\rangle + \langle\Tilde{f},v\rangle \quad \forall v\in V.
\end{align*}
Once again, existence and uniqueness of the solution are equivalent to the following infsup condition:
\begin{equation}\label{eqn:infsup for B}\marginpar{Infsup condition for $B$}
    \inf_{p\in Q}\sup_{v\in V}\frac{\langle Bv,p \rangle}{\norm{v}_V\norm{p}_Q}\geq\beta>0
\end{equation}
In conclusion, if conditions \eqref{eqn:ell-ker conditions} and \eqref{eqn:infsup for B} hold, then there is a unique solution $(u,p)\in V\times Q$ for the mixed problem \eqref{eqn:general mixed prb}.\par
Now we want to investigate whether we can estimate the norm of the solution in terms of $f,g$ and the infsup constants $\alpha$ and $\beta$. First observe that, from \eqref{eqn:ell-ker conditions} and the definition of $\Tilde{f}$, it follows immediately that
\begin{equation*}
    \norm{u_0}_V\leq\frac{1}{\alpha}\norm{\Tilde{f}}_{V'}\leq\frac{1}{\alpha}\left( 
\norm{f}_{V'}+\norm{A}_*\norm{u_g}_V \right).
\end{equation*}
Moreover, from \eqref{eqn:infsup for B} and the definition of $u_g$ we have
\begin{equation*}
    \norm{g}_{Q'}=\norm{Bu_g}_{Q'}\geq\beta\norm{u_g}_V.
\end{equation*}
Hence 
\begin{align*}
    \norm{u}_V &= \norm{u_0+u_g}_V\\
    &\leq\frac{1}{\alpha}\left(\norm{f}_{V'}+\norm{A}_*\norm{u_g}_V \right)+\frac{1}{\beta}\norm{g}_{Q'}\\
    &\leq \frac{1}{\alpha}\norm{f}_{V'}+\left( \frac{\alpha+\norm{A}_*}{\alpha\beta}\right)\norm{g}_{Q'}.
\end{align*}
An analogous computation can be made for $p,$ showing that
\begin{equation*}
    \norm{p}_Q\leq\frac{\norm{A}_*+1}{\beta}\norm{f}_{V'}+\frac{\norm{A}_*}{\beta}\left(\frac{\norm{A}_*+\alpha}{\alpha\beta}\right)\norm{g}_{Q'}.
\end{equation*}
\begin{remark}
    Call $\mathbb{V}:=V\times Q$ and
    \begin{equation*}
        \mathbb{A}:=\begin{bmatrix}
            A & B^T\\
            B & 0
        \end{bmatrix}\in\L(\mathbb{V};\mathbb{V}').
    \end{equation*}
    The problem \eqref{eqn:general mixed prb} can clearly be stated in terms of a singular variable $\psi:=(u,p)\in\mathbb{V}$ and the linear operator $\mathbb{A}$. It can be proved (but it is not easy) that the conditions \eqref{eqn:ell-ker conditions}, \eqref{eqn:infsup for B} are equivalent to the existence of a constant $\overline{\alpha}>0$ such that
    \begin{align}
        \inf_{\psi\in \mathbb{V}}\sup_{\theta\in \mathbb{V}}\frac{\langle A\psi,\theta \rangle}{\norm{\psi}_\mathbb{V}\norm{\theta}_\mathbb{V}}&\geq\overline{\alpha} \\
        \inf_{\theta\in \mathbb{V}}\sup_{\psi\in \mathbb{V}}\frac{\langle A\psi,\theta \rangle}{\norm{\psi}_\mathbb{V}\norm{\theta}_\mathbb{V}}&\geq\overline{\alpha}.
    \end{align}
    This is known in the literature as \emph{``Brezzi $\iff$ Babuska''}.
\end{remark}

\subsection{Some considerations on discrete mixed problems}
The aim of this paragraph is to give an idea as to why mixed problem are difficult. As we will see shortly, the main reason is that the ``goodness'' of the approximation for the velocity does not only depend on the approximation space $V_h$, but also depends on the choice of $Q_h$ (and the same can be said for the approximation of the pressure).\par
The main idea here is to proceed as usual, as if we wanted to prove Ceà's lemma. But first we introduce the following object, which is basically a discrete version of $\ker(B)$:
\begin{equation*}
    Z_h:=\{ v_h\in V_h \mid \langle Bv_h,q_h\rangle=0 \ \forall\,q_h\in Q_h \}.
\end{equation*}
\begin{remark}
    In general, it is not true that $Z_h\subset Z$. For example, we can consider $B=\dvg$ in dimension 1 (i.e. $B$ is the derivative operator), so that $Z$ is the set of (a.e.) constant function. However, if we consider $Q_h$ to be the set of piece-wise constant functions, then
    \begin{equation*}
        \langle Bu,q_h\rangle=0 \ \forall\, q_h\in Q_h \ \iff \ \int_T u'=0 \ \forall \, T \ \text{triangle in the triangulation}.
    \end{equation*}
    Clearly the set of functions with 0 derivative on each element of the triangulation is not contained in the set of constant functions, i.e. $Z_h\not\subset Z$.\par
    Also note that in general it is also not true that $Z\subset Z_h$.
\end{remark}
With the usual notations, we let $(u,p)$ be the solution to \eqref{eqn:general mixed prb} and let $(u_h,p_h)$ be the approximated solution. Then we can write
\begin{align*}
    \langle Au,v_h\rangle+\langle Bv_h,p\rangle&=\langle f,v_h\rangle \quad \forall\,v_h\in V_h\\
    \langle Au_h,v_h\rangle+\langle Bv_h,p_h\rangle&=\langle f,v_h\rangle \quad \forall\,v_h\in V_h,\\
\end{align*}
Subtracting the two gives
\begin{equation*}
    \langle A(u-u_h),v_h\rangle=-\langle Bv_h,p-p_h\rangle \quad \forall\, v_h\in V_h;
\end{equation*}
now observe that if we restrict ourselves to $v_h\in Z_h\subset V_h$ then $\langle Bv_h,p-p_h\rangle=\langle Bv_h,p-q_h\rangle$ for all $q_h\in Q_h$, so the previous equation can be rephrased as
\begin{equation}\label{eqn:mixed prb orthogonality}
    \langle A(u-u_h),v_h\rangle=-\langle Bv_h,p-q_h\rangle \quad \forall\, v_h\in Z_h \ \text{and} \ \forall\,q_h\in Q_h.
\end{equation}
As usual we can employ the discrete infsup condition \eqref{eqn:discrete infsup} to write
\begin{align*}
    \norm{u-u_h}_V&\leq\norm{u-v_h}_V+\norm{v_h-u_h}_V\\
    &\leq\norm{u-v_h}_V +\frac{1}{\alpha_h}\norm{A(v_h-u_h)}_{*,Z_h}
\end{align*}
for all $v_h\in Z_h$. Finally, we use \eqref{eqn:mixed prb orthogonality} in the following chain of (in)equalities:
\begin{align*}
    \norm{A(v_h-u_h)}_{*,Z_h} &= \sup_{z_h\in Z_h}\frac{\langle A(v_h-u_h),z_h\rangle}{\norm{z_h}_{V_h}}\\
    &= \sup_{z_h\in Z_h}\frac{\langle A(v_h-u),z_h\rangle + \langle A(u-u_h),z_h\rangle}{\norm{z_h}_{V_h}}\\
    &= \sup_{z_h\in Z_h}\frac{\langle A(v_h-u),z_h\rangle - \langle Bz_h,p-q_h\rangle}{\norm{z_h}_{V_h}}\\
    &= \norm{A(v_h-u)}_{*,Z_h}+\norm{B^T(q_h-p)}_{*,Z_h}\\
    &\leq \norm{A}_*\norm{v_h-u}_V+\norm{B}_*\norm{q_h-p}
\end{align*}
for all $v_h\in Z_h$ and all $q_h\in Q_h$. Taking the inf on $Z_h$ and $Q_h$ gives a similar result to the classic Ceà's lemma for the velocity:
\begin{equation*}
    \norm{u-u_h}_V\leq \left( 1+\frac{\norm{A}_*}{\alpha_h}\right)\inf_{v_h\in Z_h}\norm{u-v_h}_V+\frac{\norm{B}_*}{\alpha_h}\inf_{q_h\in Q_h}\norm{p-q_h};
\end{equation*}
similarly, we can obtain the following inequality for the pressure:
\begin{equation*}
    \norm{p-p_h}_Q\leq \left( 1+\frac{\norm{B}_*}{\beta_h}\right)\inf_{q_h\in Q_h}\norm{p-q_h}_Q+\frac{\norm{A}_*}{\beta_h}\inf_{v_h\in Z_h}\norm{u-v_h}.
\end{equation*}



\subsection{An example: the mixed Laplacian}
Another example is the mixed formulation of the Laplacian:
\begin{equation}
    \begin{aligned}
        a(u, v) + b(v, p) &= f(v) \quad \forall v \in V, \\
        b(u, q) &= g(q) \quad \forall q \in Q.
    \end{aligned}
\end{equation}

\rev{complete this}

%\part{Appendix}

%\input{Files/6-lab}

%****************************************************************
% Bibliography
%****************************************************************

\begin{thebibliography}{9}
\bibitem{ciarlet78}
P.G. Ciarlet, \emph{The finite element method for elliptic problems}. Number v. 4 in Studies in mathematics and its applications. North-Holland Pub. Co. Sole distributors for the U.S.A. and Canada, Elsevier North-Holland, Amsterdam New York, 1978. ISBN 978-0-444-85028-7.

\bibitem{eg04}
A. Ern, J.L. Guermond, \emph{Theory and Practice of Finite Elements}. Volume 159 of Applied Mathematical Sciences. Springer New York, New York, NY, 2004. ISBN 978-1-4419-1918-2 978-1-4757-4355-5. \href{http://link.springer.com/10.1007/978-1-4757-4355-5}{URL}, \href{https://doi.org/10.1007/978-1-4757-4355-5}{doi}.

\bibitem{bs10}
S.C. Brenner, L.R. Scott, \emph{The mathematical theory of finite element methods}. Number 15 in Texts in applied mathematics. Springer, New York, NY, 3. ed. edition, 2010. ISBN 978-0-387-75933-3 978-1-4419-2611-1.

\bibitem{bbf13}
D. Boffi, F. Brezzi, M. Fortin, \emph{Mixed Finite Element Methods and Applications}. Volume 44 of Springer Series in Computational Mathematics. Springer Berlin Heidelberg, Berlin, Heidelberg, 2013. ISBN 978-3-642-36518-8 978-3-642-36519-5. \href{http://link.springer.com/10.1007/978-3-642-36519-5}{URL}, \href{https://doi.org/10.1007/978-3-642-36519-5}{doi}.

\bibitem{bcms04}
F. Brezzi, B. Cockburn, L.D. Marini, E. S\"uli,
\emph{Stabilization mechanisms in discontinuous Galerkin finite element methods}. Volume 195 of Computer Methods in Applied Mechanics and Engineering, 2006. \href{https://doi.org/10.1016/j.cma.2005.06.015}{doi}.

\bibitem{qv94}
A. Quarteroni, A. Valli, \emph{Numerical Approximation of Partial Differential Equations}. Scientific Computation. Clarendon Press, Philadelphia, 1994. ISBN 978-0-19-852448-2.

\bibitem{fract_sob}
E. Di Nezza, G. Palatucci, E. Valdinoci,
\emph{Hitchhiker’s guide to the fractional Sobolev spaces}. 2011.
\href{https://arxiv.org/abs/1104.4345v3}{arXiv}, \href{https://doi.org/10.48550/arXiv.1104.4345}{doi}.

\bibitem{kelly83}
D.W. Kelly, J.P. De S. R. Gago, O.C. Zienkiewicz, I. Babu\v{s}ka. \emph{A posteriori error analysis and adaptive processes in the finite element method: Part I–error analysis}. International Journal for Numerical Methods in Engineering, 1983. \href{https://doi.org/10.1002/nme.1620191103}{doi}.
\end{thebibliography}

\end{document}
